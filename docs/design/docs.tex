\documentclass[11pt,a4paper]{report}
\usepackage{color}
\usepackage{ifthen}
\usepackage{makeidx}
\usepackage{ifpdf}
\usepackage[headings]{fullpage}
\usepackage{listings}
\lstset{language=Java,breaklines=true}
\ifpdf \usepackage[pdftex, pdfpagemode={UseOutlines},bookmarks,colorlinks,linkcolor={blue},plainpages=false,pdfpagelabels,citecolor={red},breaklinks=true]{hyperref}
  \usepackage[pdftex]{graphicx}
  \pdfcompresslevel=9
  \DeclareGraphicsRule{*}{mps}{*}{}
\else
  \usepackage[dvips]{graphicx}
\fi

\newcommand{\entityintro}[3]{%
  \hbox to \hsize{%
    \vbox{%
      \hbox to .2in{}%
    }%
    {\bf  #1}%
    \dotfill\pageref{#2}%
  }
  \makebox[\hsize]{%
    \parbox{.4in}{}%
    \parbox[l]{5in}{%
      \vspace{1mm}%
      #3%
      \vspace{1mm}%
    }%
  }%
}
\newcommand{\divideents}[1]{\vskip -1em\indent\rule{2in}{.5mm}}
\newcommand{\refdefined}[1]{
\expandafter\ifx\csname r@#1\endcsname\relax
\relax\else
{$($in \ref{#1}, page \pageref{#1}$)$}\fi}
\date{\today}
\title{OSIP - Dokumentation\bigskip\\ \Large OPC UA Simulator for Industrial Plants}
\chardef\textbackslash=`\\
\makeindex
\begin{document}
\maketitle
\setcounter{page}{25}
\sloppy
\addtocontents{toc}{\protect\markboth{Contents}{Contents}}
\tableofcontents
\chapter{Package edu.kit.pse.osip.monitoring.controller}{
\label{edu.kit.pse.osip.monitoring.controller}\section{\label{edu.kit.pse.osip.monitoring.controller.MonitoringViewInterface}\index{MonitoringViewInterface@\textit{ MonitoringViewInterface}}Interface MonitoringViewInterface}{
\rule[1em]{\hsize}{4pt}\vskip -1em
\vskip .1in 
Provides abstraction from the UI and a single interface to configure the monitoring view.\vskip .1in 
\subsection{Declaration}{
\begin{lstlisting}[frame=none]
public interface MonitoringViewInterface
\end{lstlisting}
\subsection{All known subinterfaces}{MonitoringViewFacade\small{\refdefined{edu.kit.pse.osip.monitoring.view.dashboard.MonitoringViewFacade}}}
\subsection{All classes known to implement interface}{MonitoringViewFacade\small{\refdefined{edu.kit.pse.osip.monitoring.view.dashboard.MonitoringViewFacade}}}
\subsection{Methods}{
\rule[1em]{\hsize}{2pt}\vskip -2em
\vskip -2em
\begin{itemize}
\item{ 
\index{setFillLevelProgressionEnabled(TankSelector, boolean)}
{\bf  setFillLevelProgressionEnabled}\\
\begin{lstlisting}[frame=none]
void setFillLevelProgressionEnabled(edu.kit.pse.osip.core.model.base.TankSelector tank,boolean progressionEnabled)\end{lstlisting} %end signature
\begin{itemize}
\item{
{\bf  Description}

Enables or disables the logging of a fill level progression for a specified tank.
}
\item{
{\bf  Parameters}
  \begin{itemize}
   \item{
\texttt{tank} -- The tank whose fill level progression is logged or not.}
   \item{
\texttt{progressionEnabled} -- true if the fill level progression should be logged and false otherwise.}
  \end{itemize}
}%end item
\end{itemize}
}%end item
\divideents{setMenuAboutButtonHandler}
\item{ 
\index{setMenuAboutButtonHandler(MenuAboutButtonHandler)}
{\bf  setMenuAboutButtonHandler}\\
\begin{lstlisting}[frame=none]
void setMenuAboutButtonHandler(MenuAboutButtonHandler handler)\end{lstlisting} %end signature
\begin{itemize}
\item{
{\bf  Description}

Sets the handler for the about menu button.
}
\item{
{\bf  Parameters}
  \begin{itemize}
   \item{
\texttt{handler} -- The handler for the about menu button.}
  \end{itemize}
}%end item
\end{itemize}
}%end item
\divideents{setMenuHelpButtonHandler}
\item{ 
\index{setMenuHelpButtonHandler(MenuHelpButtonHandler)}
{\bf  setMenuHelpButtonHandler}\\
\begin{lstlisting}[frame=none]
void setMenuHelpButtonHandler(MenuHelpButtonHandler handler)\end{lstlisting} %end signature
\begin{itemize}
\item{
{\bf  Description}

Sets the handler for the help menu button.
}
\item{
{\bf  Parameters}
  \begin{itemize}
   \item{
\texttt{handler} -- The handler handles a click on the help menu button.}
  \end{itemize}
}%end item
\end{itemize}
}%end item
\divideents{setMenuSettingsButtonHandler}
\item{ 
\index{setMenuSettingsButtonHandler(MenuSettingsButtonHandler)}
{\bf  setMenuSettingsButtonHandler}\\
\begin{lstlisting}[frame=none]
void setMenuSettingsButtonHandler(MenuSettingsButtonHandler handler)\end{lstlisting} %end signature
\begin{itemize}
\item{
{\bf  Description}

Sets the handler for the settings menu button.
}
\item{
{\bf  Parameters}
  \begin{itemize}
   \item{
\texttt{handler} -- The handler for the settings menu button.}
  \end{itemize}
}%end item
\end{itemize}
}%end item
\divideents{setOverflowAlarmEnabled}
\item{ 
\index{setOverflowAlarmEnabled(TankSelector, boolean)}
{\bf  setOverflowAlarmEnabled}\\
\begin{lstlisting}[frame=none]
void setOverflowAlarmEnabled(edu.kit.pse.osip.core.model.base.TankSelector tank,boolean alarmEnabled)\end{lstlisting} %end signature
\begin{itemize}
\item{
{\bf  Description}

Enables or disables the overflow alarm for a specified tank.
}
\item{
{\bf  Parameters}
  \begin{itemize}
   \item{
\texttt{tank} -- The tank whose overflow alarm will be enabled or disabled.}
   \item{
\texttt{alarmEnabled} -- true if the alarm should be enabled and false otherwise.}
  \end{itemize}
}%end item
\end{itemize}
}%end item
\divideents{setTemperatureOverflowAlarmEnabled}
\item{ 
\index{setTemperatureOverflowAlarmEnabled(TankSelector, boolean)}
{\bf  setTemperatureOverflowAlarmEnabled}\\
\begin{lstlisting}[frame=none]
void setTemperatureOverflowAlarmEnabled(edu.kit.pse.osip.core.model.base.TankSelector tank,boolean alarmEnabled)\end{lstlisting} %end signature
\begin{itemize}
\item{
{\bf  Description}

Enables or disables the alarm when the temperature becomes too high for a specified tank.
}
\item{
{\bf  Parameters}
  \begin{itemize}
   \item{
\texttt{tank} -- The tank which alarm will be turned off or on.}
   \item{
\texttt{alarmEnabled} -- true when the alarm should be enabled. false otherwise.}
  \end{itemize}
}%end item
\end{itemize}
}%end item
\divideents{setTemperatureProgressionEnabled}
\item{ 
\index{setTemperatureProgressionEnabled(TankSelector, boolean)}
{\bf  setTemperatureProgressionEnabled}\\
\begin{lstlisting}[frame=none]
void setTemperatureProgressionEnabled(edu.kit.pse.osip.core.model.base.TankSelector tank,boolean progressionEnabled)\end{lstlisting} %end signature
\begin{itemize}
\item{
{\bf  Description}

Enables or disables the logging of the temperature progression of a specified tank.
}
\item{
{\bf  Parameters}
  \begin{itemize}
   \item{
\texttt{tank} -- The tank whose temperature progression is logged or not.}
   \item{
\texttt{progressionEnabled} -- true if the temperature progression should be logged and false otherwise.}
  \end{itemize}
}%end item
\end{itemize}
}%end item
\divideents{setTemperatureUnderflowAlarmEnabled}
\item{ 
\index{setTemperatureUnderflowAlarmEnabled(TankSelector, boolean)}
{\bf  setTemperatureUnderflowAlarmEnabled}\\
\begin{lstlisting}[frame=none]
void setTemperatureUnderflowAlarmEnabled(edu.kit.pse.osip.core.model.base.TankSelector tank,boolean alarmEnabled)\end{lstlisting} %end signature
\begin{itemize}
\item{
{\bf  Description}

Enables or disables the temperature alarm when it becomes too low for a specified tank.
}
\item{
{\bf  Parameters}
  \begin{itemize}
   \item{
\texttt{tank} -- The tank which alarm will be turned off or on.}
   \item{
\texttt{alarmEnabled} -- true when the alarm should be enabled. false otherwise.}
  \end{itemize}
}%end item
\end{itemize}
}%end item
\divideents{setUnderflowAlarmEnabled}
\item{ 
\index{setUnderflowAlarmEnabled(TankSelector, boolean)}
{\bf  setUnderflowAlarmEnabled}\\
\begin{lstlisting}[frame=none]
void setUnderflowAlarmEnabled(edu.kit.pse.osip.core.model.base.TankSelector tank,boolean alarmEnabled)\end{lstlisting} %end signature
\begin{itemize}
\item{
{\bf  Description}

Enables or disables the underflow alarm for a specified tank.
}
\item{
{\bf  Parameters}
  \begin{itemize}
   \item{
\texttt{tank} -- The tank whose underflow alarm will be enabled or disabled.}
   \item{
\texttt{alarmEnabled} -- true if the alarm should be enabled and false otherwise.}
  \end{itemize}
}%end item
\end{itemize}
}%end item
\divideents{showMonitoringView}
\item{ 
\index{showMonitoringView(javafx.stage.Stage)}
{\bf  showMonitoringView}\\
\begin{lstlisting}[frame=none]
void showMonitoringView(javafx.stage.Stage stage)\end{lstlisting} %end signature
\begin{itemize}
\item{
{\bf  Description}

Shows the monitoring view to the user.
}
\item{
{\bf  Parameters}
  \begin{itemize}
   \item{
\texttt{stage} -- The stage used for displaying controls.}
  \end{itemize}
}%end item
\end{itemize}
}%end item
\end{itemize}
}
}
\section{\label{edu.kit.pse.osip.monitoring.controller.SettingsViewInterface}\index{SettingsViewInterface@\textit{ SettingsViewInterface}}Interface SettingsViewInterface}{
\rule[1em]{\hsize}{4pt}\vskip -1em
\vskip .1in 
Provides abstraction from the view and a single interface for accessing the user-set settings.\vskip .1in 
\subsection{Declaration}{
\begin{lstlisting}[frame=none]
public interface SettingsViewInterface
\end{lstlisting}
\subsection{All known subinterfaces}{SettingsViewFacade\small{\refdefined{edu.kit.pse.osip.monitoring.view.settings.SettingsViewFacade}}}
\subsection{All classes known to implement interface}{SettingsViewFacade\small{\refdefined{edu.kit.pse.osip.monitoring.view.settings.SettingsViewFacade}}}
\subsection{Methods}{
\rule[1em]{\hsize}{2pt}\vskip -2em
\vskip -2em
\begin{itemize}
\item{ 
\index{getServerHostname()}
{\bf  getServerHostname}\\
\begin{lstlisting}[frame=none]
java.lang.String getServerHostname()\end{lstlisting} %end signature
\begin{itemize}
\item{
{\bf  Description}

Returns the host name for the servers.
}
\item{{\bf  Returns} -- 
the host name for the servers. 
}%end item
\end{itemize}
}%end item
\divideents{getServerPort}
\item{ 
\index{getServerPort(int)}
{\bf  getServerPort}\\
\begin{lstlisting}[frame=none]
int getServerPort(int serverNumber)\end{lstlisting} %end signature
\begin{itemize}
\item{
{\bf  Description}

Returns the port number for a specified server.
}
\item{
{\bf  Parameters}
  \begin{itemize}
   \item{
\texttt{serverNumber} -- Number of the server whose port number will be returned.}
  \end{itemize}
}%end item
\item{{\bf  Returns} -- 
the port number of a specified server. 
}%end item
\end{itemize}
}%end item
\divideents{getTime}
\item{ 
\index{getTime()}
{\bf  getTime}\\
\begin{lstlisting}[frame=none]
int getTime()\end{lstlisting} %end signature
\begin{itemize}
\item{
{\bf  Description}

Returns the update time interval.
}
\item{{\bf  Returns} -- 
the update time interval. 
}%end item
\end{itemize}
}%end item
\divideents{hideSettingsWindow}
\item{ 
\index{hideSettingsWindow()}
{\bf  hideSettingsWindow}\\
\begin{lstlisting}[frame=none]
void hideSettingsWindow()\end{lstlisting} %end signature
\begin{itemize}
\item{
{\bf  Description}

Hides the settings view.
}
\end{itemize}
}%end item
\divideents{isFillLevelProgressionEnabled}
\item{ 
\index{isFillLevelProgressionEnabled(TankSelector)}
{\bf  isFillLevelProgressionEnabled}\\
\begin{lstlisting}[frame=none]
boolean isFillLevelProgressionEnabled(edu.kit.pse.osip.core.model.base.TankSelector tank)\end{lstlisting} %end signature
\begin{itemize}
\item{
{\bf  Description}

Returns true if the logging of the fill level progression should be enabled for a specified tank and false otherwise.
}
\item{
{\bf  Parameters}
  \begin{itemize}
   \item{
\texttt{tank} -- The tank whose fill level progression should be enabled or disabled.}
  \end{itemize}
}%end item
\item{{\bf  Returns} -- 
true if the logging of the fill level progression should be enabled for the specified tank or false otherwise. 
}%end item
\end{itemize}
}%end item
\divideents{isOverflowAlarmEnabled}
\item{ 
\index{isOverflowAlarmEnabled(TankSelector)}
{\bf  isOverflowAlarmEnabled}\\
\begin{lstlisting}[frame=none]
boolean isOverflowAlarmEnabled(edu.kit.pse.osip.core.model.base.TankSelector tank)\end{lstlisting} %end signature
\begin{itemize}
\item{
{\bf  Description}

Returns true if the overflow alarm should be enabled for a specified tank and false otherwise.
}
\item{
{\bf  Parameters}
  \begin{itemize}
   \item{
\texttt{tank} -- The tank whose overflow alarm should be enabled or disabled.}
  \end{itemize}
}%end item
\item{{\bf  Returns} -- 
true if the overflow alarm should be enabled for a specified tank. false otherwise. 
}%end item
\end{itemize}
}%end item
\divideents{isTemperatureOverflowAlarmEnabled}
\item{ 
\index{isTemperatureOverflowAlarmEnabled(TankSelector)}
{\bf  isTemperatureOverflowAlarmEnabled}\\
\begin{lstlisting}[frame=none]
boolean isTemperatureOverflowAlarmEnabled(edu.kit.pse.osip.core.model.base.TankSelector tank)\end{lstlisting} %end signature
\begin{itemize}
\item{
{\bf  Description}

Returns true if the overflow alarm for the temperature should be enabled for a specified tank and false otherwise.
}
\item{
{\bf  Parameters}
  \begin{itemize}
   \item{
\texttt{tank} -- The specified tank.}
  \end{itemize}
}%end item
\item{{\bf  Returns} -- 
true if the overflow alarm for the temperature should be enabled for a specified tank and false otherwise. 
}%end item
\end{itemize}
}%end item
\divideents{isTemperatureProgressionEnabled}
\item{ 
\index{isTemperatureProgressionEnabled(TankSelector)}
{\bf  isTemperatureProgressionEnabled}\\
\begin{lstlisting}[frame=none]
boolean isTemperatureProgressionEnabled(edu.kit.pse.osip.core.model.base.TankSelector tank)\end{lstlisting} %end signature
\begin{itemize}
\item{
{\bf  Description}

Returns true if the logging of the temperature progression should be enabled for the specified tank and false otherwise.
}
\item{
{\bf  Parameters}
  \begin{itemize}
   \item{
\texttt{tank} -- The tank whose temperature progression will be set.}
  \end{itemize}
}%end item
\item{{\bf  Returns} -- 
true if the logging of the temperature progression should be enabled for the specified tank. false otherwise. 
}%end item
\end{itemize}
}%end item
\divideents{isTemperatureUnderflowAlarmEnabled}
\item{ 
\index{isTemperatureUnderflowAlarmEnabled(TankSelector)}
{\bf  isTemperatureUnderflowAlarmEnabled}\\
\begin{lstlisting}[frame=none]
boolean isTemperatureUnderflowAlarmEnabled(edu.kit.pse.osip.core.model.base.TankSelector tank)\end{lstlisting} %end signature
\begin{itemize}
\item{
{\bf  Description}

Returns true if the underflow alarm for the temperature should be enabled for a specified tank and false otherwise.
}
\item{
{\bf  Parameters}
  \begin{itemize}
   \item{
\texttt{tank} -- The specified tank.}
  \end{itemize}
}%end item
\item{{\bf  Returns} -- 
true if the underflow alarm for the temperature should be enabled for a specified tank and false otherwise. 
}%end item
\end{itemize}
}%end item
\divideents{isUnderflowAlarmEnabled}
\item{ 
\index{isUnderflowAlarmEnabled(TankSelector)}
{\bf  isUnderflowAlarmEnabled}\\
\begin{lstlisting}[frame=none]
boolean isUnderflowAlarmEnabled(edu.kit.pse.osip.core.model.base.TankSelector tank)\end{lstlisting} %end signature
\begin{itemize}
\item{
{\bf  Description}

Returns true if the underflow alarm should be enabled for a specified tank and false otherwise.
}
\item{
{\bf  Parameters}
  \begin{itemize}
   \item{
\texttt{tank} -- Tank whose value will be looked up.}
  \end{itemize}
}%end item
\item{{\bf  Returns} -- 
true if the underflow alarm should be enabled for the specified tank. false otherwise. 
}%end item
\end{itemize}
}%end item
\divideents{setSettingsCancelButtonHandler}
\item{ 
\index{setSettingsCancelButtonHandler(SettingsCancelButtonHandler)}
{\bf  setSettingsCancelButtonHandler}\\
\begin{lstlisting}[frame=none]
void setSettingsCancelButtonHandler(SettingsCancelButtonHandler handler)\end{lstlisting} %end signature
\begin{itemize}
\item{
{\bf  Description}

Sets the handler for the cancel button in the settings view.
}
\item{
{\bf  Parameters}
  \begin{itemize}
   \item{
\texttt{handler} -- The handler for the cancel button in the settings view.}
  \end{itemize}
}%end item
\end{itemize}
}%end item
\divideents{setSettingsSaveButtonHandler}
\item{ 
\index{setSettingsSaveButtonHandler(SettingsSaveButtonHandler)}
{\bf  setSettingsSaveButtonHandler}\\
\begin{lstlisting}[frame=none]
void setSettingsSaveButtonHandler(SettingsSaveButtonHandler handler)\end{lstlisting} %end signature
\begin{itemize}
\item{
{\bf  Description}

Sets the handler for the save button in the settings view.
}
\item{
{\bf  Parameters}
  \begin{itemize}
   \item{
\texttt{handler} -- The handler for the save button in the settings view.}
  \end{itemize}
}%end item
\end{itemize}
}%end item
\divideents{showSettingsWindow}
\item{ 
\index{showSettingsWindow(ClientSettingsWrapper)}
{\bf  showSettingsWindow}\\
\begin{lstlisting}[frame=none]
void showSettingsWindow(edu.kit.pse.osip.core.io.files.ClientSettingsWrapper currentSettings)\end{lstlisting} %end signature
\begin{itemize}
\item{
{\bf  Description}

Shows the settings view with the current settings.
}
\item{
{\bf  Parameters}
  \begin{itemize}
   \item{
\texttt{currentSettings} -- The current settings used for displaying them.}
  \end{itemize}
}%end item
\end{itemize}
}%end item
\end{itemize}
}
}
\section{\label{edu.kit.pse.osip.monitoring.controller.AbstractTankClient}\index{AbstractTankClient}Class AbstractTankClient}{
\rule[1em]{\hsize}{4pt}\vskip -1em
\vskip .1in 
Client for retreiving generic tank information from OPC UA. Allows to subscribe to a concrete value on the server without having to think about NodeIds and namespaces.\vskip .1in 
\subsection{Declaration}{
\begin{lstlisting}[frame=none]
public abstract class AbstractTankClient
 extends edu.kit.pse.osip.core.opcua.client.UAClientWrapper\end{lstlisting}
\subsection{All known subclasses}{TankClient\small{\refdefined{edu.kit.pse.osip.monitoring.controller.TankClient}}, MixTankClient\small{\refdefined{edu.kit.pse.osip.monitoring.controller.MixTankClient}}}
\subsection{Constructors}{
\rule[1em]{\hsize}{2pt}\vskip -2em
\vskip -2em
\begin{itemize}
\item{ 
\index{AbstractTankClient(String, String)}
{\bf  AbstractTankClient}\\
\begin{lstlisting}[frame=none]
public AbstractTankClient(java.lang.String serverUrl,java.lang.String namespace)\end{lstlisting} %end signature
\begin{itemize}
\item{
{\bf  Description}

Creates a new tank client
}
\item{
{\bf  Parameters}
  \begin{itemize}
   \item{
\texttt{serverUrl} -- The url of the server}
   \item{
\texttt{namespace} -- The namespace to use}
  \end{itemize}
}%end item
\end{itemize}
}%end item
\end{itemize}
}
\subsection{Methods}{
\rule[1em]{\hsize}{2pt}\vskip -2em
\vskip -2em
\begin{itemize}
\item{ 
\index{subscribeColor(int, IntReceivedListener)}
{\bf  subscribeColor}\\
\begin{lstlisting}[frame=none]
public final void subscribeColor(int interval,edu.kit.pse.osip.core.opcua.client.IntReceivedListener listener)\end{lstlisting} %end signature
\begin{itemize}
\item{
{\bf  Description}

Subscribes the color of the tank at the given interval. Can be called again with the same listener to change interval.
}
\item{
{\bf  Parameters}
  \begin{itemize}
   \item{
\texttt{interval} -- The interval to use when subscribing to the value}
   \item{
\texttt{listener} -- The callback function to be called as soon as the subscribed value was received}
  \end{itemize}
}%end item
\end{itemize}
}%end item
\divideents{subscribeFillLevel}
\item{ 
\index{subscribeFillLevel(int, FloatReceivedListener)}
{\bf  subscribeFillLevel}\\
\begin{lstlisting}[frame=none]
public final void subscribeFillLevel(int interval,edu.kit.pse.osip.core.opcua.client.FloatReceivedListener listener)\end{lstlisting} %end signature
\begin{itemize}
\item{
{\bf  Description}

Subscribes the fill level of the tank at the given interval. Can be called again with the same listener to change interval.
}
\item{
{\bf  Parameters}
  \begin{itemize}
   \item{
\texttt{interval} -- The interval to use when subscribing to the value}
   \item{
\texttt{listener} -- The callback function to be called as soon as the subscribed value was received}
  \end{itemize}
}%end item
\end{itemize}
}%end item
\divideents{subscribeOutputFlowRate}
\item{ 
\index{subscribeOutputFlowRate(int, FloatReceivedListener)}
{\bf  subscribeOutputFlowRate}\\
\begin{lstlisting}[frame=none]
public final void subscribeOutputFlowRate(int interval,edu.kit.pse.osip.core.opcua.client.FloatReceivedListener listener)\end{lstlisting} %end signature
\begin{itemize}
\item{
{\bf  Description}

Subscribes the output flow rate of the tank at the given interval. Can be called again with the same listener to change interval.
}
\item{
{\bf  Parameters}
  \begin{itemize}
   \item{
\texttt{interval} -- The interval to use when subscribing to the value}
   \item{
\texttt{listener} -- The callback function to be called as soon as the subscribed value was received}
  \end{itemize}
}%end item
\end{itemize}
}%end item
\divideents{subscribeOverflowSensor}
\item{ 
\index{subscribeOverflowSensor(int, BooleanReceivedListener)}
{\bf  subscribeOverflowSensor}\\
\begin{lstlisting}[frame=none]
public final void subscribeOverflowSensor(int interval,edu.kit.pse.osip.core.opcua.client.BooleanReceivedListener listener)\end{lstlisting} %end signature
\begin{itemize}
\item{
{\bf  Description}

Subscribes the overflow sensor of the tank at the given interval. Can be called again with the same listener to change interval.
}
\item{
{\bf  Parameters}
  \begin{itemize}
   \item{
\texttt{interval} -- The interval to use when subscribing to the value}
   \item{
\texttt{listener} -- The callback function to be called as soon as the subscribed value was received}
  \end{itemize}
}%end item
\end{itemize}
}%end item
\divideents{subscribeOverheatSensor}
\item{ 
\index{subscribeOverheatSensor(int, BooleanReceivedListener)}
{\bf  subscribeOverheatSensor}\\
\begin{lstlisting}[frame=none]
public final void subscribeOverheatSensor(int interval,edu.kit.pse.osip.core.opcua.client.BooleanReceivedListener listener)\end{lstlisting} %end signature
\begin{itemize}
\item{
{\bf  Description}

Subscribes the overheat sensor of the tank at the given interval. Can be called again with the same listener to change interval.
}
\item{
{\bf  Parameters}
  \begin{itemize}
   \item{
\texttt{interval} -- The interval to use when subscribing to the value}
   \item{
\texttt{listener} -- The callback function to be called as soon as the subscribed value was received}
  \end{itemize}
}%end item
\end{itemize}
}%end item
\divideents{subscribeTemperature}
\item{ 
\index{subscribeTemperature(int, FloatReceivedListener)}
{\bf  subscribeTemperature}\\
\begin{lstlisting}[frame=none]
public final void subscribeTemperature(int interval,edu.kit.pse.osip.core.opcua.client.FloatReceivedListener listener)\end{lstlisting} %end signature
\begin{itemize}
\item{
{\bf  Description}

Subscribes the temperature of the tank at the given interval. Can be called again with the same listener to change interval.
}
\item{
{\bf  Parameters}
  \begin{itemize}
   \item{
\texttt{interval} -- The interval to use when subscribing to the value}
   \item{
\texttt{listener} -- The callback function to be called as soon as the subscribed value was received}
  \end{itemize}
}%end item
\end{itemize}
}%end item
\divideents{subscribeUndercoolSensor}
\item{ 
\index{subscribeUndercoolSensor(int, BooleanReceivedListener)}
{\bf  subscribeUndercoolSensor}\\
\begin{lstlisting}[frame=none]
public final void subscribeUndercoolSensor(int interval,edu.kit.pse.osip.core.opcua.client.BooleanReceivedListener listener)\end{lstlisting} %end signature
\begin{itemize}
\item{
{\bf  Description}

Subscribes the undercool sensor of the tank at the given interval. Can be called again with the same listener to change interval.
}
\item{
{\bf  Parameters}
  \begin{itemize}
   \item{
\texttt{interval} -- The interval to use when subscribing to the value}
   \item{
\texttt{listener} -- The callback function to be called as soon as the subscribed value was received}
  \end{itemize}
}%end item
\end{itemize}
}%end item
\divideents{subscribeUnderflowSensor}
\item{ 
\index{subscribeUnderflowSensor(int, BooleanReceivedListener)}
{\bf  subscribeUnderflowSensor}\\
\begin{lstlisting}[frame=none]
public final void subscribeUnderflowSensor(int interval,edu.kit.pse.osip.core.opcua.client.BooleanReceivedListener listener)\end{lstlisting} %end signature
\begin{itemize}
\item{
{\bf  Description}

Subscribes the underflow sensor of the tank at the given interval. Can be called again with the same listener to change interval.
}
\item{
{\bf  Parameters}
  \begin{itemize}
   \item{
\texttt{interval} -- The interval to use when subscribing to the value}
   \item{
\texttt{listener} -- The callback function to be called as soon as the subscribed value was received}
  \end{itemize}
}%end item
\end{itemize}
}%end item
\end{itemize}
}
}
\section{\label{edu.kit.pse.osip.monitoring.controller.MainClass}\index{MainClass}Class MainClass}{
\rule[1em]{\hsize}{4pt}\vskip -1em
\vskip .1in 
The main class as entry point for the whole monitoring.\vskip .1in 
\subsection{Declaration}{
\begin{lstlisting}[frame=none]
public class MainClass
 extends java.lang.Object\end{lstlisting}
\subsection{Methods}{
\rule[1em]{\hsize}{2pt}\vskip -2em
\vskip -2em
\begin{itemize}
\item{ 
\index{main(String\lbrack \rbrack )}
{\bf  main}\\
\begin{lstlisting}[frame=none]
public static final void main(java.lang.String[] args)\end{lstlisting} %end signature
\begin{itemize}
\item{
{\bf  Description}

Main entry point into the jar
}
\item{
{\bf  Parameters}
  \begin{itemize}
   \item{
\texttt{args} -- Command line arguments.}
  \end{itemize}
}%end item
\end{itemize}
}%end item
\end{itemize}
}
}
\section{\label{edu.kit.pse.osip.monitoring.controller.MenuAboutButtonHandler}\index{MenuAboutButtonHandler}Class MenuAboutButtonHandler}{
\rule[1em]{\hsize}{4pt}\vskip -1em
\vskip .1in 
Handles a click on the about menu button in the monitoring view.\vskip .1in 
\subsection{Declaration}{
\begin{lstlisting}[frame=none]
public class MenuAboutButtonHandler
 extends java.lang.Object\end{lstlisting}
\subsection{Constructors}{
\rule[1em]{\hsize}{2pt}\vskip -2em
\vskip -2em
\begin{itemize}
\item{ 
\index{MenuAboutButtonHandler()}
{\bf  MenuAboutButtonHandler}\\
\begin{lstlisting}[frame=none]
public MenuAboutButtonHandler()\end{lstlisting} %end signature
}%end item
\end{itemize}
}
\subsection{Methods}{
\rule[1em]{\hsize}{2pt}\vskip -2em
\vskip -2em
\begin{itemize}
\item{ 
\index{handle(javafx.event.ActionEvent)}
{\bf  handle}\\
\begin{lstlisting}[frame=none]
public final void handle(javafx.event.ActionEvent event)\end{lstlisting} %end signature
\begin{itemize}
\item{
{\bf  Description}

Handles a click on the about menu button in the monitoring view.
}
\item{
{\bf  Parameters}
  \begin{itemize}
   \item{
\texttt{event} -- The occurred event.}
  \end{itemize}
}%end item
\end{itemize}
}%end item
\end{itemize}
}
}
\section{\label{edu.kit.pse.osip.monitoring.controller.MenuHelpButtonHandler}\index{MenuHelpButtonHandler}Class MenuHelpButtonHandler}{
\rule[1em]{\hsize}{4pt}\vskip -1em
\vskip .1in 
Handles a click on the help menu button in the monitoring view.\vskip .1in 
\subsection{Declaration}{
\begin{lstlisting}[frame=none]
public class MenuHelpButtonHandler
 extends java.lang.Object\end{lstlisting}
\subsection{Constructors}{
\rule[1em]{\hsize}{2pt}\vskip -2em
\vskip -2em
\begin{itemize}
\item{ 
\index{MenuHelpButtonHandler()}
{\bf  MenuHelpButtonHandler}\\
\begin{lstlisting}[frame=none]
public MenuHelpButtonHandler()\end{lstlisting} %end signature
}%end item
\end{itemize}
}
\subsection{Methods}{
\rule[1em]{\hsize}{2pt}\vskip -2em
\vskip -2em
\begin{itemize}
\item{ 
\index{handle(javafx.event.ActionEvent)}
{\bf  handle}\\
\begin{lstlisting}[frame=none]
public final void handle(javafx.event.ActionEvent event)\end{lstlisting} %end signature
\begin{itemize}
\item{
{\bf  Description}

Handles a click on the help menu button in the monitoring view.
}
\item{
{\bf  Parameters}
  \begin{itemize}
   \item{
\texttt{event} -- The ocurred event.}
  \end{itemize}
}%end item
\end{itemize}
}%end item
\end{itemize}
}
}
\section{\label{edu.kit.pse.osip.monitoring.controller.MenuSettingsButtonHandler}\index{MenuSettingsButtonHandler}Class MenuSettingsButtonHandler}{
\rule[1em]{\hsize}{4pt}\vskip -1em
\vskip .1in 
Handles a click on the settings menu button in the monitoring view.\vskip .1in 
\subsection{Declaration}{
\begin{lstlisting}[frame=none]
public class MenuSettingsButtonHandler
 extends java.lang.Object\end{lstlisting}
\subsection{Constructors}{
\rule[1em]{\hsize}{2pt}\vskip -2em
\vskip -2em
\begin{itemize}
\item{ 
\index{MenuSettingsButtonHandler(ClientSettingsWrapper, SettingsViewFacade)}
{\bf  MenuSettingsButtonHandler}\\
\begin{lstlisting}[frame=none]
protected MenuSettingsButtonHandler(edu.kit.pse.osip.core.io.files.ClientSettingsWrapper currentSettings,edu.kit.pse.osip.monitoring.view.settings.SettingsViewFacade currentSettingsViewFacade)\end{lstlisting} %end signature
\begin{itemize}
\item{
{\bf  Description}

Creates a new handler.
}
\item{
{\bf  Parameters}
  \begin{itemize}
   \item{
\texttt{currentSettings} -- The current settings for displaying.}
   \item{
\texttt{currentSettingsViewFacade} -- The current facade for the settings view.}
  \end{itemize}
}%end item
\end{itemize}
}%end item
\end{itemize}
}
\subsection{Methods}{
\rule[1em]{\hsize}{2pt}\vskip -2em
\vskip -2em
\begin{itemize}
\item{ 
\index{handle(javafx.event.ActionEvent)}
{\bf  handle}\\
\begin{lstlisting}[frame=none]
public final void handle(javafx.event.ActionEvent event)\end{lstlisting} %end signature
\begin{itemize}
\item{
{\bf  Description}

Handles a click on the settings menu button in the monitoring view.
}
\item{
{\bf  Parameters}
  \begin{itemize}
   \item{
\texttt{event} -- The occured event.}
  \end{itemize}
}%end item
\end{itemize}
}%end item
\end{itemize}
}
}
\section{\label{edu.kit.pse.osip.monitoring.controller.MixTankClient}\index{MixTankClient}Class MixTankClient}{
\rule[1em]{\hsize}{4pt}\vskip -1em
\vskip .1in 
Client for reading the values of a mixtank. Contains a motor.\vskip .1in 
\subsection{Declaration}{
\begin{lstlisting}[frame=none]
public class MixTankClient
 extends edu.kit.pse.osip.monitoring.controller.AbstractTankClient\end{lstlisting}
\subsection{Constructors}{
\rule[1em]{\hsize}{2pt}\vskip -2em
\vskip -2em
\begin{itemize}
\item{ 
\index{MixTankClient(RemoteMachine)}
{\bf  MixTankClient}\\
\begin{lstlisting}[frame=none]
public MixTankClient(edu.kit.pse.osip.core.io.networking.RemoteMachine machine)\end{lstlisting} %end signature
\begin{itemize}
\item{
{\bf  Description}

Creates a new OPC UA client to allow reading the values of a mixtank
}
\item{
{\bf  Parameters}
  \begin{itemize}
   \item{
\texttt{machine} -- The machine to connect to}
  \end{itemize}
}%end item
\end{itemize}
}%end item
\end{itemize}
}
\subsection{Methods}{
\rule[1em]{\hsize}{2pt}\vskip -2em
\vskip -2em
\begin{itemize}
\item{ 
\index{subscribeMotorSpeed(int, FloatReceivedListener)}
{\bf  subscribeMotorSpeed}\\
\begin{lstlisting}[frame=none]
public final void subscribeMotorSpeed(int interval,edu.kit.pse.osip.core.opcua.client.FloatReceivedListener listener)\end{lstlisting} %end signature
\begin{itemize}
\item{
{\bf  Description}

Subscribes the motor speed of the tank at the given interval. Can be called again with the same listener to change interval.
}
\item{
{\bf  Parameters}
  \begin{itemize}
   \item{
\texttt{interval} -- The interval to use when subscribing to the value}
   \item{
\texttt{listener} -- The callback function to be called as soon as the subscribed value was received}
  \end{itemize}
}%end item
\end{itemize}
}%end item
\end{itemize}
}
}
\section{\label{edu.kit.pse.osip.monitoring.controller.MonitoringController}\index{MonitoringController}Class MonitoringController}{
\rule[1em]{\hsize}{4pt}\vskip -1em
\vskip .1in 
The controller starts the monitoring view and initializes the model with the main loop. It is also responsible for the communication between the view and model.\vskip .1in 
\subsection{Declaration}{
\begin{lstlisting}[frame=none]
public class MonitoringController
 extends javafx.application.Application\end{lstlisting}
\subsection{Constructors}{
\rule[1em]{\hsize}{2pt}\vskip -2em
\vskip -2em
\begin{itemize}
\item{ 
\index{MonitoringController()}
{\bf  MonitoringController}\\
\begin{lstlisting}[frame=none]
public MonitoringController()\end{lstlisting} %end signature
}%end item
\end{itemize}
}
\subsection{Methods}{
\rule[1em]{\hsize}{2pt}\vskip -2em
\vskip -2em
\begin{itemize}
\item{ 
\index{Controller()}
{\bf  Controller}\\
\begin{lstlisting}[frame=none]
protected final void Controller()\end{lstlisting} %end signature
\begin{itemize}
\item{
{\bf  Description}

Creates a new controller instance.
}
\end{itemize}
}%end item
\divideents{init}
\item{ 
\index{init()}
{\bf  init}\\
\begin{lstlisting}[frame=none]
public final void init()\end{lstlisting} %end signature
\begin{itemize}
\item{
{\bf  Description}

Initializes all necessary objects.
}
\end{itemize}
}%end item
\divideents{start}
\item{ 
\index{start(javafx.stage.Stage)}
{\bf  start}\\
\begin{lstlisting}[frame=none]
public final void start(javafx.stage.Stage primaryStage)\end{lstlisting} %end signature
\begin{itemize}
\item{
{\bf  Description}

Called when JavaFX creates and starts the application.
}
\item{
{\bf  Parameters}
  \begin{itemize}
   \item{
\texttt{primaryStage} -- The stage used for displaying all controls.}
  \end{itemize}
}%end item
\end{itemize}
}%end item
\divideents{stop}
\item{ 
\index{stop()}
{\bf  stop}\\
\begin{lstlisting}[frame=none]
public final void stop()\end{lstlisting} %end signature
\begin{itemize}
\item{
{\bf  Description}

Called when the last window is closed.
}
\end{itemize}
}%end item
\end{itemize}
}
}
\section{\label{edu.kit.pse.osip.monitoring.controller.SettingsCancelButtonHandler}\index{SettingsCancelButtonHandler}Class SettingsCancelButtonHandler}{
\rule[1em]{\hsize}{4pt}\vskip -1em
\vskip .1in 
Handles a click on the cancel button in the settings view.\vskip .1in 
\subsection{Declaration}{
\begin{lstlisting}[frame=none]
public class SettingsCancelButtonHandler
 extends java.lang.Object\end{lstlisting}
\subsection{Constructors}{
\rule[1em]{\hsize}{2pt}\vskip -2em
\vskip -2em
\begin{itemize}
\item{ 
\index{SettingsCancelButtonHandler(SettingsViewFacade)}
{\bf  SettingsCancelButtonHandler}\\
\begin{lstlisting}[frame=none]
protected SettingsCancelButtonHandler(edu.kit.pse.osip.monitoring.view.settings.SettingsViewFacade currentSettingsViewFacade)\end{lstlisting} %end signature
\begin{itemize}
\item{
{\bf  Description}

Creates a new handler.
}
\item{
{\bf  Parameters}
  \begin{itemize}
   \item{
\texttt{currentSettingsViewFacade} -- The current facade to access the settings view.}
  \end{itemize}
}%end item
\end{itemize}
}%end item
\end{itemize}
}
\subsection{Methods}{
\rule[1em]{\hsize}{2pt}\vskip -2em
\vskip -2em
\begin{itemize}
\item{ 
\index{handle(javafx.event.ActionEvent)}
{\bf  handle}\\
\begin{lstlisting}[frame=none]
public final void handle(javafx.event.ActionEvent event)\end{lstlisting} %end signature
\begin{itemize}
\item{
{\bf  Description}

Handles a click on the cancel button in the settings view.
}
\item{
{\bf  Parameters}
  \begin{itemize}
   \item{
\texttt{event} -- The occured event.}
  \end{itemize}
}%end item
\end{itemize}
}%end item
\end{itemize}
}
}
\section{\label{edu.kit.pse.osip.monitoring.controller.SettingsSaveButtonHandler}\index{SettingsSaveButtonHandler}Class SettingsSaveButtonHandler}{
\rule[1em]{\hsize}{4pt}\vskip -1em
\vskip .1in 
Handler for a click on the save button in the settings view.\vskip .1in 
\subsection{Declaration}{
\begin{lstlisting}[frame=none]
public class SettingsSaveButtonHandler
 extends java.lang.Object\end{lstlisting}
\subsection{Constructors}{
\rule[1em]{\hsize}{2pt}\vskip -2em
\vskip -2em
\begin{itemize}
\item{ 
\index{SettingsSaveButtonHandler(ClientSettingsWrapper, MonitoringViewFacade, SettingsViewFacade)}
{\bf  SettingsSaveButtonHandler}\\
\begin{lstlisting}[frame=none]
protected SettingsSaveButtonHandler(edu.kit.pse.osip.core.io.files.ClientSettingsWrapper settings,edu.kit.pse.osip.monitoring.view.dashboard.MonitoringViewFacade currentMonitoringViewFacade,edu.kit.pse.osip.monitoring.view.settings.SettingsViewFacade currentSettingsViewFacade)\end{lstlisting} %end signature
\begin{itemize}
\item{
{\bf  Description}

Creates a new handler.
}
\item{
{\bf  Parameters}
  \begin{itemize}
   \item{
\texttt{settings} -- The settings object to save the data}
   \item{
\texttt{currentMonitoringViewFacade} -- Stores the access point to the monitoring view to set the user-set settings.}
   \item{
\texttt{currentSettingsViewFacade} -- The facade to the settings view.}
  \end{itemize}
}%end item
\end{itemize}
}%end item
\end{itemize}
}
\subsection{Methods}{
\rule[1em]{\hsize}{2pt}\vskip -2em
\vskip -2em
\begin{itemize}
\item{ 
\index{handle(javafx.event.ActionEvent)}
{\bf  handle}\\
\begin{lstlisting}[frame=none]
public final void handle(javafx.event.ActionEvent event)\end{lstlisting} %end signature
\begin{itemize}
\item{
{\bf  Description}

Handles a click on the save button in the settings view.
}
\item{
{\bf  Parameters}
  \begin{itemize}
   \item{
\texttt{event} -- The occured event.}
  \end{itemize}
}%end item
\end{itemize}
}%end item
\end{itemize}
}
}
\section{\label{edu.kit.pse.osip.monitoring.controller.TankClient}\index{TankClient}Class TankClient}{
\rule[1em]{\hsize}{4pt}\vskip -1em
\vskip .1in 
Client for receiving information about the upper tanks. Contains additional valve.\vskip .1in 
\subsection{Declaration}{
\begin{lstlisting}[frame=none]
public class TankClient
 extends edu.kit.pse.osip.monitoring.controller.AbstractTankClient\end{lstlisting}
\subsection{Constructors}{
\rule[1em]{\hsize}{2pt}\vskip -2em
\vskip -2em
\begin{itemize}
\item{ 
\index{TankClient(RemoteMachine)}
{\bf  TankClient}\\
\begin{lstlisting}[frame=none]
public TankClient(edu.kit.pse.osip.core.io.networking.RemoteMachine machine)\end{lstlisting} %end signature
\begin{itemize}
\item{
{\bf  Description}

Creates a new OPC UA client to allow reading the values of an input tank
}
\item{
{\bf  Parameters}
  \begin{itemize}
   \item{
\texttt{machine} -- The machine to connnect to}
  \end{itemize}
}%end item
\end{itemize}
}%end item
\end{itemize}
}
\subsection{Methods}{
\rule[1em]{\hsize}{2pt}\vskip -2em
\vskip -2em
\begin{itemize}
\item{ 
\index{subscribeInputFlowRate(int, FloatReceivedListener)}
{\bf  subscribeInputFlowRate}\\
\begin{lstlisting}[frame=none]
public final void subscribeInputFlowRate(int interval,edu.kit.pse.osip.core.opcua.client.FloatReceivedListener listener)\end{lstlisting} %end signature
\begin{itemize}
\item{
{\bf  Description}

Subscribes the input flow rate of the tank at the given interval. Can be called again with the same listener to change interval.
}
\item{
{\bf  Parameters}
  \begin{itemize}
   \item{
\texttt{interval} -- The interval to use when subscribing to the value}
   \item{
\texttt{listener} -- The callback function to be called as soon as the subscribed value was received}
  \end{itemize}
}%end item
\end{itemize}
}%end item
\end{itemize}
}
}
}
\chapter{Package edu.kit.pse.osip.monitoring.view.settings}{
\label{edu.kit.pse.osip.monitoring.view.settings}\section{\label{edu.kit.pse.osip.monitoring.view.settings.AlarmSettings}\index{AlarmSettings}Class AlarmSettings}{
\rule[1em]{\hsize}{4pt}\vskip -1em
\vskip .1in 
Contains all controls for enabling / disabling the alarms in the monitoring view.\vskip .1in 
\subsection{Declaration}{
\begin{lstlisting}[frame=none]
public class AlarmSettings
 extends javafx.scene.control.ScrollPane\end{lstlisting}
\subsection{Constructors}{
\rule[1em]{\hsize}{2pt}\vskip -2em
\vskip -2em
\begin{itemize}
\item{ 
\index{AlarmSettings(ClientSettingsWrapper)}
{\bf  AlarmSettings}\\
\begin{lstlisting}[frame=none]
protected AlarmSettings(edu.kit.pse.osip.core.io.files.ClientSettingsWrapper currentSettings)\end{lstlisting} %end signature
\begin{itemize}
\item{
{\bf  Description}

Creates and initializes all controls regarding alarms for the settings.
}
\item{
{\bf  Parameters}
  \begin{itemize}
   \item{
\texttt{currentSettings} -- The current settings used to display them.}
  \end{itemize}
}%end item
\end{itemize}
}%end item
\end{itemize}
}
\subsection{Methods}{
\rule[1em]{\hsize}{2pt}\vskip -2em
\vskip -2em
\begin{itemize}
\item{ 
\index{isOverflowEnabled(TankSelector)}
{\bf  isOverflowEnabled}\\
\begin{lstlisting}[frame=none]
protected final boolean isOverflowEnabled(edu.kit.pse.osip.core.model.base.TankSelector tank)\end{lstlisting} %end signature
\begin{itemize}
\item{
{\bf  Description}

Returns true if the overflow alarm for a specified tank should be enabled and false otherwise.
}
\item{
{\bf  Parameters}
  \begin{itemize}
   \item{
\texttt{tank} -- Tank for which should be looked up if the overflow alarm should be enabled or not.}
  \end{itemize}
}%end item
\item{{\bf  Returns} -- 
true if the overflow alarm for a specified tank should be enabled and false otherwise. 
}%end item
\end{itemize}
}%end item
\divideents{isTemperatureOverflowEnabled}
\item{ 
\index{isTemperatureOverflowEnabled(TankSelector)}
{\bf  isTemperatureOverflowEnabled}\\
\begin{lstlisting}[frame=none]
protected final boolean isTemperatureOverflowEnabled(edu.kit.pse.osip.core.model.base.TankSelector tank)\end{lstlisting} %end signature
\begin{itemize}
\item{
{\bf  Description}

Returns true if the overflow alarm for the temperature for a specified tank should be enabled and false otherwise.
}
\item{
{\bf  Parameters}
  \begin{itemize}
   \item{
\texttt{tank} -- The tank whose value for the temperature overflow alarm will be returned.}
  \end{itemize}
}%end item
\item{{\bf  Returns} -- 
true if the overflow alarm for the temperature for a specified tank should be enabled. false otherwise. 
}%end item
\end{itemize}
}%end item
\divideents{isTemperatureUnderflowEnabled}
\item{ 
\index{isTemperatureUnderflowEnabled(TankSelector)}
{\bf  isTemperatureUnderflowEnabled}\\
\begin{lstlisting}[frame=none]
protected final boolean isTemperatureUnderflowEnabled(edu.kit.pse.osip.core.model.base.TankSelector tank)\end{lstlisting} %end signature
\begin{itemize}
\item{
{\bf  Description}

Returns true if the underflow alarm for the temperature for a specified tank should be enabled and false otherwise.
}
\item{
{\bf  Parameters}
  \begin{itemize}
   \item{
\texttt{tank} -- The tank whose value for enabling / disabling the temperature underflow alarm will be returned.}
  \end{itemize}
}%end item
\item{{\bf  Returns} -- 
true if the overflow alarm for the temperature for a specified tank should be enabled. false otherwise. 
}%end item
\end{itemize}
}%end item
\divideents{isUnderflowEnabled}
\item{ 
\index{isUnderflowEnabled(TankSelector)}
{\bf  isUnderflowEnabled}\\
\begin{lstlisting}[frame=none]
protected final boolean isUnderflowEnabled(edu.kit.pse.osip.core.model.base.TankSelector tank)\end{lstlisting} %end signature
\begin{itemize}
\item{
{\bf  Description}

Returns true if the underflow alarm for a specified tank should be enabled and false otherwise.
}
\item{
{\bf  Parameters}
  \begin{itemize}
   \item{
\texttt{tank} -- Tank for which should be looked up if the underflow alarm should be enabled or not.}
  \end{itemize}
}%end item
\item{{\bf  Returns} -- 
true if the underflow alarm for a specified tank should be enabled and false otherwise. 
}%end item
\end{itemize}
}%end item
\end{itemize}
}
}
\section{\label{edu.kit.pse.osip.monitoring.view.settings.GeneralSettings}\index{GeneralSettings}Class GeneralSettings}{
\rule[1em]{\hsize}{4pt}\vskip -1em
\vskip .1in 
Contains all controls for setting the general settings.\vskip .1in 
\subsection{Declaration}{
\begin{lstlisting}[frame=none]
public class GeneralSettings
 extends javafx.scene.control.ScrollPane\end{lstlisting}
\subsection{Constructors}{
\rule[1em]{\hsize}{2pt}\vskip -2em
\vskip -2em
\begin{itemize}
\item{ 
\index{GeneralSettings(ClientSettingsWrapper)}
{\bf  GeneralSettings}\\
\begin{lstlisting}[frame=none]
protected GeneralSettings(edu.kit.pse.osip.core.io.files.ClientSettingsWrapper currentSettings)\end{lstlisting} %end signature
\begin{itemize}
\item{
{\bf  Description}

Creates a new tab that holds the general settings
}
\item{
{\bf  Parameters}
  \begin{itemize}
   \item{
\texttt{currentSettings} -- The current settings used for displaying them.}
  \end{itemize}
}%end item
\end{itemize}
}%end item
\end{itemize}
}
\subsection{Methods}{
\rule[1em]{\hsize}{2pt}\vskip -2em
\vskip -2em
\begin{itemize}
\item{ 
\index{getServerHost()}
{\bf  getServerHost}\\
\begin{lstlisting}[frame=none]
protected final java.lang.String getServerHost()\end{lstlisting} %end signature
\begin{itemize}
\item{
{\bf  Description}

Returns the servers' hostname.
}
\item{{\bf  Returns} -- 
the servers' host name. 
}%end item
\end{itemize}
}%end item
\divideents{getServerPort}
\item{ 
\index{getServerPort(TankSelector)}
{\bf  getServerPort}\\
\begin{lstlisting}[frame=none]
protected final int getServerPort(edu.kit.pse.osip.core.model.base.TankSelector tank)\end{lstlisting} %end signature
\begin{itemize}
\item{
{\bf  Description}

Returns the port number for a specified server.
}
\item{
{\bf  Parameters}
  \begin{itemize}
   \item{
\texttt{tank} -- Selector of the tank of which's server to get the port for}
  \end{itemize}
}%end item
\item{{\bf  Returns} -- 
the port number for the specified server. 
}%end item
\end{itemize}
}%end item
\divideents{getTime}
\item{ 
\index{getTime()}
{\bf  getTime}\\
\begin{lstlisting}[frame=none]
protected final int getTime()\end{lstlisting} %end signature
\begin{itemize}
\item{
{\bf  Description}

Returns the update interval.
}
\item{{\bf  Returns} -- 
the update interval. 
}%end item
\end{itemize}
}%end item
\end{itemize}
}
}
\section{\label{edu.kit.pse.osip.monitoring.view.settings.Progressions}\index{Progressions}Class Progressions}{
\rule[1em]{\hsize}{4pt}\vskip -1em
\vskip .1in 
Controls all controls for disabling / enabling the progressions in the monitoring view.\vskip .1in 
\subsection{Declaration}{
\begin{lstlisting}[frame=none]
public class Progressions
 extends javafx.scene.control.ScrollPane\end{lstlisting}
\subsection{Constructors}{
\rule[1em]{\hsize}{2pt}\vskip -2em
\vskip -2em
\begin{itemize}
\item{ 
\index{Progressions(ClientSettingsWrapper)}
{\bf  Progressions}\\
\begin{lstlisting}[frame=none]
protected Progressions(edu.kit.pse.osip.core.io.files.ClientSettingsWrapper currentSettings)\end{lstlisting} %end signature
\begin{itemize}
\item{
{\bf  Description}

Creates and initializes all controls regarding the progressions for the settings.
}
\item{
{\bf  Parameters}
  \begin{itemize}
   \item{
\texttt{currentSettings} -- The current settings used for displaying the current state.}
  \end{itemize}
}%end item
\end{itemize}
}%end item
\end{itemize}
}
\subsection{Methods}{
\rule[1em]{\hsize}{2pt}\vskip -2em
\vskip -2em
\begin{itemize}
\item{ 
\index{isFillLevelEnabled(TankSelector)}
{\bf  isFillLevelEnabled}\\
\begin{lstlisting}[frame=none]
protected final boolean isFillLevelEnabled(edu.kit.pse.osip.core.model.base.TankSelector tank)\end{lstlisting} %end signature
\begin{itemize}
\item{
{\bf  Description}

Returns true if the fill level progression should be enabled for a specified tank and false otherwise.
}
\item{
{\bf  Parameters}
  \begin{itemize}
   \item{
\texttt{tank} -- The tank whose value will be returned.}
  \end{itemize}
}%end item
\item{{\bf  Returns} -- 
true if the fill level progression should be enabled for the specified tank or false otherwise. 
}%end item
\end{itemize}
}%end item
\divideents{isTemperatureEnabled}
\item{ 
\index{isTemperatureEnabled(TankSelector)}
{\bf  isTemperatureEnabled}\\
\begin{lstlisting}[frame=none]
protected final boolean isTemperatureEnabled(edu.kit.pse.osip.core.model.base.TankSelector tank)\end{lstlisting} %end signature
\begin{itemize}
\item{
{\bf  Description}

Returns true if the temperature pogression should be enabled for a specified tank and false otherwise.
}
\item{
{\bf  Parameters}
  \begin{itemize}
   \item{
\texttt{tank} -- The tank whose value for enabling / disabling the temperature progression will be returned.}
  \end{itemize}
}%end item
\item{{\bf  Returns} -- 
true if the temperature progression should be enabled for the specified tank. false otherwise. 
}%end item
\end{itemize}
}%end item
\end{itemize}
}
}
\section{\label{edu.kit.pse.osip.monitoring.view.settings.SettingsMainWindow}\index{SettingsMainWindow}Class SettingsMainWindow}{
\rule[1em]{\hsize}{4pt}\vskip -1em
\vskip .1in 
The main entry point for the settings view. Within, the user can set all available parameters for his needs.\vskip .1in 
\subsection{Declaration}{
\begin{lstlisting}[frame=none]
public class SettingsMainWindow
 extends java.lang.Object\end{lstlisting}
\subsection{Fields}{
\rule[1em]{\hsize}{2pt}
\begin{itemize}
\item{
\index{alarmsTab}
\label{edu.kit.pse.osip.monitoring.view.settings.SettingsMainWindow.alarmsTab}\texttt{public AlarmSettings\ {\bf  alarmsTab}}
}
\item{
\index{progressionsTab}
\label{edu.kit.pse.osip.monitoring.view.settings.SettingsMainWindow.progressionsTab}\texttt{public Progressions\ {\bf  progressionsTab}}
}
\item{
\index{generalSettingsTab}
\label{edu.kit.pse.osip.monitoring.view.settings.SettingsMainWindow.generalSettingsTab}\texttt{public GeneralSettings\ {\bf  generalSettingsTab}}
}
\end{itemize}
}
\subsection{Constructors}{
\rule[1em]{\hsize}{2pt}\vskip -2em
\vskip -2em
\begin{itemize}
\item{ 
\index{SettingsMainWindow(ClientSettingsWrapper)}
{\bf  SettingsMainWindow}\\
\begin{lstlisting}[frame=none]
protected SettingsMainWindow(edu.kit.pse.osip.core.io.files.ClientSettingsWrapper currentSettings)\end{lstlisting} %end signature
\begin{itemize}
\item{
{\bf  Description}

Creates all controls for displaying and setting the settings. It shows also the window.
}
\item{
{\bf  Parameters}
  \begin{itemize}
   \item{
\texttt{currentSettings} -- The current settings used for displaying.}
  \end{itemize}
}%end item
\end{itemize}
}%end item
\end{itemize}
}
\subsection{Methods}{
\rule[1em]{\hsize}{2pt}\vskip -2em
\vskip -2em
\begin{itemize}
\item{ 
\index{getAlarmSettings()}
{\bf  getAlarmSettings}\\
\begin{lstlisting}[frame=none]
protected final AlarmSettings getAlarmSettings()\end{lstlisting} %end signature
\begin{itemize}
\item{
{\bf  Description}

Returns the current AlarmSettings object used for the content of the "Alarm" tab.
}
\item{{\bf  Returns} -- 
the current used AlarmSettings object. 
}%end item
\end{itemize}
}%end item
\divideents{getGeneralSettings}
\item{ 
\index{getGeneralSettings()}
{\bf  getGeneralSettings}\\
\begin{lstlisting}[frame=none]
protected final GeneralSettings getGeneralSettings()\end{lstlisting} %end signature
\begin{itemize}
\item{
{\bf  Description}

Returns the current used GeneralSettings fobject for the "General Settings" tab.
}
\item{{\bf  Returns} -- 
the current used GeneralSettings object. 
}%end item
\end{itemize}
}%end item
\divideents{getProgressions}
\item{ 
\index{getProgressions()}
{\bf  getProgressions}\\
\begin{lstlisting}[frame=none]
protected final Progressions getProgressions()\end{lstlisting} %end signature
\begin{itemize}
\item{
{\bf  Description}

Returns the current used Progressions object containing all controls for the "Progressions" tab.
}
\item{{\bf  Returns} -- 
the current used Progressions object. 
}%end item
\end{itemize}
}%end item
\divideents{getStage}
\item{ 
\index{getStage()}
{\bf  getStage}\\
\begin{lstlisting}[frame=none]
protected final javafx.stage.Stage getStage()\end{lstlisting} %end signature
\begin{itemize}
\item{
{\bf  Description}

Returns the used window
}
\item{{\bf  Returns} -- 
the current used window. 
}%end item
\end{itemize}
}%end item
\divideents{setSettingsCancelButtonHandler}
\item{ 
\index{setSettingsCancelButtonHandler(SettingsCancelButtonHandler)}
{\bf  setSettingsCancelButtonHandler}\\
\begin{lstlisting}[frame=none]
protected final void setSettingsCancelButtonHandler(edu.kit.pse.osip.monitoring.controller.SettingsCancelButtonHandler handler)\end{lstlisting} %end signature
\begin{itemize}
\item{
{\bf  Description}

Sets the handler for the cancel button in the settings view.
}
\item{
{\bf  Parameters}
  \begin{itemize}
   \item{
\texttt{handler} -- The handler for the cancel button in the settings view.}
  \end{itemize}
}%end item
\end{itemize}
}%end item
\divideents{setSettingsSaveButtonHandler}
\item{ 
\index{setSettingsSaveButtonHandler(SettingsSaveButtonHandler)}
{\bf  setSettingsSaveButtonHandler}\\
\begin{lstlisting}[frame=none]
protected final void setSettingsSaveButtonHandler(edu.kit.pse.osip.monitoring.controller.SettingsSaveButtonHandler handler)\end{lstlisting} %end signature
\begin{itemize}
\item{
{\bf  Description}

Sets the handler for the save button in the settings view.
}
\item{
{\bf  Parameters}
  \begin{itemize}
   \item{
\texttt{handler} -- The handler for the save button int eh settings view.}
  \end{itemize}
}%end item
\end{itemize}
}%end item
\end{itemize}
}
}
\section{\label{edu.kit.pse.osip.monitoring.view.settings.SettingsViewFacade}\index{SettingsViewFacade}Class SettingsViewFacade}{
\rule[1em]{\hsize}{4pt}\vskip -1em
\vskip .1in 
Provides a single access point to the user-set settings.\vskip .1in 
\subsection{Declaration}{
\begin{lstlisting}[frame=none]
public class SettingsViewFacade
 extends java.lang.Object implements edu.kit.pse.osip.monitoring.controller.SettingsViewInterface\end{lstlisting}
\subsection{Constructors}{
\rule[1em]{\hsize}{2pt}\vskip -2em
\vskip -2em
\begin{itemize}
\item{ 
\index{SettingsViewFacade()}
{\bf  SettingsViewFacade}\\
\begin{lstlisting}[frame=none]
public SettingsViewFacade()\end{lstlisting} %end signature
}%end item
\end{itemize}
}
\subsection{Methods}{
\rule[1em]{\hsize}{2pt}\vskip -2em
\vskip -2em
\begin{itemize}
\item{ 
\index{getServerHostname()}
{\bf  getServerHostname}\\
\begin{lstlisting}[frame=none]
public final java.lang.String getServerHostname()\end{lstlisting} %end signature
\begin{itemize}
\item{
{\bf  Description}

Returns the host name for the servers.
}
\item{{\bf  Returns} -- 
the host name for the servers. 
}%end item
\end{itemize}
}%end item
\divideents{getServerPort}
\item{ 
\index{getServerPort(int)}
{\bf  getServerPort}\\
\begin{lstlisting}[frame=none]
public final int getServerPort(int serverNumber)\end{lstlisting} %end signature
\begin{itemize}
\item{
{\bf  Description}

Returns the port number for a specified server.
}
\item{
{\bf  Parameters}
  \begin{itemize}
   \item{
\texttt{serverNumber} -- Number of the server whose port number will be returned.}
  \end{itemize}
}%end item
\item{{\bf  Returns} -- 
the port number of a specified server. 
}%end item
\end{itemize}
}%end item
\divideents{getTime}
\item{ 
\index{getTime()}
{\bf  getTime}\\
\begin{lstlisting}[frame=none]
public final int getTime()\end{lstlisting} %end signature
\begin{itemize}
\item{
{\bf  Description}

Returns the update time interval.
}
\item{{\bf  Returns} -- 
the update time interval. 
}%end item
\end{itemize}
}%end item
\divideents{hideSettingsWindow}
\item{ 
\index{hideSettingsWindow()}
{\bf  hideSettingsWindow}\\
\begin{lstlisting}[frame=none]
public final void hideSettingsWindow()\end{lstlisting} %end signature
\begin{itemize}
\item{
{\bf  Description}

Hides the settings view.
}
\end{itemize}
}%end item
\divideents{isFillLevelProgressionEnabled}
\item{ 
\index{isFillLevelProgressionEnabled(TankSelector)}
{\bf  isFillLevelProgressionEnabled}\\
\begin{lstlisting}[frame=none]
public final boolean isFillLevelProgressionEnabled(edu.kit.pse.osip.core.model.base.TankSelector tank)\end{lstlisting} %end signature
\begin{itemize}
\item{
{\bf  Description}

Returns true if the logging of the fill level progression should be enabled for a specified tank and false otherwise.
}
\item{
{\bf  Parameters}
  \begin{itemize}
   \item{
\texttt{tank} -- The tank whose fill level progression should be enabled or disabled.}
  \end{itemize}
}%end item
\item{{\bf  Returns} -- 
true if the logging of the fill level progression should be enabled for the specified tank or false otherwise. 
}%end item
\end{itemize}
}%end item
\divideents{isOverflowAlarmEnabled}
\item{ 
\index{isOverflowAlarmEnabled(TankSelector)}
{\bf  isOverflowAlarmEnabled}\\
\begin{lstlisting}[frame=none]
public final boolean isOverflowAlarmEnabled(edu.kit.pse.osip.core.model.base.TankSelector tank)\end{lstlisting} %end signature
\begin{itemize}
\item{
{\bf  Description}

Returns true if the overflow alarm should be enabled for a specified tank and false otherwise.
}
\item{
{\bf  Parameters}
  \begin{itemize}
   \item{
\texttt{tank} -- The tank whose overflow alarm should be enabled or disabled.}
  \end{itemize}
}%end item
\item{{\bf  Returns} -- 
true if the overflow alarm should be enabled for a specified tank. false otherwise. 
}%end item
\end{itemize}
}%end item
\divideents{isTemperatureOverflowAlarmEnabled}
\item{ 
\index{isTemperatureOverflowAlarmEnabled(TankSelector)}
{\bf  isTemperatureOverflowAlarmEnabled}\\
\begin{lstlisting}[frame=none]
public final boolean isTemperatureOverflowAlarmEnabled(edu.kit.pse.osip.core.model.base.TankSelector tank)\end{lstlisting} %end signature
\begin{itemize}
\item{
{\bf  Description}

Returns true if the overflow alarm for the temperature should be enabled for a specified tank and false otherwise.
}
\item{
{\bf  Parameters}
  \begin{itemize}
   \item{
\texttt{tank} -- The specified tank.}
  \end{itemize}
}%end item
\item{{\bf  Returns} -- 
true if the overflow alarm for the temperature should be enabled for a specified tank and false otherwise. 
}%end item
\end{itemize}
}%end item
\divideents{isTemperatureProgressionEnabled}
\item{ 
\index{isTemperatureProgressionEnabled(TankSelector)}
{\bf  isTemperatureProgressionEnabled}\\
\begin{lstlisting}[frame=none]
public final boolean isTemperatureProgressionEnabled(edu.kit.pse.osip.core.model.base.TankSelector tank)\end{lstlisting} %end signature
\begin{itemize}
\item{
{\bf  Description}

Returns true if the logging of the temperature progression should be enabled for the specified tank and false otherwise.
}
\item{
{\bf  Parameters}
  \begin{itemize}
   \item{
\texttt{tank} -- The tank whose temperature progression will be set.}
  \end{itemize}
}%end item
\item{{\bf  Returns} -- 
true if the logging of the temperature progression should be enabled for the specified tank. false otherwise. 
}%end item
\end{itemize}
}%end item
\divideents{isTemperatureUnderflowAlarmEnabled}
\item{ 
\index{isTemperatureUnderflowAlarmEnabled(TankSelector)}
{\bf  isTemperatureUnderflowAlarmEnabled}\\
\begin{lstlisting}[frame=none]
public final boolean isTemperatureUnderflowAlarmEnabled(edu.kit.pse.osip.core.model.base.TankSelector tank)\end{lstlisting} %end signature
\begin{itemize}
\item{
{\bf  Description}

Returns true if the underflow alarm for the temperature should be enabled for a specified tank and false otherwise.
}
\item{
{\bf  Parameters}
  \begin{itemize}
   \item{
\texttt{tank} -- The specified tank.}
  \end{itemize}
}%end item
\item{{\bf  Returns} -- 
true if the underflow alarm for the temperature should be enabled for a specified tank and false otherwise. 
}%end item
\end{itemize}
}%end item
\divideents{isUnderflowAlarmEnabled}
\item{ 
\index{isUnderflowAlarmEnabled(TankSelector)}
{\bf  isUnderflowAlarmEnabled}\\
\begin{lstlisting}[frame=none]
public final boolean isUnderflowAlarmEnabled(edu.kit.pse.osip.core.model.base.TankSelector tank)\end{lstlisting} %end signature
\begin{itemize}
\item{
{\bf  Description}

Returns true if the underflow alarm should be enabled for a specified tank and false otherwise.
}
\item{
{\bf  Parameters}
  \begin{itemize}
   \item{
\texttt{tank} -- Tank whose value will be looked up.}
  \end{itemize}
}%end item
\item{{\bf  Returns} -- 
true if the underflow alarm should be enabled for the specified tank. false otherwise. 
}%end item
\end{itemize}
}%end item
\divideents{setSettingsCancelButtonHandler}
\item{ 
\index{setSettingsCancelButtonHandler(SettingsCancelButtonHandler)}
{\bf  setSettingsCancelButtonHandler}\\
\begin{lstlisting}[frame=none]
public final void setSettingsCancelButtonHandler(edu.kit.pse.osip.monitoring.controller.SettingsCancelButtonHandler handler)\end{lstlisting} %end signature
\begin{itemize}
\item{
{\bf  Description}

Sets the handler for the cancel button in the settings view.
}
\item{
{\bf  Parameters}
  \begin{itemize}
   \item{
\texttt{handler} -- The handler for the cancel button in the settings view.}
  \end{itemize}
}%end item
\end{itemize}
}%end item
\divideents{setSettingsSaveButtonHandler}
\item{ 
\index{setSettingsSaveButtonHandler(SettingsSaveButtonHandler)}
{\bf  setSettingsSaveButtonHandler}\\
\begin{lstlisting}[frame=none]
public final void setSettingsSaveButtonHandler(edu.kit.pse.osip.monitoring.controller.SettingsSaveButtonHandler handler)\end{lstlisting} %end signature
\begin{itemize}
\item{
{\bf  Description}

Sets the handler for the save button in the settings view.
}
\item{
{\bf  Parameters}
  \begin{itemize}
   \item{
\texttt{handler} -- The handler for the save button in the settings view.}
  \end{itemize}
}%end item
\end{itemize}
}%end item
\divideents{showSettingsWindow}
\item{ 
\index{showSettingsWindow(ClientSettingsWrapper)}
{\bf  showSettingsWindow}\\
\begin{lstlisting}[frame=none]
public final void showSettingsWindow(edu.kit.pse.osip.core.io.files.ClientSettingsWrapper currentSettings)\end{lstlisting} %end signature
\begin{itemize}
\item{
{\bf  Description}

Shows the settings view with the current settings.
}
\item{
{\bf  Parameters}
  \begin{itemize}
   \item{
\texttt{currentSettings} -- The current settings used for displaying them.}
  \end{itemize}
}%end item
\end{itemize}
}%end item
\end{itemize}
}
}
}
\chapter{Package edu.kit.pse.osip.monitoring.view.dialogs}{
\label{edu.kit.pse.osip.monitoring.view.dialogs}\section{\label{edu.kit.pse.osip.monitoring.view.dialogs.AboutDialog}\index{AboutDialog}Class AboutDialog}{
\rule[1em]{\hsize}{4pt}\vskip -1em
\vskip .1in 
This window shows information about the creators of OSIP.\vskip .1in 
\subsection{Declaration}{
\begin{lstlisting}[frame=none]
public class AboutDialog
 extends javafx.stage.Stage\end{lstlisting}
\subsection{Constructors}{
\rule[1em]{\hsize}{2pt}\vskip -2em
\vskip -2em
\begin{itemize}
\item{ 
\index{AboutDialog()}
{\bf  AboutDialog}\\
\begin{lstlisting}[frame=none]
public AboutDialog()\end{lstlisting} %end signature
}%end item
\end{itemize}
}
}
\section{\label{edu.kit.pse.osip.monitoring.view.dialogs.HelpDialog}\index{HelpDialog}Class HelpDialog}{
\rule[1em]{\hsize}{4pt}\vskip -1em
\vskip .1in 
This window shows a short description of OSIP and how to use it.\vskip .1in 
\subsection{Declaration}{
\begin{lstlisting}[frame=none]
public class HelpDialog
 extends javafx.stage.Stage\end{lstlisting}
\subsection{Constructors}{
\rule[1em]{\hsize}{2pt}\vskip -2em
\vskip -2em
\begin{itemize}
\item{ 
\index{HelpDialog()}
{\bf  HelpDialog}\\
\begin{lstlisting}[frame=none]
public HelpDialog()\end{lstlisting} %end signature
}%end item
\end{itemize}
}
}
}
\chapter{Package edu.kit.pse.osip.monitoring.view.dashboard}{
\label{edu.kit.pse.osip.monitoring.view.dashboard}\section{\label{edu.kit.pse.osip.monitoring.view.dashboard.AbstractTankVisualization}\index{AbstractTankVisualization}Class AbstractTankVisualization}{
\rule[1em]{\hsize}{4pt}\vskip -1em
\vskip .1in 
Visualises a general tank.\vskip .1in 
\subsection{Declaration}{
\begin{lstlisting}[frame=none]
public abstract class AbstractTankVisualization
 extends javafx.scene.layout.Pane\end{lstlisting}
\subsection{All known subclasses}{MixTankVisualization\small{\refdefined{edu.kit.pse.osip.monitoring.view.dashboard.MixTankVisualization}}, TankVisualization\small{\refdefined{edu.kit.pse.osip.monitoring.view.dashboard.TankVisualization}}}
\subsection{Fields}{
\rule[1em]{\hsize}{2pt}
\begin{itemize}
\item{
\index{overflowAlarm}
\label{edu.kit.pse.osip.monitoring.view.dashboard.AbstractTankVisualization.overflowAlarm}\texttt{protected AlarmVisualization\ {\bf  overflowAlarm}}
\begin{itemize}
\item{\vskip -.9ex 
Visualization for the overflow alarm.}
\end{itemize}
}
\item{
\index{underflowAlarm}
\label{edu.kit.pse.osip.monitoring.view.dashboard.AbstractTankVisualization.underflowAlarm}\texttt{protected AlarmVisualization\ {\bf  underflowAlarm}}
\begin{itemize}
\item{\vskip -.9ex 
Visualization for the underflow alarm.}
\end{itemize}
}
\item{
\index{temperatureOverflowAlarm}
\label{edu.kit.pse.osip.monitoring.view.dashboard.AbstractTankVisualization.temperatureOverflowAlarm}\texttt{protected AlarmVisualization\ {\bf  temperatureOverflowAlarm}}
\begin{itemize}
\item{\vskip -.9ex 
Alarm when the temperature becomes too high.}
\end{itemize}
}
\item{
\index{temperatureUnderflowAlarm}
\label{edu.kit.pse.osip.monitoring.view.dashboard.AbstractTankVisualization.temperatureUnderflowAlarm}\texttt{protected AlarmVisualization\ {\bf  temperatureUnderflowAlarm}}
\begin{itemize}
\item{\vskip -.9ex 
Alarm when a temperature becomes too low.}
\end{itemize}
}
\item{
\index{drain}
\label{edu.kit.pse.osip.monitoring.view.dashboard.AbstractTankVisualization.drain}\texttt{protected BarVisualization\ {\bf  drain}}
\begin{itemize}
\item{\vskip -.9ex 
Visualization for the drain pipe.}
\end{itemize}
}
\item{
\index{fillLevel}
\label{edu.kit.pse.osip.monitoring.view.dashboard.AbstractTankVisualization.fillLevel}\texttt{protected FillLevelVisualization\ {\bf  fillLevel}}
\begin{itemize}
\item{\vskip -.9ex 
Visualization of the fill level.}
\end{itemize}
}
\item{
\index{temperature}
\label{edu.kit.pse.osip.monitoring.view.dashboard.AbstractTankVisualization.temperature}\texttt{protected TemperatureVisualization\ {\bf  temperature}}
\begin{itemize}
\item{\vskip -.9ex 
Visualization of the temperature.}
\end{itemize}
}
\item{
\index{progresses}
\label{edu.kit.pse.osip.monitoring.view.dashboard.AbstractTankVisualization.progresses}\texttt{protected ProgressOverview\ {\bf  progresses}}
\begin{itemize}
\item{\vskip -.9ex 
Visualization of the progressions.}
\end{itemize}
}
\end{itemize}
}
\subsection{Constructors}{
\rule[1em]{\hsize}{2pt}\vskip -2em
\vskip -2em
\begin{itemize}
\item{ 
\index{AbstractTankVisualization()}
{\bf  AbstractTankVisualization}\\
\begin{lstlisting}[frame=none]
public AbstractTankVisualization()\end{lstlisting} %end signature
}%end item
\end{itemize}
}
\subsection{Methods}{
\rule[1em]{\hsize}{2pt}\vskip -2em
\vskip -2em
\begin{itemize}
\item{ 
\index{setFillLevelProgressionEnabled(boolean)}
{\bf  setFillLevelProgressionEnabled}\\
\begin{lstlisting}[frame=none]
protected final void setFillLevelProgressionEnabled(boolean progressionEnabled)\end{lstlisting} %end signature
\begin{itemize}
\item{
{\bf  Description}

Enables or disables the logging of the fill level progression.
}
\item{
{\bf  Parameters}
  \begin{itemize}
   \item{
\texttt{progressionEnabled} -- true if the fill level progression should be logged. false otherwise.}
  \end{itemize}
}%end item
\end{itemize}
}%end item
\divideents{setOverflowAlarmEnabled}
\item{ 
\index{setOverflowAlarmEnabled(boolean)}
{\bf  setOverflowAlarmEnabled}\\
\begin{lstlisting}[frame=none]
protected final void setOverflowAlarmEnabled(boolean alarmEnabled)\end{lstlisting} %end signature
\begin{itemize}
\item{
{\bf  Description}

Enables or disables the overflow alarm of this tank.
}
\item{
{\bf  Parameters}
  \begin{itemize}
   \item{
\texttt{alarmEnabled} -- true if the overflow alarm should be enabled. false otherwise.}
  \end{itemize}
}%end item
\end{itemize}
}%end item
\divideents{setTemperatureOverflowAlarmEnabled}
\item{ 
\index{setTemperatureOverflowAlarmEnabled(boolean)}
{\bf  setTemperatureOverflowAlarmEnabled}\\
\begin{lstlisting}[frame=none]
protected final void setTemperatureOverflowAlarmEnabled(boolean alarmEnabled)\end{lstlisting} %end signature
\begin{itemize}
\item{
{\bf  Description}

Sets the overflow alarm for the temperature enabled or disabled.
}
\item{
{\bf  Parameters}
  \begin{itemize}
   \item{
\texttt{alarmEnabled} -- true when the alarm should be enabled. false otherwise.}
  \end{itemize}
}%end item
\end{itemize}
}%end item
\divideents{setTemperatureProgressionEnabled}
\item{ 
\index{setTemperatureProgressionEnabled(boolean)}
{\bf  setTemperatureProgressionEnabled}\\
\begin{lstlisting}[frame=none]
protected final void setTemperatureProgressionEnabled(boolean progressionEnabled)\end{lstlisting} %end signature
\begin{itemize}
\item{
{\bf  Description}

Enables or disables the logging of the temperature progression.
}
\item{
{\bf  Parameters}
  \begin{itemize}
   \item{
\texttt{progressionEnabled} -- true if the temperature should be logged. false otherwise.}
  \end{itemize}
}%end item
\end{itemize}
}%end item
\divideents{setTemperatureUnderflowAlarmEnabled}
\item{ 
\index{setTemperatureUnderflowAlarmEnabled(boolean)}
{\bf  setTemperatureUnderflowAlarmEnabled}\\
\begin{lstlisting}[frame=none]
protected final void setTemperatureUnderflowAlarmEnabled(boolean alarmEnabled)\end{lstlisting} %end signature
\begin{itemize}
\item{
{\bf  Description}

Sets the underflow alarm for the temperature enabled or disabled.
}
\item{
{\bf  Parameters}
  \begin{itemize}
   \item{
\texttt{alarmEnabled} -- true when the alarm should be enabled and false otherwise.}
  \end{itemize}
}%end item
\end{itemize}
}%end item
\divideents{setUnderflowAlarmEnabled}
\item{ 
\index{setUnderflowAlarmEnabled(boolean)}
{\bf  setUnderflowAlarmEnabled}\\
\begin{lstlisting}[frame=none]
protected final void setUnderflowAlarmEnabled(boolean alarmEnabled)\end{lstlisting} %end signature
\begin{itemize}
\item{
{\bf  Description}

Enables or disables the underflow alarm for this tank.
}
\item{
{\bf  Parameters}
  \begin{itemize}
   \item{
\texttt{alarmEnabled} -- true if the overflow alarm should be enabled. false otherwise.}
  \end{itemize}
}%end item
\end{itemize}
}%end item
\end{itemize}
}
}
\section{\label{edu.kit.pse.osip.monitoring.view.dashboard.AlarmState}\index{AlarmState}Class AlarmState}{
\rule[1em]{\hsize}{4pt}\vskip -1em
\vskip .1in 
Defines all possible states of an alarm in the monitoring view.\vskip .1in 
\subsection{Declaration}{
\begin{lstlisting}[frame=none]
public final class AlarmState
 extends java.lang.Enum\end{lstlisting}
\subsection{Fields}{
\rule[1em]{\hsize}{2pt}
\begin{itemize}
\item{
\index{ALARM\_ENABLED}
\label{edu.kit.pse.osip.monitoring.view.dashboard.AlarmState.ALARM_ENABLED}\texttt{public static final AlarmState\ {\bf  ALARM\_ENABLED}}
\begin{itemize}
\item{\vskip -.9ex 
The alarm is enabled and can receive notifications.}
\end{itemize}
}
\item{
\index{ALARM\_DISABLED}
\label{edu.kit.pse.osip.monitoring.view.dashboard.AlarmState.ALARM_DISABLED}\texttt{public static final AlarmState\ {\bf  ALARM\_DISABLED}}
\begin{itemize}
\item{\vskip -.9ex 
The alarm is disabled and receives no de- or activation.}
\end{itemize}
}
\end{itemize}
}
\subsection{Methods}{
\rule[1em]{\hsize}{2pt}\vskip -2em
\vskip -2em
\begin{itemize}
\item{ 
\index{valueOf(String)}
{\bf  valueOf}\\
\begin{lstlisting}[frame=none]
public static AlarmState valueOf(java.lang.String name)\end{lstlisting} %end signature
}%end item
\divideents{values}
\item{ 
\index{values()}
{\bf  values}\\
\begin{lstlisting}[frame=none]
public static AlarmState[] values()\end{lstlisting} %end signature
}%end item
\end{itemize}
}
}
\section{\label{edu.kit.pse.osip.monitoring.view.dashboard.AlarmVisualization}\index{AlarmVisualization}Class AlarmVisualization}{
\rule[1em]{\hsize}{4pt}\vskip -1em
\vskip .1in 
\subsection{Declaration}{
\begin{lstlisting}[frame=none]
public class AlarmVisualization
 extends java.lang.Object implements java.util.Observer\end{lstlisting}
\subsection{Constructors}{
\rule[1em]{\hsize}{2pt}\vskip -2em
\vskip -2em
\begin{itemize}
\item{ 
\index{AlarmVisualization(String)}
{\bf  AlarmVisualization}\\
\begin{lstlisting}[frame=none]
protected AlarmVisualization(java.lang.String alarmName)\end{lstlisting} %end signature
\begin{itemize}
\item{
{\bf  Description}

Creates and initializes a new alarm visualization.
}
\item{
{\bf  Parameters}
  \begin{itemize}
   \item{
\texttt{alarmName} -- The name of the alarm.}
  \end{itemize}
}%end item
\end{itemize}
}%end item
\end{itemize}
}
\subsection{Methods}{
\rule[1em]{\hsize}{2pt}\vskip -2em
\vskip -2em
\begin{itemize}
\item{ 
\index{setAlarmState(AlarmState)}
{\bf  setAlarmState}\\
\begin{lstlisting}[frame=none]
protected final void setAlarmState(AlarmState newState)\end{lstlisting} %end signature
\begin{itemize}
\item{
{\bf  Description}

Sets the current state of the alarm.
}
\item{
{\bf  Parameters}
  \begin{itemize}
   \item{
\texttt{newState} -- The new state of the alarm.}
  \end{itemize}
}%end item
\end{itemize}
}%end item
\divideents{update}
\item{ 
\index{update(Observable, Object)}
{\bf  update}\\
\begin{lstlisting}[frame=none]
public final void update(java.util.Observable observable,java.lang.Object object)\end{lstlisting} %end signature
\begin{itemize}
\item{
{\bf  Description}

Called when the observed object is updated.
}
\item{
{\bf  Parameters}
  \begin{itemize}
   \item{
\texttt{observable} -- The observed object.}
   \item{
\texttt{object} -- The new value.}
  \end{itemize}
}%end item
\end{itemize}
}%end item
\end{itemize}
}
}
\section{\label{edu.kit.pse.osip.monitoring.view.dashboard.BarVisualization}\index{BarVisualization}Class BarVisualization}{
\rule[1em]{\hsize}{4pt}\vskip -1em
\vskip .1in 
Visualizes the amount of liquid flowing through a pipe or the motor speed.\vskip .1in 
\subsection{Declaration}{
\begin{lstlisting}[frame=none]
public class BarVisualization
 extends java.lang.Object implements java.util.Observer\end{lstlisting}
\subsection{Constructors}{
\rule[1em]{\hsize}{2pt}\vskip -2em
\vskip -2em
\begin{itemize}
\item{ 
\index{BarVisualization(String)}
{\bf  BarVisualization}\\
\begin{lstlisting}[frame=none]
protected BarVisualization(java.lang.String name)\end{lstlisting} %end signature
\begin{itemize}
\item{
{\bf  Description}

Creates and initializes a new visualization.
}
\item{
{\bf  Parameters}
  \begin{itemize}
   \item{
\texttt{name} -- The name of the represented bar.}
  \end{itemize}
}%end item
\end{itemize}
}%end item
\end{itemize}
}
\subsection{Methods}{
\rule[1em]{\hsize}{2pt}\vskip -2em
\vskip -2em
\begin{itemize}
\item{ 
\index{getBar()}
{\bf  getBar}\\
\begin{lstlisting}[frame=none]
protected final jfxtras.scene.control.gauge.linear.SimpleMetroArcGauge getBar()\end{lstlisting} %end signature
\begin{itemize}
\item{
{\bf  Description}

Returns the gauge showing the current value
}
\item{{\bf  Returns} -- 
the gauge showing the current value 
}%end item
\end{itemize}
}%end item
\divideents{getLabel}
\item{ 
\index{getLabel()}
{\bf  getLabel}\\
\begin{lstlisting}[frame=none]
protected final javafx.scene.control.Label getLabel()\end{lstlisting} %end signature
\begin{itemize}
\item{
{\bf  Description}

Returns the label for showing which type is represented.
}
\item{{\bf  Returns} -- 
the label 
}%end item
\end{itemize}
}%end item
\divideents{update}
\item{ 
\index{update(Observable, Object)}
{\bf  update}\\
\begin{lstlisting}[frame=none]
public final void update(java.util.Observable observable,java.lang.Object object)\end{lstlisting} %end signature
\begin{itemize}
\item{
{\bf  Description}

Called when the observed object has changed.
}
\item{
{\bf  Parameters}
  \begin{itemize}
   \item{
\texttt{observable} -- The observed object.}
   \item{
\texttt{object} -- The new value.}
  \end{itemize}
}%end item
\end{itemize}
}%end item
\end{itemize}
}
}
\section{\label{edu.kit.pse.osip.monitoring.view.dashboard.ColorVisualization}\index{ColorVisualization}Class ColorVisualization}{
\rule[1em]{\hsize}{4pt}\vskip -1em
\vskip .1in 
Visualises the color of the mixtank.\vskip .1in 
\subsection{Declaration}{
\begin{lstlisting}[frame=none]
public class ColorVisualization
 extends java.lang.Object implements java.util.Observer\end{lstlisting}
\subsection{Constructors}{
\rule[1em]{\hsize}{2pt}\vskip -2em
\vskip -2em
\begin{itemize}
\item{ 
\index{ColorVisualization()}
{\bf  ColorVisualization}\\
\begin{lstlisting}[frame=none]
protected ColorVisualization()\end{lstlisting} %end signature
\begin{itemize}
\item{
{\bf  Description}

Creates and initializes all controls for the color visualization.
}
\end{itemize}
}%end item
\end{itemize}
}
\subsection{Methods}{
\rule[1em]{\hsize}{2pt}\vskip -2em
\vskip -2em
\begin{itemize}
\item{ 
\index{getColorLabel()}
{\bf  getColorLabel}\\
\begin{lstlisting}[frame=none]
protected final javafx.scene.control.Label getColorLabel()\end{lstlisting} %end signature
\begin{itemize}
\item{
{\bf  Description}

Returns the label showing the color section.
}
\item{{\bf  Returns} -- 
the label showing the color section. 
}%end item
\end{itemize}
}%end item
\divideents{getColorRectangle}
\item{ 
\index{getColorRectangle()}
{\bf  getColorRectangle}\\
\begin{lstlisting}[frame=none]
protected final javafx.scene.shape.Rectangle getColorRectangle()\end{lstlisting} %end signature
\begin{itemize}
\item{
{\bf  Description}

Returns the rectangle containing the current color.
}
\item{{\bf  Returns} -- 
the rectangle containing the current color. 
}%end item
\end{itemize}
}%end item
\divideents{update}
\item{ 
\index{update(Observable, Object)}
{\bf  update}\\
\begin{lstlisting}[frame=none]
public final void update(java.util.Observable observable,java.lang.Object object)\end{lstlisting} %end signature
\begin{itemize}
\item{
{\bf  Description}

Called when the observed object changed.
}
\item{
{\bf  Parameters}
  \begin{itemize}
   \item{
\texttt{observable} -- The observed object.}
   \item{
\texttt{object} -- The new value.}
  \end{itemize}
}%end item
\end{itemize}
}%end item
\end{itemize}
}
}
\section{\label{edu.kit.pse.osip.monitoring.view.dashboard.FillLevelVisualization}\index{FillLevelVisualization}Class FillLevelVisualization}{
\rule[1em]{\hsize}{4pt}\vskip -1em
\vskip .1in 
Visualises a fill level.\vskip .1in 
\subsection{Declaration}{
\begin{lstlisting}[frame=none]
public class FillLevelVisualization
 extends java.lang.Object implements java.util.Observer\end{lstlisting}
\subsection{Constructors}{
\rule[1em]{\hsize}{2pt}\vskip -2em
\vskip -2em
\begin{itemize}
\item{ 
\index{FillLevelVisualization()}
{\bf  FillLevelVisualization}\\
\begin{lstlisting}[frame=none]
protected FillLevelVisualization()\end{lstlisting} %end signature
\begin{itemize}
\item{
{\bf  Description}

Creates a new fill level visualization.
}
\end{itemize}
}%end item
\end{itemize}
}
\subsection{Methods}{
\rule[1em]{\hsize}{2pt}\vskip -2em
\vskip -2em
\begin{itemize}
\item{ 
\index{getFillLevelBar()}
{\bf  getFillLevelBar}\\
\begin{lstlisting}[frame=none]
protected final jfxtras.scene.control.gauge.linear.BasicRoundDailGauge getFillLevelBar()\end{lstlisting} %end signature
\begin{itemize}
\item{
{\bf  Description}

Returns the gauge showing the current fill level.
}
\item{{\bf  Returns} -- 
the gauge showing the current fill level. 
}%end item
\end{itemize}
}%end item
\divideents{getFillLevelLabel}
\item{ 
\index{getFillLevelLabel()}
{\bf  getFillLevelLabel}\\
\begin{lstlisting}[frame=none]
protected final javafx.scene.control.Label getFillLevelLabel()\end{lstlisting} %end signature
\begin{itemize}
\item{
{\bf  Description}

Returns the label for showing the fill level section.
}
\item{{\bf  Returns} -- 
the label for showing the fill level section. 
}%end item
\end{itemize}
}%end item
\divideents{update}
\item{ 
\index{update(Observable, Object)}
{\bf  update}\\
\begin{lstlisting}[frame=none]
public final void update(java.util.Observable observable,java.lang.Object object)\end{lstlisting} %end signature
\begin{itemize}
\item{
{\bf  Description}

Called when the observed object is updated.
}
\item{
{\bf  Parameters}
  \begin{itemize}
   \item{
\texttt{observable} -- The observed object.}
   \item{
\texttt{object} -- The new value.}
  \end{itemize}
}%end item
\end{itemize}
}%end item
\end{itemize}
}
}
\section{\label{edu.kit.pse.osip.monitoring.view.dashboard.Light}\index{Light}Class Light}{
\rule[1em]{\hsize}{4pt}\vskip -1em
\vskip .1in 
Visualizes a traffic light.\vskip .1in 
\subsection{Declaration}{
\begin{lstlisting}[frame=none]
public class Light
 extends javafx.scene.layout.Pane implements java.util.Observer\end{lstlisting}
\subsection{Constructors}{
\rule[1em]{\hsize}{2pt}\vskip -2em
\vskip -2em
\begin{itemize}
\item{ 
\index{Light()}
{\bf  Light}\\
\begin{lstlisting}[frame=none]
protected Light()\end{lstlisting} %end signature
\begin{itemize}
\item{
{\bf  Description}

Creates a new light.
}
\end{itemize}
}%end item
\end{itemize}
}
\subsection{Methods}{
\rule[1em]{\hsize}{2pt}\vskip -2em
\vskip -2em
\begin{itemize}
\item{ 
\index{setAlarmState(AlarmState)}
{\bf  setAlarmState}\\
\begin{lstlisting}[frame=none]
protected final void setAlarmState(AlarmState newState)\end{lstlisting} %end signature
\begin{itemize}
\item{
{\bf  Description}

Sets the alarm state.
}
\item{
{\bf  Parameters}
  \begin{itemize}
   \item{
\texttt{newState} -- The new alarm state.}
  \end{itemize}
}%end item
\end{itemize}
}%end item
\divideents{update}
\item{ 
\index{update(Observable, Object)}
{\bf  update}\\
\begin{lstlisting}[frame=none]
public final void update(java.util.Observable observable,java.lang.Object object)\end{lstlisting} %end signature
\begin{itemize}
\item{
{\bf  Description}

Called when the observed object updates.
}
\item{
{\bf  Parameters}
  \begin{itemize}
   \item{
\texttt{observable} -- The observed object.}
   \item{
\texttt{object} -- The new value.}
  \end{itemize}
}%end item
\end{itemize}
}%end item
\end{itemize}
}
}
\section{\label{edu.kit.pse.osip.monitoring.view.dashboard.LoggingConsole}\index{LoggingConsole}Class LoggingConsole}{
\rule[1em]{\hsize}{4pt}\vskip -1em
\vskip .1in 
The graphical console to show all logged events.\vskip .1in 
\subsection{Declaration}{
\begin{lstlisting}[frame=none]
public class LoggingConsole
 extends javafx.scene.control.ScrollPane\end{lstlisting}
\subsection{Constructors}{
\rule[1em]{\hsize}{2pt}\vskip -2em
\vskip -2em
\begin{itemize}
\item{ 
\index{LoggingConsole()}
{\bf  LoggingConsole}\\
\begin{lstlisting}[frame=none]
protected LoggingConsole()\end{lstlisting} %end signature
\begin{itemize}
\item{
{\bf  Description}

Creates and initializes the console.
}
\end{itemize}
}%end item
\end{itemize}
}
\subsection{Methods}{
\rule[1em]{\hsize}{2pt}\vskip -2em
\vskip -2em
\begin{itemize}
\item{ 
\index{log(String)}
{\bf  log}\\
\begin{lstlisting}[frame=none]
public final void log(java.lang.String message)\end{lstlisting} %end signature
\begin{itemize}
\item{
{\bf  Description}

Logs an event.
}
\item{
{\bf  Parameters}
  \begin{itemize}
   \item{
\texttt{message} -- The logging message for the occured event.}
  \end{itemize}
}%end item
\end{itemize}
}%end item
\end{itemize}
}
}
\section{\label{edu.kit.pse.osip.monitoring.view.dashboard.MixTankVisualization}\index{MixTankVisualization}Class MixTankVisualization}{
\rule[1em]{\hsize}{4pt}\vskip -1em
\vskip .1in 
Visualization for the mixtank.\vskip .1in 
\subsection{Declaration}{
\begin{lstlisting}[frame=none]
public class MixTankVisualization
 extends edu.kit.pse.osip.monitoring.view.dashboard.AbstractTankVisualization\end{lstlisting}
\subsection{Constructors}{
\rule[1em]{\hsize}{2pt}\vskip -2em
\vskip -2em
\begin{itemize}
\item{ 
\index{MixTankVisualization(MixTank)}
{\bf  MixTankVisualization}\\
\begin{lstlisting}[frame=none]
protected MixTankVisualization(edu.kit.pse.osip.core.model.base.MixTank tank)\end{lstlisting} %end signature
\begin{itemize}
\item{
{\bf  Description}

Creates a new visualization.
}
\item{
{\bf  Parameters}
  \begin{itemize}
   \item{
\texttt{tank} -- The tank to display}
  \end{itemize}
}%end item
\end{itemize}
}%end item
\end{itemize}
}
}
\section{\label{edu.kit.pse.osip.monitoring.view.dashboard.MonitoringMainWindow}\index{MonitoringMainWindow}Class MonitoringMainWindow}{
\rule[1em]{\hsize}{4pt}\vskip -1em
\vskip .1in 
The entry point for the monitoring view. It shows all sensor datas graphically.\vskip .1in 
\subsection{Declaration}{
\begin{lstlisting}[frame=none]
public class MonitoringMainWindow
 extends java.lang.Object\end{lstlisting}
\subsection{Fields}{
\rule[1em]{\hsize}{2pt}
\begin{itemize}
\item{
\index{mixTank}
\label{edu.kit.pse.osip.monitoring.view.dashboard.MonitoringMainWindow.mixTank}\texttt{public MixTankVisualization\ {\bf  mixTank}}
}
\item{
\index{tanks}
\label{edu.kit.pse.osip.monitoring.view.dashboard.MonitoringMainWindow.tanks}\texttt{public TankVisualization\ {\bf  tanks}}
}
\item{
\index{light}
\label{edu.kit.pse.osip.monitoring.view.dashboard.MonitoringMainWindow.light}\texttt{public Light\ {\bf  light}}
}
\item{
\index{log}
\label{edu.kit.pse.osip.monitoring.view.dashboard.MonitoringMainWindow.log}\texttt{public LoggingConsole\ {\bf  log}}
}
\item{
\index{menu}
\label{edu.kit.pse.osip.monitoring.view.dashboard.MonitoringMainWindow.menu}\texttt{public MonitoringMenu\ {\bf  menu}}
}
\end{itemize}
}
\subsection{Constructors}{
\rule[1em]{\hsize}{2pt}\vskip -2em
\vskip -2em
\begin{itemize}
\item{ 
\index{MonitoringMainWindow(javafx.stage.Stage)}
{\bf  MonitoringMainWindow}\\
\begin{lstlisting}[frame=none]
protected MonitoringMainWindow(javafx.stage.Stage primaryStage)\end{lstlisting} %end signature
\begin{itemize}
\item{
{\bf  Description}

Initializes the window.
}
\item{
{\bf  Parameters}
  \begin{itemize}
   \item{
\texttt{primaryStage} -- The primary window.}
  \end{itemize}
}%end item
\end{itemize}
}%end item
\end{itemize}
}
\subsection{Methods}{
\rule[1em]{\hsize}{2pt}\vskip -2em
\vskip -2em
\begin{itemize}
\item{ 
\index{getMonitoringMenu()}
{\bf  getMonitoringMenu}\\
\begin{lstlisting}[frame=none]
protected final MonitoringMenu getMonitoringMenu()\end{lstlisting} %end signature
\begin{itemize}
\item{
{\bf  Description}

Returns the current used menu for the monitoring view.
}
\item{{\bf  Returns} -- 
the current used menu for the monitoring view. 
}%end item
\end{itemize}
}%end item
\divideents{getTank}
\item{ 
\index{getTank(TankSelector)}
{\bf  getTank}\\
\begin{lstlisting}[frame=none]
protected final AbstractTankVisualization getTank(edu.kit.pse.osip.core.model.base.TankSelector tank)\end{lstlisting} %end signature
\begin{itemize}
\item{
{\bf  Description}

Returns the current used visualization for a specified tank.
}
\item{
{\bf  Parameters}
  \begin{itemize}
   \item{
\texttt{tank} -- The tank whose current used visualization should be returned.}
  \end{itemize}
}%end item
\item{{\bf  Returns} -- 
The current used visualization of a specified tank. 
}%end item
\end{itemize}
}%end item
\end{itemize}
}
}
\section{\label{edu.kit.pse.osip.monitoring.view.dashboard.MonitoringMenu}\index{MonitoringMenu}Class MonitoringMenu}{
\rule[1em]{\hsize}{4pt}\vskip -1em
\vskip .1in 
Contains all controls for the menu bar.\vskip .1in 
\subsection{Declaration}{
\begin{lstlisting}[frame=none]
public class MonitoringMenu
 extends javafx.scene.control.MenuBar\end{lstlisting}
\subsection{Constructors}{
\rule[1em]{\hsize}{2pt}\vskip -2em
\vskip -2em
\begin{itemize}
\item{ 
\index{MonitoringMenu()}
{\bf  MonitoringMenu}\\
\begin{lstlisting}[frame=none]
protected MonitoringMenu()\end{lstlisting} %end signature
\begin{itemize}
\item{
{\bf  Description}

Creates and initializes the menu for the monitoring view.
}
\end{itemize}
}%end item
\end{itemize}
}
\subsection{Methods}{
\rule[1em]{\hsize}{2pt}\vskip -2em
\vskip -2em
\begin{itemize}
\item{ 
\index{setMenuAboutButtonHandler(MenuAboutButtonHandler)}
{\bf  setMenuAboutButtonHandler}\\
\begin{lstlisting}[frame=none]
protected final void setMenuAboutButtonHandler(edu.kit.pse.osip.monitoring.controller.MenuAboutButtonHandler handler)\end{lstlisting} %end signature
\begin{itemize}
\item{
{\bf  Description}

Sets the handler for the about menu button.
}
\item{
{\bf  Parameters}
  \begin{itemize}
   \item{
\texttt{handler} -- The handler for the about menu button.}
  \end{itemize}
}%end item
\end{itemize}
}%end item
\divideents{setMenuHelpButtonHandler}
\item{ 
\index{setMenuHelpButtonHandler(MenuHelpButtonHandler)}
{\bf  setMenuHelpButtonHandler}\\
\begin{lstlisting}[frame=none]
protected final void setMenuHelpButtonHandler(edu.kit.pse.osip.monitoring.controller.MenuHelpButtonHandler handler)\end{lstlisting} %end signature
\begin{itemize}
\item{
{\bf  Description}

Sets the handler for the help menu button.
}
\item{
{\bf  Parameters}
  \begin{itemize}
   \item{
\texttt{handler} -- The handler for the help menu button.}
  \end{itemize}
}%end item
\end{itemize}
}%end item
\divideents{setMenuSettingsButtonHandler}
\item{ 
\index{setMenuSettingsButtonHandler(MenuSettingsButtonHandler)}
{\bf  setMenuSettingsButtonHandler}\\
\begin{lstlisting}[frame=none]
protected final void setMenuSettingsButtonHandler(edu.kit.pse.osip.monitoring.controller.MenuSettingsButtonHandler handler)\end{lstlisting} %end signature
\begin{itemize}
\item{
{\bf  Description}

Sets the handler for the settings menu button.
}
\item{
{\bf  Parameters}
  \begin{itemize}
   \item{
\texttt{handler} -- The handler for the settings menu button.}
  \end{itemize}
}%end item
\end{itemize}
}%end item
\end{itemize}
}
}
\section{\label{edu.kit.pse.osip.monitoring.view.dashboard.MonitoringViewFacade}\index{MonitoringViewFacade}Class MonitoringViewFacade}{
\rule[1em]{\hsize}{4pt}\vskip -1em
\vskip .1in 
Provides one single access point to the monitoring view.\vskip .1in 
\subsection{Declaration}{
\begin{lstlisting}[frame=none]
public class MonitoringViewFacade
 extends java.lang.Object implements edu.kit.pse.osip.monitoring.controller.MonitoringViewInterface\end{lstlisting}
\subsection{Fields}{
\rule[1em]{\hsize}{2pt}
\begin{itemize}
\item{
\index{mainWindow}
\label{edu.kit.pse.osip.monitoring.view.dashboard.MonitoringViewFacade.mainWindow}\texttt{public MonitoringMainWindow\ {\bf  mainWindow}}
}
\end{itemize}
}
\subsection{Constructors}{
\rule[1em]{\hsize}{2pt}\vskip -2em
\vskip -2em
\begin{itemize}
\item{ 
\index{MonitoringViewFacade()}
{\bf  MonitoringViewFacade}\\
\begin{lstlisting}[frame=none]
public MonitoringViewFacade()\end{lstlisting} %end signature
}%end item
\end{itemize}
}
\subsection{Methods}{
\rule[1em]{\hsize}{2pt}\vskip -2em
\vskip -2em
\begin{itemize}
\item{ 
\index{setFillLevelProgressionEnabled(TankSelector, boolean)}
{\bf  setFillLevelProgressionEnabled}\\
\begin{lstlisting}[frame=none]
public final void setFillLevelProgressionEnabled(edu.kit.pse.osip.core.model.base.TankSelector tank,boolean progressionEnabled)\end{lstlisting} %end signature
\begin{itemize}
\item{
{\bf  Description}

Enables or disables the logging of a fill level progression for a specified tank.
}
\item{
{\bf  Parameters}
  \begin{itemize}
   \item{
\texttt{tank} -- The tank whose fill level progression is logged or not.}
   \item{
\texttt{progressionEnabled} -- true if the fill level progression should be logged and false otherwise.}
  \end{itemize}
}%end item
\end{itemize}
}%end item
\divideents{setMenuAboutButtonHandler}
\item{ 
\index{setMenuAboutButtonHandler(MenuAboutButtonHandler)}
{\bf  setMenuAboutButtonHandler}\\
\begin{lstlisting}[frame=none]
public final void setMenuAboutButtonHandler(edu.kit.pse.osip.monitoring.controller.MenuAboutButtonHandler handler)\end{lstlisting} %end signature
\begin{itemize}
\item{
{\bf  Description}

Sets the handler for the about menu button.
}
\item{
{\bf  Parameters}
  \begin{itemize}
   \item{
\texttt{handler} -- The handler for the about menu button.}
  \end{itemize}
}%end item
\end{itemize}
}%end item
\divideents{setMenuHelpButtonHandler}
\item{ 
\index{setMenuHelpButtonHandler(MenuHelpButtonHandler)}
{\bf  setMenuHelpButtonHandler}\\
\begin{lstlisting}[frame=none]
public final void setMenuHelpButtonHandler(edu.kit.pse.osip.monitoring.controller.MenuHelpButtonHandler handler)\end{lstlisting} %end signature
\begin{itemize}
\item{
{\bf  Description}

Sets the handler for the help menu button.
}
\item{
{\bf  Parameters}
  \begin{itemize}
   \item{
\texttt{handler} -- The handler handles a click on the help menu button.}
  \end{itemize}
}%end item
\end{itemize}
}%end item
\divideents{setMenuSettingsButtonHandler}
\item{ 
\index{setMenuSettingsButtonHandler(MenuSettingsButtonHandler)}
{\bf  setMenuSettingsButtonHandler}\\
\begin{lstlisting}[frame=none]
public final void setMenuSettingsButtonHandler(edu.kit.pse.osip.monitoring.controller.MenuSettingsButtonHandler handler)\end{lstlisting} %end signature
\begin{itemize}
\item{
{\bf  Description}

Sets the handler for the settings menu button.
}
\item{
{\bf  Parameters}
  \begin{itemize}
   \item{
\texttt{handler} -- The handler for the settings menu button.}
  \end{itemize}
}%end item
\end{itemize}
}%end item
\divideents{setOverflowAlarmEnabled}
\item{ 
\index{setOverflowAlarmEnabled(TankSelector, boolean)}
{\bf  setOverflowAlarmEnabled}\\
\begin{lstlisting}[frame=none]
public final void setOverflowAlarmEnabled(edu.kit.pse.osip.core.model.base.TankSelector tank,boolean alarmEnabled)\end{lstlisting} %end signature
\begin{itemize}
\item{
{\bf  Description}

Enables or disables the overflow alarm for a specified tank.
}
\item{
{\bf  Parameters}
  \begin{itemize}
   \item{
\texttt{tank} -- The tank whose overflow alarm will be enabled or disabled.}
   \item{
\texttt{alarmEnabled} -- true if the alarm should be enabled and false otherwise.}
  \end{itemize}
}%end item
\end{itemize}
}%end item
\divideents{setTemperatureOverflowAlarmEnabled}
\item{ 
\index{setTemperatureOverflowAlarmEnabled(TankSelector, boolean)}
{\bf  setTemperatureOverflowAlarmEnabled}\\
\begin{lstlisting}[frame=none]
public final void setTemperatureOverflowAlarmEnabled(edu.kit.pse.osip.core.model.base.TankSelector tank,boolean alarmEnabled)\end{lstlisting} %end signature
\begin{itemize}
\item{
{\bf  Description}

Enables or disables the alarm when the temperature becomes too high for a specified tank.
}
\item{
{\bf  Parameters}
  \begin{itemize}
   \item{
\texttt{tank} -- The tank which alarm will be enabled or disabled.}
   \item{
\texttt{alarmEnabled} -- true when the alarm should be enabled. false otherwise.}
  \end{itemize}
}%end item
\end{itemize}
}%end item
\divideents{setTemperatureProgressionEnabled}
\item{ 
\index{setTemperatureProgressionEnabled(TankSelector, boolean)}
{\bf  setTemperatureProgressionEnabled}\\
\begin{lstlisting}[frame=none]
public final void setTemperatureProgressionEnabled(edu.kit.pse.osip.core.model.base.TankSelector tank,boolean progressionEnabled)\end{lstlisting} %end signature
\begin{itemize}
\item{
{\bf  Description}

Enables or disables the logging of the temperature progression of a specified tank.
}
\item{
{\bf  Parameters}
  \begin{itemize}
   \item{
\texttt{tank} -- The tank whose temperature progression is logged or not.}
   \item{
\texttt{progressionEnabled} -- true if the temperature progression should be logged and false otherwise.}
  \end{itemize}
}%end item
\end{itemize}
}%end item
\divideents{setTemperatureUnderflowAlarmEnabled}
\item{ 
\index{setTemperatureUnderflowAlarmEnabled(TankSelector, boolean)}
{\bf  setTemperatureUnderflowAlarmEnabled}\\
\begin{lstlisting}[frame=none]
public final void setTemperatureUnderflowAlarmEnabled(edu.kit.pse.osip.core.model.base.TankSelector tank,boolean alarmEnabled)\end{lstlisting} %end signature
\begin{itemize}
\item{
{\bf  Description}

Enables or disables the temperature alarm when it becomes too low for a specified tank.
}
\item{
{\bf  Parameters}
  \begin{itemize}
   \item{
\texttt{tank} -- The tank which alarm will be turned on or off.}
   \item{
\texttt{alarmEnabled} -- true when the alarm should be enabled. false otherwise.}
  \end{itemize}
}%end item
\end{itemize}
}%end item
\divideents{setUnderflowAlarmEnabled}
\item{ 
\index{setUnderflowAlarmEnabled(TankSelector, boolean)}
{\bf  setUnderflowAlarmEnabled}\\
\begin{lstlisting}[frame=none]
public final void setUnderflowAlarmEnabled(edu.kit.pse.osip.core.model.base.TankSelector tank,boolean alarmEnabled)\end{lstlisting} %end signature
\begin{itemize}
\item{
{\bf  Description}

Enables or disables the underflow alarm for a specified tank.
}
\item{
{\bf  Parameters}
  \begin{itemize}
   \item{
\texttt{tank} -- The tank whose underflow alarm will be enabled or disabled.}
   \item{
\texttt{alarmEnabled} -- true if the alarm should be enabled and false otherwise.}
  \end{itemize}
}%end item
\end{itemize}
}%end item
\divideents{showMonitoringView}
\item{ 
\index{showMonitoringView(javafx.stage.Stage)}
{\bf  showMonitoringView}\\
\begin{lstlisting}[frame=none]
public final void showMonitoringView(javafx.stage.Stage stage)\end{lstlisting} %end signature
\begin{itemize}
\item{
{\bf  Description}

Shows the monitoring view to the user.
}
\item{
{\bf  Parameters}
  \begin{itemize}
   \item{
\texttt{stage} -- The stage used for displaying the controls.}
  \end{itemize}
}%end item
\end{itemize}
}%end item
\end{itemize}
}
}
\section{\label{edu.kit.pse.osip.monitoring.view.dashboard.ProgressOverview}\index{ProgressOverview}Class ProgressOverview}{
\rule[1em]{\hsize}{4pt}\vskip -1em
\vskip .1in 
Provides access to and all controls for the temperature and fill level progressions.\vskip .1in 
\subsection{Declaration}{
\begin{lstlisting}[frame=none]
public class ProgressOverview
 extends javafx.scene.control.TabPane\end{lstlisting}
\subsection{Fields}{
\rule[1em]{\hsize}{2pt}
\begin{itemize}
\item{
\index{fillLevel}
\label{edu.kit.pse.osip.monitoring.view.dashboard.ProgressOverview.fillLevel}\texttt{public ProgressVisualization\ {\bf  fillLevel}}
}
\item{
\index{temperature}
\label{edu.kit.pse.osip.monitoring.view.dashboard.ProgressOverview.temperature}\texttt{public ProgressVisualization\ {\bf  temperature}}
}
\end{itemize}
}
\subsection{Constructors}{
\rule[1em]{\hsize}{2pt}\vskip -2em
\vskip -2em
\begin{itemize}
\item{ 
\index{ProgressOverview(Tank)}
{\bf  ProgressOverview}\\
\begin{lstlisting}[frame=none]
protected ProgressOverview(edu.kit.pse.osip.core.model.base.Tank tank)\end{lstlisting} %end signature
\begin{itemize}
\item{
{\bf  Description}

Creates and initializes all controls for the progressions.
}
\item{
{\bf  Parameters}
  \begin{itemize}
   \item{
\texttt{tank} -- The progressions are assigned to this tank.}
  \end{itemize}
}%end item
\end{itemize}
}%end item
\end{itemize}
}
\subsection{Methods}{
\rule[1em]{\hsize}{2pt}\vskip -2em
\vskip -2em
\begin{itemize}
\item{ 
\index{setFillLevelProgressEnabled(boolean)}
{\bf  setFillLevelProgressEnabled}\\
\begin{lstlisting}[frame=none]
protected final void setFillLevelProgressEnabled(boolean progressEnabled)\end{lstlisting} %end signature
\begin{itemize}
\item{
{\bf  Description}

Enables or disables the fill level progression.
}
\item{
{\bf  Parameters}
  \begin{itemize}
   \item{
\texttt{progressEnabled} -- true if the fill level progression should be enabled and false otherwise.}
  \end{itemize}
}%end item
\end{itemize}
}%end item
\divideents{setTemperatureProgressEnabled}
\item{ 
\index{setTemperatureProgressEnabled(boolean)}
{\bf  setTemperatureProgressEnabled}\\
\begin{lstlisting}[frame=none]
protected final void setTemperatureProgressEnabled(boolean progressEnabled)\end{lstlisting} %end signature
\begin{itemize}
\item{
{\bf  Description}

Enables or disables the temperature rprogression.
}
\item{
{\bf  Parameters}
  \begin{itemize}
   \item{
\texttt{progressEnabled} -- true if the temperature progression should be enabled and false otherwise.}
  \end{itemize}
}%end item
\end{itemize}
}%end item
\end{itemize}
}
}
\section{\label{edu.kit.pse.osip.monitoring.view.dashboard.ProgressVisualization}\index{ProgressVisualization}Class ProgressVisualization}{
\rule[1em]{\hsize}{4pt}\vskip -1em
\vskip .1in 
Visualises a progression.\vskip .1in 
\subsection{Declaration}{
\begin{lstlisting}[frame=none]
public class ProgressVisualization
 extends java.lang.Object implements java.util.Observer\end{lstlisting}
\subsection{Constructors}{
\rule[1em]{\hsize}{2pt}\vskip -2em
\vskip -2em
\begin{itemize}
\item{ 
\index{ProgressVisualization(String)}
{\bf  ProgressVisualization}\\
\begin{lstlisting}[frame=none]
protected ProgressVisualization(java.lang.String progressName)\end{lstlisting} %end signature
\begin{itemize}
\item{
{\bf  Description}

Creates and initializes a new chart for the progression.
}
\item{
{\bf  Parameters}
  \begin{itemize}
   \item{
\texttt{progressName} -- The name of this progression.}
  \end{itemize}
}%end item
\end{itemize}
}%end item
\end{itemize}
}
\subsection{Methods}{
\rule[1em]{\hsize}{2pt}\vskip -2em
\vskip -2em
\begin{itemize}
\item{ 
\index{getProgressChart()}
{\bf  getProgressChart}\\
\begin{lstlisting}[frame=none]
protected final <any> getProgressChart()\end{lstlisting} %end signature
\begin{itemize}
\item{
{\bf  Description}

Returns the chart showing the progression.
}
\item{{\bf  Returns} -- 
the chart showing the progression. 
}%end item
\end{itemize}
}%end item
\divideents{getProgressName}
\item{ 
\index{getProgressName()}
{\bf  getProgressName}\\
\begin{lstlisting}[frame=none]
protected final java.lang.String getProgressName()\end{lstlisting} %end signature
\begin{itemize}
\item{
{\bf  Description}

Returns the name of this progression.
}
\item{{\bf  Returns} -- 
the progression name. 
}%end item
\end{itemize}
}%end item
\divideents{setProgressEnabled}
\item{ 
\index{setProgressEnabled(boolean)}
{\bf  setProgressEnabled}\\
\begin{lstlisting}[frame=none]
protected final void setProgressEnabled(boolean progressEnabled)\end{lstlisting} %end signature
\begin{itemize}
\item{
{\bf  Description}

Enables or disables this progression.
}
\item{
{\bf  Parameters}
  \begin{itemize}
   \item{
\texttt{progressEnabled} -- true if the progression should be enabled and false otherwise.}
  \end{itemize}
}%end item
\end{itemize}
}%end item
\divideents{update}
\item{ 
\index{update(Observable, Object)}
{\bf  update}\\
\begin{lstlisting}[frame=none]
public final void update(java.util.Observable observable,java.lang.Object object)\end{lstlisting} %end signature
\begin{itemize}
\item{
{\bf  Description}

Called when the observed object has updated.
}
\item{
{\bf  Parameters}
  \begin{itemize}
   \item{
\texttt{observable} -- The observed object.}
   \item{
\texttt{object} -- The new value.}
  \end{itemize}
}%end item
\end{itemize}
}%end item
\end{itemize}
}
}
\section{\label{edu.kit.pse.osip.monitoring.view.dashboard.TankVisualization}\index{TankVisualization}Class TankVisualization}{
\rule[1em]{\hsize}{4pt}\vskip -1em
\vskip .1in 
Visualization for a normal tank.\vskip .1in 
\subsection{Declaration}{
\begin{lstlisting}[frame=none]
public class TankVisualization
 extends edu.kit.pse.osip.monitoring.view.dashboard.AbstractTankVisualization\end{lstlisting}
\subsection{Constructors}{
\rule[1em]{\hsize}{2pt}\vskip -2em
\vskip -2em
\begin{itemize}
\item{ 
\index{TankVisualization(Tank)}
{\bf  TankVisualization}\\
\begin{lstlisting}[frame=none]
protected TankVisualization(edu.kit.pse.osip.core.model.base.Tank tank)\end{lstlisting} %end signature
\begin{itemize}
\item{
{\bf  Description}

Creates a new visualization.
}
\item{
{\bf  Parameters}
  \begin{itemize}
   \item{
\texttt{tank} -- The tank to display}
  \end{itemize}
}%end item
\end{itemize}
}%end item
\end{itemize}
}
}
\section{\label{edu.kit.pse.osip.monitoring.view.dashboard.TemperatureVisualization}\index{TemperatureVisualization}Class TemperatureVisualization}{
\rule[1em]{\hsize}{4pt}\vskip -1em
\vskip .1in 
Visualises the current temperature.\vskip .1in 
\subsection{Declaration}{
\begin{lstlisting}[frame=none]
public class TemperatureVisualization
 extends java.lang.Object implements java.util.Observer\end{lstlisting}
\subsection{Constructors}{
\rule[1em]{\hsize}{2pt}\vskip -2em
\vskip -2em
\begin{itemize}
\item{ 
\index{TemperatureVisualization()}
{\bf  TemperatureVisualization}\\
\begin{lstlisting}[frame=none]
protected TemperatureVisualization()\end{lstlisting} %end signature
\begin{itemize}
\item{
{\bf  Description}

Creates and initializes a new temperature visualization.
}
\end{itemize}
}%end item
\end{itemize}
}
\subsection{Methods}{
\rule[1em]{\hsize}{2pt}\vskip -2em
\vskip -2em
\begin{itemize}
\item{ 
\index{getTemperatureBar()}
{\bf  getTemperatureBar}\\
\begin{lstlisting}[frame=none]
protected final javafx.scene.control.Slider getTemperatureBar()\end{lstlisting} %end signature
\begin{itemize}
\item{
{\bf  Description}

Slider for showing the actual temperature.
}
\item{{\bf  Returns} -- 
the slider showing the actual temperature. 
}%end item
\end{itemize}
}%end item
\divideents{getTemperatureLabel}
\item{ 
\index{getTemperatureLabel()}
{\bf  getTemperatureLabel}\\
\begin{lstlisting}[frame=none]
protected final javafx.scene.control.Label getTemperatureLabel()\end{lstlisting} %end signature
\begin{itemize}
\item{
{\bf  Description}

Returns the label for showing the temperature section.
}
\item{{\bf  Returns} -- 
the label showing the temperature section. 
}%end item
\end{itemize}
}%end item
\divideents{update}
\item{ 
\index{update(Observable, Object)}
{\bf  update}\\
\begin{lstlisting}[frame=none]
public final void update(java.util.Observable observable,java.lang.Object object)\end{lstlisting} %end signature
\begin{itemize}
\item{
{\bf  Description}

Called when the observed object is updated.
}
\item{
{\bf  Parameters}
  \begin{itemize}
   \item{
\texttt{observable} -- The observed object.}
   \item{
\texttt{object} -- The new value.}
  \end{itemize}
}%end item
\end{itemize}
}%end item
\end{itemize}
}
}
}
\chapter{Package edu.kit.pse.osip.core.io.networking}{
\label{edu.kit.pse.osip.core.io.networking}\section{\label{edu.kit.pse.osip.core.io.networking.RemoteMachine}\index{RemoteMachine}Class RemoteMachine}{
\rule[1em]{\hsize}{4pt}\vskip -1em
\vskip .1in 
Class for saving connection and networking details\vskip .1in 
\subsection{Declaration}{
\begin{lstlisting}[frame=none]
public class RemoteMachine
 extends java.lang.Object\end{lstlisting}
\subsection{Constructors}{
\rule[1em]{\hsize}{2pt}\vskip -2em
\vskip -2em
\begin{itemize}
\item{ 
\index{RemoteMachine(String, int)}
{\bf  RemoteMachine}\\
\begin{lstlisting}[frame=none]
public RemoteMachine(java.lang.String hostname,int port)\end{lstlisting} %end signature
\begin{itemize}
\item{
{\bf  Description}

Constructor of RemoteMachine
}
\item{
{\bf  Parameters}
  \begin{itemize}
   \item{
\texttt{hostname} -- host name of RemoteMachine}
   \item{
\texttt{port} -- The port of the RemoteMachine}
  \end{itemize}
}%end item
\end{itemize}
}%end item
\end{itemize}
}
\subsection{Methods}{
\rule[1em]{\hsize}{2pt}\vskip -2em
\vskip -2em
\begin{itemize}
\item{ 
\index{getHostname()}
{\bf  getHostname}\\
\begin{lstlisting}[frame=none]
public final java.lang.String getHostname()\end{lstlisting} %end signature
\begin{itemize}
\item{
{\bf  Description}

Getter method of the hostname @return hostname of RemoteMachine
}
\end{itemize}
}%end item
\divideents{getPort}
\item{ 
\index{getPort()}
{\bf  getPort}\\
\begin{lstlisting}[frame=none]
public final int getPort()\end{lstlisting} %end signature
\begin{itemize}
\item{
{\bf  Description}

Getter method of port @return port of RemoteMachine
}
\end{itemize}
}%end item
\end{itemize}
}
}
}
\chapter{Package edu.kit.pse.osip.core.io.files}{
\label{edu.kit.pse.osip.core.io.files}\section{\label{edu.kit.pse.osip.core.io.files.BaseParser}\index{BaseParser}Class BaseParser}{
\rule[1em]{\hsize}{4pt}\vskip -1em
\vskip .1in 
Basic parser class\vskip .1in 
\subsection{Declaration}{
\begin{lstlisting}[frame=none]
public class BaseParser
 extends java.lang.Object\end{lstlisting}
\subsection{All known subclasses}{ScenarioParser\small{\refdefined{edu.kit.pse.osip.core.io.files.ScenarioParser}}, ExtendedParser\small{\refdefined{edu.kit.pse.osip.core.io.files.ExtendedParser}}}
\subsection{Constructors}{
\rule[1em]{\hsize}{2pt}\vskip -2em
\vskip -2em
\begin{itemize}
\item{ 
\index{BaseParser(String)}
{\bf  BaseParser}\\
\begin{lstlisting}[frame=none]
public BaseParser(java.lang.String toParse)\end{lstlisting} %end signature
\begin{itemize}
\item{
{\bf  Description}

Constructor of BaseParser
}
\item{
{\bf  Parameters}
  \begin{itemize}
   \item{
\texttt{toParse} -- The string that should be parsed}
  \end{itemize}
}%end item
\end{itemize}
}%end item
\end{itemize}
}
\subsection{Methods}{
\rule[1em]{\hsize}{2pt}\vskip -2em
\vskip -2em
\begin{itemize}
\item{ 
\index{available()}
{\bf  available}\\
\begin{lstlisting}[frame=none]
public final boolean available()\end{lstlisting} %end signature
\begin{itemize}
\item{
{\bf  Description}

Check if there are more signs to read @return true if there are more
}
\end{itemize}
}%end item
\divideents{check}
\item{ 
\index{check(char)}
{\bf  check}\\
\begin{lstlisting}[frame=none]
protected final void check(char c)\end{lstlisting} %end signature
\begin{itemize}
\item{
{\bf  Description}

If peek != char then die
}
\item{
{\bf  Parameters}
  \begin{itemize}
   \item{
\texttt{c} -- c char to check}
  \end{itemize}
}%end item
\item{{\bf  Throws}
  \begin{itemize}
   \item{\vskip -.6ex \texttt{edu.kit.pse.osip.core.io.files.ParserException} -- If something goes wrong}
  \end{itemize}
}%end item
\end{itemize}
}%end item
\divideents{consume}
\item{ 
\index{consume(int)}
{\bf  consume}\\
\begin{lstlisting}[frame=none]
protected final void consume(int n)\end{lstlisting} %end signature
\begin{itemize}
\item{
{\bf  Description}

Read multiple symbols
}
\item{
{\bf  Parameters}
  \begin{itemize}
   \item{
\texttt{n} -- n times read}
  \end{itemize}
}%end item
\item{{\bf  Throws}
  \begin{itemize}
   \item{\vskip -.6ex \texttt{edu.kit.pse.osip.core.io.files.ParserException} -- If something goes wrong}
  \end{itemize}
}%end item
\end{itemize}
}%end item
\divideents{die}
\item{ 
\index{die()}
{\bf  die}\\
\begin{lstlisting}[frame=none]
protected final void die()\end{lstlisting} %end signature
\begin{itemize}
\item{
{\bf  Description}

Generate and throw exception (initialize with line number etc)
}
\item{{\bf  Throws}
  \begin{itemize}
   \item{\vskip -.6ex \texttt{edu.kit.pse.osip.core.io.files.ParserException} -- Always.}
  \end{itemize}
}%end item
\end{itemize}
}%end item
\divideents{peek}
\item{ 
\index{peek()}
{\bf  peek}\\
\begin{lstlisting}[frame=none]
protected final char peek()\end{lstlisting} %end signature
\begin{itemize}
\item{
{\bf  Description}

Look at next char
}
\item{{\bf  Returns} -- 
peeked char 
}%end item
\item{{\bf  Throws}
  \begin{itemize}
   \item{\vskip -.6ex \texttt{edu.kit.pse.osip.core.io.files.ParserException} -- If something goes wrong}
  \end{itemize}
}%end item
\end{itemize}
}%end item
\divideents{pop}
\item{ 
\index{pop()}
{\bf  pop}\\
\begin{lstlisting}[frame=none]
protected final char pop()\end{lstlisting} %end signature
\begin{itemize}
\item{
{\bf  Description}

Look at next char and remove it
}
\item{{\bf  Returns} -- 
read char 
}%end item
\item{{\bf  Throws}
  \begin{itemize}
   \item{\vskip -.6ex \texttt{edu.kit.pse.osip.core.io.files.ParserException} -- If something goes wrong}
  \end{itemize}
}%end item
\end{itemize}
}%end item
\divideents{skipComments}
\item{ 
\index{skipComments()}
{\bf  skipComments}\\
\begin{lstlisting}[frame=none]
public final void skipComments()\end{lstlisting} %end signature
\begin{itemize}
\item{
{\bf  Description}

Skip comments
}
\item{{\bf  Throws}
  \begin{itemize}
   \item{\vskip -.6ex \texttt{edu.kit.pse.osip.core.io.files.ParserException} -- If something goes wrong}
  \end{itemize}
}%end item
\end{itemize}
}%end item
\divideents{skipWhitespaces}
\item{ 
\index{skipWhitespaces()}
{\bf  skipWhitespaces}\\
\begin{lstlisting}[frame=none]
public final void skipWhitespaces()\end{lstlisting} %end signature
\begin{itemize}
\item{
{\bf  Description}

Skip whitespaces
}
\item{{\bf  Throws}
  \begin{itemize}
   \item{\vskip -.6ex \texttt{edu.kit.pse.osip.core.io.files.ParserException} -- If something goes wrong}
  \end{itemize}
}%end item
\end{itemize}
}%end item
\end{itemize}
}
}
\section{\label{edu.kit.pse.osip.core.io.files.ClientSettingsWrapper}\index{ClientSettingsWrapper}Class ClientSettingsWrapper}{
\rule[1em]{\hsize}{4pt}\vskip -1em
\vskip .1in 
Wrapper for client settings file where settings of networking and monitoring are stored\vskip .1in 
\subsection{Declaration}{
\begin{lstlisting}[frame=none]
public class ClientSettingsWrapper
 extends java.lang.Object\end{lstlisting}
\subsection{Constructors}{
\rule[1em]{\hsize}{2pt}\vskip -2em
\vskip -2em
\begin{itemize}
\item{ 
\index{ClientSettingsWrapper()}
{\bf  ClientSettingsWrapper}\\
\begin{lstlisting}[frame=none]
public ClientSettingsWrapper()\end{lstlisting} %end signature
}%end item
\end{itemize}
}
\subsection{Methods}{
\rule[1em]{\hsize}{2pt}\vskip -2em
\vskip -2em
\begin{itemize}
\item{ 
\index{ClientSettingWrapper(File)}
{\bf  ClientSettingWrapper}\\
\begin{lstlisting}[frame=none]
public final void ClientSettingWrapper(java.io.File settingsFile)\end{lstlisting} %end signature
\begin{itemize}
\item{
{\bf  Description}

Constructor of ClientSettingsWrapper
}
\item{
{\bf  Parameters}
  \begin{itemize}
   \item{
\texttt{settingsFile} -- Where to save the settings}
  \end{itemize}
}%end item
\end{itemize}
}%end item
\divideents{getFetchInterval}
\item{ 
\index{getFetchInterval()}
{\bf  getFetchInterval}\\
\begin{lstlisting}[frame=none]
public final int getFetchInterval()\end{lstlisting} %end signature
\begin{itemize}
\item{
{\bf  Description}

Getter method of fetch interval
}
\item{{\bf  Returns} -- 
fetch interval in ms 
}%end item
\end{itemize}
}%end item
\divideents{getFillLevelDiagram}
\item{ 
\index{getFillLevelDiagram(TankSelector)}
{\bf  getFillLevelDiagram}\\
\begin{lstlisting}[frame=none]
public final boolean getFillLevelDiagram(edu.kit.pse.osip.core.model.base.TankSelector tank)\end{lstlisting} %end signature
\begin{itemize}
\item{
{\bf  Description}

Getter method of fill level diagram @return true if diagram is enabled
}
\item{
{\bf  Parameters}
  \begin{itemize}
   \item{
\texttt{tank} -- The tank to get the value for}
  \end{itemize}
}%end item
\end{itemize}
}%end item
\divideents{getHostname}
\item{ 
\index{getHostname(TankSelector)}
{\bf  getHostname}\\
\begin{lstlisting}[frame=none]
public final java.lang.String getHostname(edu.kit.pse.osip.core.model.base.TankSelector tank)\end{lstlisting} %end signature
\begin{itemize}
\item{
{\bf  Description}

Get the hostname or IP adress
}
\item{
{\bf  Parameters}
  \begin{itemize}
   \item{
\texttt{tank} -- The tank to get the hostname}
  \end{itemize}
}%end item
\item{{\bf  Returns} -- 
the hostname of IP adress 
}%end item
\end{itemize}
}%end item
\divideents{getOverflowAlarm}
\item{ 
\index{getOverflowAlarm(TankSelector)}
{\bf  getOverflowAlarm}\\
\begin{lstlisting}[frame=none]
public final boolean getOverflowAlarm(edu.kit.pse.osip.core.model.base.TankSelector tank)\end{lstlisting} %end signature
\begin{itemize}
\item{
{\bf  Description}

Getter method of overflow alarm
}
\item{
{\bf  Parameters}
  \begin{itemize}
   \item{
\texttt{tank} -- The tank to get the value for}
  \end{itemize}
}%end item
\item{{\bf  Returns} -- 
true if alarm is enabled 
}%end item
\end{itemize}
}%end item
\divideents{getPort}
\item{ 
\index{getPort(TankSelector)}
{\bf  getPort}\\
\begin{lstlisting}[frame=none]
public final int getPort(edu.kit.pse.osip.core.model.base.TankSelector tank)\end{lstlisting} %end signature
\begin{itemize}
\item{
{\bf  Description}

get the port
}
\item{
{\bf  Parameters}
  \begin{itemize}
   \item{
\texttt{tank} -- The tank to get the port of}
  \end{itemize}
}%end item
\item{{\bf  Returns} -- 
the port number 
}%end item
\end{itemize}
}%end item
\divideents{getTempDiagram}
\item{ 
\index{getTempDiagram(TankSelector)}
{\bf  getTempDiagram}\\
\begin{lstlisting}[frame=none]
public final boolean getTempDiagram(edu.kit.pse.osip.core.model.base.TankSelector tank)\end{lstlisting} %end signature
\begin{itemize}
\item{
{\bf  Description}

Getter method of temperature diagram @return true if diagram is enabled
}
\item{
{\bf  Parameters}
  \begin{itemize}
   \item{
\texttt{tank} -- The tank to get the value for}
  \end{itemize}
}%end item
\end{itemize}
}%end item
\divideents{getUnderflowAlarm}
\item{ 
\index{getUnderflowAlarm(TankSelector)}
{\bf  getUnderflowAlarm}\\
\begin{lstlisting}[frame=none]
public final boolean getUnderflowAlarm(edu.kit.pse.osip.core.model.base.TankSelector tank)\end{lstlisting} %end signature
\begin{itemize}
\item{
{\bf  Description}

Getter method of underflow alarm
}
\item{
{\bf  Parameters}
  \begin{itemize}
   \item{
\texttt{tank} -- The tank to get the value for}
  \end{itemize}
}%end item
\item{{\bf  Returns} -- 
true if alarm is enabled 
}%end item
\end{itemize}
}%end item
\divideents{saveSettings}
\item{ 
\index{saveSettings()}
{\bf  saveSettings}\\
\begin{lstlisting}[frame=none]
public final void saveSettings()\end{lstlisting} %end signature
\begin{itemize}
\item{
{\bf  Description}

Saves settings in file
}
\end{itemize}
}%end item
\divideents{setFetchInterval}
\item{ 
\index{setFetchInterval(int)}
{\bf  setFetchInterval}\\
\begin{lstlisting}[frame=none]
public final void setFetchInterval(int intervalMs)\end{lstlisting} %end signature
\begin{itemize}
\item{
{\bf  Description}

Setter method of fetch interval
}
\item{
{\bf  Parameters}
  \begin{itemize}
   \item{
\texttt{intervalMs} -- fetch interval}
  \end{itemize}
}%end item
\end{itemize}
}%end item
\divideents{setFillLevelDiagram}
\item{ 
\index{setFillLevelDiagram(TankSelector, boolean)}
{\bf  setFillLevelDiagram}\\
\begin{lstlisting}[frame=none]
public final void setFillLevelDiagram(edu.kit.pse.osip.core.model.base.TankSelector tank,boolean enabled)\end{lstlisting} %end signature
\begin{itemize}
\item{
{\bf  Description}

Setter method of fill level diagram
}
\item{
{\bf  Parameters}
  \begin{itemize}
   \item{
\texttt{tank} -- The tank to save the value for}
   \item{
\texttt{enabled} -- true if diagram is enabled}
  \end{itemize}
}%end item
\end{itemize}
}%end item
\divideents{setOverflowAlarm}
\item{ 
\index{setOverflowAlarm(TankSelector, boolean)}
{\bf  setOverflowAlarm}\\
\begin{lstlisting}[frame=none]
public final void setOverflowAlarm(edu.kit.pse.osip.core.model.base.TankSelector tank,boolean enabled)\end{lstlisting} %end signature
\begin{itemize}
\item{
{\bf  Description}

Setter method of overflow alarm
}
\item{
{\bf  Parameters}
  \begin{itemize}
   \item{
\texttt{tank} -- The tank to save the value for}
   \item{
\texttt{enabled} -- true if alarm is enabled}
  \end{itemize}
}%end item
\end{itemize}
}%end item
\divideents{setServerHostname}
\item{ 
\index{setServerHostname(TankSelector, String)}
{\bf  setServerHostname}\\
\begin{lstlisting}[frame=none]
public final void setServerHostname(edu.kit.pse.osip.core.model.base.TankSelector tank,java.lang.String hostname)\end{lstlisting} %end signature
\begin{itemize}
\item{
{\bf  Description}

Setter method of the server hostname
}
\item{
{\bf  Parameters}
  \begin{itemize}
   \item{
\texttt{tank} -- The tank to save the hostname for}
   \item{
\texttt{hostname} -- The server hostname}
  \end{itemize}
}%end item
\end{itemize}
}%end item
\divideents{setServerPort}
\item{ 
\index{setServerPort(TankSelector, int)}
{\bf  setServerPort}\\
\begin{lstlisting}[frame=none]
public final void setServerPort(edu.kit.pse.osip.core.model.base.TankSelector tank,int portServer)\end{lstlisting} %end signature
\begin{itemize}
\item{
{\bf  Description}

Setter method of server port
}
\item{
{\bf  Parameters}
  \begin{itemize}
   \item{
\texttt{tank} -- The tank to save the port for}
   \item{
\texttt{portServer} -- The port}
  \end{itemize}
}%end item
\end{itemize}
}%end item
\divideents{setTempDiagram}
\item{ 
\index{setTempDiagram(TankSelector, boolean)}
{\bf  setTempDiagram}\\
\begin{lstlisting}[frame=none]
public final void setTempDiagram(edu.kit.pse.osip.core.model.base.TankSelector tank,boolean enabled)\end{lstlisting} %end signature
\begin{itemize}
\item{
{\bf  Description}

Setter method of temperature diagram
}
\item{
{\bf  Parameters}
  \begin{itemize}
   \item{
\texttt{tank} -- The tank to save the value for}
   \item{
\texttt{enabled} -- true if diagram is enabled}
  \end{itemize}
}%end item
\end{itemize}
}%end item
\divideents{setUnderflowAlarm}
\item{ 
\index{setUnderflowAlarm(TankSelector, boolean)}
{\bf  setUnderflowAlarm}\\
\begin{lstlisting}[frame=none]
public final void setUnderflowAlarm(edu.kit.pse.osip.core.model.base.TankSelector tank,boolean enabled)\end{lstlisting} %end signature
\begin{itemize}
\item{
{\bf  Description}

Setter method of underflow alarm
}
\item{
{\bf  Parameters}
  \begin{itemize}
   \item{
\texttt{tank} -- The tank to save the value for}
   \item{
\texttt{enabled} -- true if alarm is enabled}
  \end{itemize}
}%end item
\end{itemize}
}%end item
\end{itemize}
}
}
\section{\label{edu.kit.pse.osip.core.io.files.ExtendedParser}\index{ExtendedParser}Class ExtendedParser}{
\rule[1em]{\hsize}{4pt}\vskip -1em
\vskip .1in 
Extended parser for more complex expressions\vskip .1in 
\subsection{Declaration}{
\begin{lstlisting}[frame=none]
public class ExtendedParser
 extends edu.kit.pse.osip.core.io.files.BaseParser\end{lstlisting}
\subsection{All known subclasses}{ScenarioParser\small{\refdefined{edu.kit.pse.osip.core.io.files.ScenarioParser}}}
\subsection{Constructors}{
\rule[1em]{\hsize}{2pt}\vskip -2em
\vskip -2em
\begin{itemize}
\item{ 
\index{ExtendedParser(String)}
{\bf  ExtendedParser}\\
\begin{lstlisting}[frame=none]
public ExtendedParser(java.lang.String toParse)\end{lstlisting} %end signature
\begin{itemize}
\item{
{\bf  Description}

Constructor of ExtendedParser
}
\item{
{\bf  Parameters}
  \begin{itemize}
   \item{
\texttt{toParse} -- The string that should be parsed}
  \end{itemize}
}%end item
\end{itemize}
}%end item
\end{itemize}
}
\subsection{Methods}{
\rule[1em]{\hsize}{2pt}\vskip -2em
\vskip -2em
\begin{itemize}
\item{ 
\index{readCommand()}
{\bf  readCommand}\\
\begin{lstlisting}[frame=none]
public final java.util.function.Consumer readCommand()\end{lstlisting} %end signature
\begin{itemize}
\item{
{\bf  Description}

Reads a command in the input file
}
\item{{\bf  Returns} -- 
The command 
}%end item
\item{{\bf  Throws}
  \begin{itemize}
   \item{\vskip -.6ex \texttt{edu.kit.pse.osip.core.io.files.ParserException} -- If something goes wrong}
  \end{itemize}
}%end item
\end{itemize}
}%end item
\divideents{readExpression}
\item{ 
\index{readExpression()}
{\bf  readExpression}\\
\begin{lstlisting}[frame=none]
protected final float readExpression()\end{lstlisting} %end signature
\begin{itemize}
\item{
{\bf  Description}

Reads expression
}
\item{{\bf  Returns} -- 
the expression 
}%end item
\item{{\bf  Throws}
  \begin{itemize}
   \item{\vskip -.6ex \texttt{edu.kit.pse.osip.core.io.files.ParserException} -- If something goes wrong}
  \end{itemize}
}%end item
\end{itemize}
}%end item
\divideents{readFactor}
\item{ 
\index{readFactor()}
{\bf  readFactor}\\
\begin{lstlisting}[frame=none]
protected final float readFactor()\end{lstlisting} %end signature
\begin{itemize}
\item{
{\bf  Description}

Reads factor
}
\item{{\bf  Returns} -- 
the factor 
}%end item
\item{{\bf  Throws}
  \begin{itemize}
   \item{\vskip -.6ex \texttt{edu.kit.pse.osip.core.io.files.ParserException} -- If something goes wrong}
  \end{itemize}
}%end item
\end{itemize}
}%end item
\divideents{readNumber}
\item{ 
\index{readNumber()}
{\bf  readNumber}\\
\begin{lstlisting}[frame=none]
protected final float readNumber()\end{lstlisting} %end signature
\begin{itemize}
\item{
{\bf  Description}

Reads number
}
\item{{\bf  Returns} -- 
the number 
}%end item
\item{{\bf  Throws}
  \begin{itemize}
   \item{\vskip -.6ex \texttt{edu.kit.pse.osip.core.io.files.ParserException} -- If something goes wrong}
  \end{itemize}
}%end item
\end{itemize}
}%end item
\divideents{readTerm}
\item{ 
\index{readTerm()}
{\bf  readTerm}\\
\begin{lstlisting}[frame=none]
protected final float readTerm()\end{lstlisting} %end signature
\begin{itemize}
\item{
{\bf  Description}

Reads term
}
\item{{\bf  Returns} -- 
the term 
}%end item
\item{{\bf  Throws}
  \begin{itemize}
   \item{\vskip -.6ex \texttt{edu.kit.pse.osip.core.io.files.ParserException} -- If something goes wrong}
  \end{itemize}
}%end item
\end{itemize}
}%end item
\end{itemize}
}
}
\section{\label{edu.kit.pse.osip.core.io.files.ParserException}\index{ParserException}Class ParserException}{
\rule[1em]{\hsize}{4pt}\vskip -1em
\vskip .1in 
Exception class for exceptions in parsing\vskip .1in 
\subsection{Declaration}{
\begin{lstlisting}[frame=none]
public class ParserException
 extends java.lang.Object\end{lstlisting}
\subsection{Constructors}{
\rule[1em]{\hsize}{2pt}\vskip -2em
\vskip -2em
\begin{itemize}
\item{ 
\index{ParserException(String, int, int)}
{\bf  ParserException}\\
\begin{lstlisting}[frame=none]
public ParserException(java.lang.String msg,int line,int character)\end{lstlisting} %end signature
\begin{itemize}
\item{
{\bf  Description}

Constructor of ParserException
}
\item{
{\bf  Parameters}
  \begin{itemize}
   \item{
\texttt{msg} -- Message}
   \item{
\texttt{line} -- The line in which the exception occured}
   \item{
\texttt{character} -- The character within the line where the error occured}
  \end{itemize}
}%end item
\end{itemize}
}%end item
\end{itemize}
}
}
\section{\label{edu.kit.pse.osip.core.io.files.ScenarioFile}\index{ScenarioFile}Class ScenarioFile}{
\rule[1em]{\hsize}{4pt}\vskip -1em
\vskip .1in 
ScenarioFile saves scenario settings and creates scenarios from the settings\vskip .1in 
\subsection{Declaration}{
\begin{lstlisting}[frame=none]
public class ScenarioFile
 extends java.lang.Object\end{lstlisting}
\subsection{Fields}{
\rule[1em]{\hsize}{2pt}
\begin{itemize}
\item{
\index{parser}
\label{edu.kit.pse.osip.core.io.files.ScenarioFile.parser}\texttt{public ScenarioParser\ {\bf  parser}}
}
\end{itemize}
}
\subsection{Constructors}{
\rule[1em]{\hsize}{2pt}\vskip -2em
\vskip -2em
\begin{itemize}
\item{ 
\index{ScenarioFile(File)}
{\bf  ScenarioFile}\\
\begin{lstlisting}[frame=none]
public ScenarioFile(java.io.File file)\end{lstlisting} %end signature
\begin{itemize}
\item{
{\bf  Description}

Constructor of ScenarioFile
}
\item{
{\bf  Parameters}
  \begin{itemize}
   \item{
\texttt{file} -- The scenario definition file}
  \end{itemize}
}%end item
\end{itemize}
}%end item
\end{itemize}
}
\subsection{Methods}{
\rule[1em]{\hsize}{2pt}\vskip -2em
\vskip -2em
\begin{itemize}
\item{ 
\index{getScenario()}
{\bf  getScenario}\\
\begin{lstlisting}[frame=none]
public final edu.kit.pse.osip.core.model.behavior.Scenario getScenario()\end{lstlisting} %end signature
\begin{itemize}
\item{
{\bf  Description}

Getter method of scenario @return parsed scenario
}
\end{itemize}
}%end item
\end{itemize}
}
}
\section{\label{edu.kit.pse.osip.core.io.files.ScenarioParser}\index{ScenarioParser}Class ScenarioParser}{
\rule[1em]{\hsize}{4pt}\vskip -1em
\vskip .1in 
Parser for OSIP scenarios\vskip .1in 
\subsection{Declaration}{
\begin{lstlisting}[frame=none]
public class ScenarioParser
 extends edu.kit.pse.osip.core.io.files.ExtendedParser\end{lstlisting}
\subsection{Constructors}{
\rule[1em]{\hsize}{2pt}\vskip -2em
\vskip -2em
\begin{itemize}
\item{ 
\index{ScenarioParser(String)}
{\bf  ScenarioParser}\\
\begin{lstlisting}[frame=none]
public ScenarioParser(java.lang.String toParse)\end{lstlisting} %end signature
\begin{itemize}
\item{
{\bf  Description}

Constructor of ScenarioParser
}
\item{
{\bf  Parameters}
  \begin{itemize}
   \item{
\texttt{toParse} -- @param toParse String to parse}
  \end{itemize}
}%end item
\end{itemize}
}%end item
\end{itemize}
}
\subsection{Methods}{
\rule[1em]{\hsize}{2pt}\vskip -2em
\vskip -2em
\begin{itemize}
\item{ 
\index{readScenario()}
{\bf  readScenario}\\
\begin{lstlisting}[frame=none]
public final edu.kit.pse.osip.core.model.behavior.Scenario readScenario()\end{lstlisting} %end signature
\begin{itemize}
\item{
{\bf  Description}

Parse scenario @return parsed scenario
}
\item{{\bf  Throws}
  \begin{itemize}
   \item{\vskip -.6ex \texttt{edu.kit.pse.osip.core.io.files.ParserException} -- If something goes wrong}
  \end{itemize}
}%end item
\end{itemize}
}%end item
\end{itemize}
}
}
\section{\label{edu.kit.pse.osip.core.io.files.ServerSettingsWrapper}\index{ServerSettingsWrapper}Class ServerSettingsWrapper}{
\rule[1em]{\hsize}{4pt}\vskip -1em
\vskip .1in 
Wrapper for server settings file where ports of tanks are stored\vskip .1in 
\subsection{Declaration}{
\begin{lstlisting}[frame=none]
public class ServerSettingsWrapper
 extends java.lang.Object\end{lstlisting}
\subsection{Constructors}{
\rule[1em]{\hsize}{2pt}\vskip -2em
\vskip -2em
\begin{itemize}
\item{ 
\index{ServerSettingsWrapper(File)}
{\bf  ServerSettingsWrapper}\\
\begin{lstlisting}[frame=none]
public ServerSettingsWrapper(java.io.File settingsFile)\end{lstlisting} %end signature
\begin{itemize}
\item{
{\bf  Description}

Constructor of ServerSettingsWrapper
}
\item{
{\bf  Parameters}
  \begin{itemize}
   \item{
\texttt{settingsFile} -- The file to save the settings to}
  \end{itemize}
}%end item
\end{itemize}
}%end item
\end{itemize}
}
\subsection{Methods}{
\rule[1em]{\hsize}{2pt}\vskip -2em
\vskip -2em
\begin{itemize}
\item{ 
\index{getServerPort(TankSelector)}
{\bf  getServerPort}\\
\begin{lstlisting}[frame=none]
public final int getServerPort(edu.kit.pse.osip.core.model.base.TankSelector tank)\end{lstlisting} %end signature
\begin{itemize}
\item{
{\bf  Description}

Getter method of server port @return The saved port
}
\item{
{\bf  Parameters}
  \begin{itemize}
   \item{
\texttt{tank} -- The tank to get the value for}
  \end{itemize}
}%end item
\end{itemize}
}%end item
\divideents{saveSettings}
\item{ 
\index{saveSettings()}
{\bf  saveSettings}\\
\begin{lstlisting}[frame=none]
public final void saveSettings()\end{lstlisting} %end signature
\begin{itemize}
\item{
{\bf  Description}

Saves settings in file
}
\end{itemize}
}%end item
\divideents{setServerPort}
\item{ 
\index{setServerPort(TankSelector, int)}
{\bf  setServerPort}\\
\begin{lstlisting}[frame=none]
public final void setServerPort(edu.kit.pse.osip.core.model.base.TankSelector tank,int port)\end{lstlisting} %end signature
\begin{itemize}
\item{
{\bf  Description}

Setter method of server port
}
\item{
{\bf  Parameters}
  \begin{itemize}
   \item{
\texttt{tank} -- The tank to save the value for}
   \item{
\texttt{port} -- The port to save}
  \end{itemize}
}%end item
\end{itemize}
}%end item
\end{itemize}
}
}
}
\chapter{Package edu.kit.pse.osip.core.opcua.client}{
\label{edu.kit.pse.osip.core.opcua.client}\section{\label{edu.kit.pse.osip.core.opcua.client.BooleanReceivedListener}\index{BooleanReceivedListener@\textit{ BooleanReceivedListener}}Interface BooleanReceivedListener}{
\rule[1em]{\hsize}{4pt}\vskip -1em
\vskip .1in 
Listener that is called when a boolean value was loaded from the server.\vskip .1in 
\subsection{Declaration}{
\begin{lstlisting}[frame=none]
public interface BooleanReceivedListener
 extends ReceivedListener\end{lstlisting}
\subsection{Methods}{
\rule[1em]{\hsize}{2pt}\vskip -2em
\vskip -2em
\begin{itemize}
\item{ 
\index{onReceived(boolean)}
{\bf  onReceived}\\
\begin{lstlisting}[frame=none]
void onReceived(boolean value)\end{lstlisting} %end signature
\begin{itemize}
\item{
{\bf  Description}

A boolean was received from the server
}
\item{
{\bf  Parameters}
  \begin{itemize}
   \item{
\texttt{value} -- The received value}
  \end{itemize}
}%end item
\end{itemize}
}%end item
\end{itemize}
}
}
\section{\label{edu.kit.pse.osip.core.opcua.client.FloatReceivedListener}\index{FloatReceivedListener@\textit{ FloatReceivedListener}}Interface FloatReceivedListener}{
\rule[1em]{\hsize}{4pt}\vskip -1em
\vskip .1in 
Listener that is called when a float value was loaded from the server.\vskip .1in 
\subsection{Declaration}{
\begin{lstlisting}[frame=none]
public interface FloatReceivedListener
 extends ReceivedListener\end{lstlisting}
\subsection{Methods}{
\rule[1em]{\hsize}{2pt}\vskip -2em
\vskip -2em
\begin{itemize}
\item{ 
\index{onReceived(float)}
{\bf  onReceived}\\
\begin{lstlisting}[frame=none]
void onReceived(float value)\end{lstlisting} %end signature
\begin{itemize}
\item{
{\bf  Description}

A float was received from the server
}
\item{
{\bf  Parameters}
  \begin{itemize}
   \item{
\texttt{value} -- The received value}
  \end{itemize}
}%end item
\end{itemize}
}%end item
\end{itemize}
}
}
\section{\label{edu.kit.pse.osip.core.opcua.client.IntReceivedListener}\index{IntReceivedListener@\textit{ IntReceivedListener}}Interface IntReceivedListener}{
\rule[1em]{\hsize}{4pt}\vskip -1em
\vskip .1in 
Listener that is called when an integer value was loaded from the server.\vskip .1in 
\subsection{Declaration}{
\begin{lstlisting}[frame=none]
public interface IntReceivedListener
 extends ReceivedListener\end{lstlisting}
\subsection{Methods}{
\rule[1em]{\hsize}{2pt}\vskip -2em
\vskip -2em
\begin{itemize}
\item{ 
\index{onReceived(int)}
{\bf  onReceived}\\
\begin{lstlisting}[frame=none]
void onReceived(int value)\end{lstlisting} %end signature
\begin{itemize}
\item{
{\bf  Description}

An int was received from the server
}
\item{
{\bf  Parameters}
  \begin{itemize}
   \item{
\texttt{value} -- The received value}
  \end{itemize}
}%end item
\end{itemize}
}%end item
\end{itemize}
}
}
\section{\label{edu.kit.pse.osip.core.opcua.client.ReceivedListener}\index{ReceivedListener@\textit{ ReceivedListener}}Interface ReceivedListener}{
\rule[1em]{\hsize}{4pt}\vskip -1em
\vskip .1in 
Listener that is called when a value was received from the server.\vskip .1in 
\subsection{Declaration}{
\begin{lstlisting}[frame=none]
public interface ReceivedListener
\end{lstlisting}
\subsection{All known subinterfaces}{BooleanReceivedListener\small{\refdefined{edu.kit.pse.osip.core.opcua.client.BooleanReceivedListener}}, FloatReceivedListener\small{\refdefined{edu.kit.pse.osip.core.opcua.client.FloatReceivedListener}}, IntReceivedListener\small{\refdefined{edu.kit.pse.osip.core.opcua.client.IntReceivedListener}}}
\subsection{Methods}{
\rule[1em]{\hsize}{2pt}\vskip -2em
\vskip -2em
\begin{itemize}
\item{ 
\index{onError()}
{\bf  onError}\\
\begin{lstlisting}[frame=none]
void onError()\end{lstlisting} %end signature
\begin{itemize}
\item{
{\bf  Description}

An error occured while trying to fetch data from the server
}
\end{itemize}
}%end item
\end{itemize}
}
}
\section{\label{edu.kit.pse.osip.core.opcua.client.UAClientWrapper}\index{UAClientWrapper}Class UAClientWrapper}{
\rule[1em]{\hsize}{4pt}\vskip -1em
\vskip .1in 
Simplifies the interface of Milo. Provides a way to easily read values by using their identifier. Allows to add subscriptions without having to deal with milo internals. Directly converts values to the 3 major types.\vskip .1in 
\subsection{Declaration}{
\begin{lstlisting}[frame=none]
public abstract class UAClientWrapper
 extends java.lang.Object\end{lstlisting}
\subsection{All known subclasses}{AbstractTankClient\small{\refdefined{edu.kit.pse.osip.monitoring.controller.AbstractTankClient}}, TankClient\small{\refdefined{edu.kit.pse.osip.monitoring.controller.TankClient}}, MixTankClient\small{\refdefined{edu.kit.pse.osip.monitoring.controller.MixTankClient}}}
\subsection{Fields}{
\rule[1em]{\hsize}{2pt}
\begin{itemize}
\item{
\index{client}
\label{edu.kit.pse.osip.core.opcua.client.UAClientWrapper.client}\texttt{public org.eclipse.milo.opcua.sdk.client.OpcUaClient\ {\bf  client}}
}
\end{itemize}
}
\subsection{Constructors}{
\rule[1em]{\hsize}{2pt}\vskip -2em
\vskip -2em
\begin{itemize}
\item{ 
\index{UAClientWrapper(String, String)}
{\bf  UAClientWrapper}\\
\begin{lstlisting}[frame=none]
public UAClientWrapper(java.lang.String serverUrl,java.lang.String namespace)\end{lstlisting} %end signature
\begin{itemize}
\item{
{\bf  Description}

Wraps the milo client implementation to simplify process
}
\item{
{\bf  Parameters}
  \begin{itemize}
   \item{
\texttt{serverUrl} -- The url of the server}
   \item{
\texttt{namespace} -- Name of the expected default namespace. Will fail if the namespaces do not match}
  \end{itemize}
}%end item
\end{itemize}
}%end item
\end{itemize}
}
\subsection{Methods}{
\rule[1em]{\hsize}{2pt}\vskip -2em
\vskip -2em
\begin{itemize}
\item{ 
\index{connectClient()}
{\bf  connectClient}\\
\begin{lstlisting}[frame=none]
public final void connectClient()\end{lstlisting} %end signature
\begin{itemize}
\item{
{\bf  Description}

Connects the client to the server
}
\end{itemize}
}%end item
\divideents{disconnectClient}
\item{ 
\index{disconnectClient()}
{\bf  disconnectClient}\\
\begin{lstlisting}[frame=none]
public final void disconnectClient()\end{lstlisting} %end signature
\begin{itemize}
\item{
{\bf  Description}

Disconnects the client from the server
}
\end{itemize}
}%end item
\divideents{subscribeBoolean}
\item{ 
\index{subscribeBoolean(String, int, BooleanReceivedListener)}
{\bf  subscribeBoolean}\\
\begin{lstlisting}[frame=none]
protected final void subscribeBoolean(java.lang.String nodeName,int interval,BooleanReceivedListener listener)\end{lstlisting} %end signature
\begin{itemize}
\item{
{\bf  Description}

Subscribes to a boolean from the server. Subscribe again with same listener and other interval to change interval.
}
\item{
{\bf  Parameters}
  \begin{itemize}
   \item{
\texttt{nodeName} -- The path of the node to subscribe}
   \item{
\texttt{interval} -- Fetch interval in ms}
   \item{
\texttt{listener} -- Callback function that is called when the value was changed}
  \end{itemize}
}%end item
\end{itemize}
}%end item
\divideents{subscribeFloat}
\item{ 
\index{subscribeFloat(String, int, FloatReceivedListener)}
{\bf  subscribeFloat}\\
\begin{lstlisting}[frame=none]
protected final void subscribeFloat(java.lang.String nodeName,int interval,FloatReceivedListener listener)\end{lstlisting} %end signature
\begin{itemize}
\item{
{\bf  Description}

Subscribes to a float from the server. Subscribe again with same listener and other interval to change interval.
}
\item{
{\bf  Parameters}
  \begin{itemize}
   \item{
\texttt{nodeName} -- The path of the node to subscribe}
   \item{
\texttt{interval} -- Fetch interval in ms}
   \item{
\texttt{listener} -- Callback function that is called when the value was changed}
  \end{itemize}
}%end item
\end{itemize}
}%end item
\divideents{subscribeInt}
\item{ 
\index{subscribeInt(String, int, IntReceivedListener)}
{\bf  subscribeInt}\\
\begin{lstlisting}[frame=none]
protected final void subscribeInt(java.lang.String nodeName,int interval,IntReceivedListener listener)\end{lstlisting} %end signature
\begin{itemize}
\item{
{\bf  Description}

Subscribes to an int from the server. Subscribe again with same listener and other interval to change interval.
}
\item{
{\bf  Parameters}
  \begin{itemize}
   \item{
\texttt{nodeName} -- The path of the node to subscribe}
   \item{
\texttt{interval} -- Fetch interval in ms}
   \item{
\texttt{listener} -- Callback function that is called when the value was changed}
  \end{itemize}
}%end item
\end{itemize}
}%end item
\divideents{unsubscribe}
\item{ 
\index{unsubscribe(ReceivedListener)}
{\bf  unsubscribe}\\
\begin{lstlisting}[frame=none]
public final void unsubscribe(ReceivedListener listener)\end{lstlisting} %end signature
\begin{itemize}
\item{
{\bf  Description}

Unsubscribes the listener from getting refreshed
}
\item{
{\bf  Parameters}
  \begin{itemize}
   \item{
\texttt{listener} -- The listener of the subscription to cancel}
  \end{itemize}
}%end item
\end{itemize}
}%end item
\end{itemize}
}
}
}
\chapter{Package edu.kit.pse.osip.core.opcua.server}{
\label{edu.kit.pse.osip.core.opcua.server}\section{\label{edu.kit.pse.osip.core.opcua.server.UANamespaceWrapper}\index{UANamespaceWrapper}Class UANamespaceWrapper}{
\rule[1em]{\hsize}{4pt}\vskip -1em
\vskip .1in 
Implements a namespace needed by Milo. Contains methods to add folders and variables from outside to make it easier to create a server (by using inheritance). Methods are called by UAServerWrapper on its default namespace.\vskip .1in 
\subsection{Declaration}{
\begin{lstlisting}[frame=none]
public class UANamespaceWrapper
 extends java.lang.Object\end{lstlisting}
\subsection{Constructors}{
\rule[1em]{\hsize}{2pt}\vskip -2em
\vskip -2em
\begin{itemize}
\item{ 
\index{UANamespaceWrapper(OpcUaServer, UShort)}
{\bf  UANamespaceWrapper}\\
\begin{lstlisting}[frame=none]
protected UANamespaceWrapper(OpcUaServer server,UShort namespaceIndex)\end{lstlisting} %end signature
\begin{itemize}
\item{
{\bf  Description}

Creates a new namespace to simply manage a milo namespace
}
\item{
{\bf  Parameters}
  \begin{itemize}
   \item{
\texttt{server} -- The server that this namespace belongs to}
   \item{
\texttt{namespaceIndex} -- The index of this namespace}
  \end{itemize}
}%end item
\end{itemize}
}%end item
\end{itemize}
}
\subsection{Methods}{
\rule[1em]{\hsize}{2pt}\vskip -2em
\vskip -2em
\begin{itemize}
\item{ 
\index{addFolder(String, String)}
{\bf  addFolder}\\
\begin{lstlisting}[frame=none]
protected final void addFolder(java.lang.String path,java.lang.String displayName)\end{lstlisting} %end signature
\begin{itemize}
\item{
{\bf  Description}

Add a folder to the server
}
\item{
{\bf  Parameters}
  \begin{itemize}
   \item{
\texttt{path} -- The path of the folder to add}
   \item{
\texttt{displayName} -- Name that is displayed to the user}
  \end{itemize}
}%end item
\end{itemize}
}%end item
\divideents{addVariable}
\item{ 
\index{addVariable(String, String)}
{\bf  addVariable}\\
\begin{lstlisting}[frame=none]
protected final void addVariable(java.lang.String path,java.lang.String displayName)\end{lstlisting} %end signature
\begin{itemize}
\item{
{\bf  Description}

Add a variable to the server
}
\item{
{\bf  Parameters}
  \begin{itemize}
   \item{
\texttt{path} -- The path of the variable to add}
   \item{
\texttt{displayName} -- Name that is displayed to the user}
  \end{itemize}
}%end item
\end{itemize}
}%end item
\divideents{browse}
\item{ 
\index{browse(AccessContext, NodeId)}
{\bf  browse}\\
\begin{lstlisting}[frame=none]
public final java.util.concurrent.CompletableFuture browse(AccessContext context,NodeId nodeId)\end{lstlisting} %end signature
\begin{itemize}
\item{
{\bf  Description}

Needed by milo. Allows browsing the nodes inside this namespace
}
\item{
{\bf  Parameters}
  \begin{itemize}
   \item{
\texttt{context} -- The context to write the values back}
   \item{
\texttt{nodeId} -- The id of the element to bowse}
  \end{itemize}
}%end item
\item{{\bf  Returns} -- 
a list of references tto nodes on the server 
}%end item
\end{itemize}
}%end item
\divideents{getNamespaceIndex}
\item{ 
\index{getNamespaceIndex()}
{\bf  getNamespaceIndex}\\
\begin{lstlisting}[frame=none]
public final UShort getNamespaceIndex()\end{lstlisting} %end signature
\begin{itemize}
\item{
{\bf  Description}

Needed by milo: Returns the namespace index of this namespace
}
\item{{\bf  Returns} -- 
The index of this namespace 
}%end item
\end{itemize}
}%end item
\divideents{getNamespaceUri}
\item{ 
\index{getNamespaceUri()}
{\bf  getNamespaceUri}\\
\begin{lstlisting}[frame=none]
public final java.lang.String getNamespaceUri()\end{lstlisting} %end signature
\begin{itemize}
\item{
{\bf  Description}

Returns the identifier for this namespace
}
\item{{\bf  Returns} -- 
the identifying string 
}%end item
\end{itemize}
}%end item
\divideents{onDataItemsCreated}
\item{ 
\index{onDataItemsCreated(List)}
{\bf  onDataItemsCreated}\\
\begin{lstlisting}[frame=none]
public final void onDataItemsCreated(java.util.List dataItems)\end{lstlisting} %end signature
\begin{itemize}
\item{
{\bf  Description}

Needed by milo. Gets called when the data inside the server are created
}
\item{
{\bf  Parameters}
  \begin{itemize}
   \item{
\texttt{dataItems} -- The created items}
  \end{itemize}
}%end item
\end{itemize}
}%end item
\divideents{onDataItemsDeleted}
\item{ 
\index{onDataItemsDeleted(List)}
{\bf  onDataItemsDeleted}\\
\begin{lstlisting}[frame=none]
public final void onDataItemsDeleted(java.util.List dataItems)\end{lstlisting} %end signature
\begin{itemize}
\item{
{\bf  Description}

Needed by milo. Called when items inside the server are deleted
}
\item{
{\bf  Parameters}
  \begin{itemize}
   \item{
\texttt{dataItems} -- The deleted items}
  \end{itemize}
}%end item
\end{itemize}
}%end item
\divideents{onDataItemsModified}
\item{ 
\index{onDataItemsModified(List)}
{\bf  onDataItemsModified}\\
\begin{lstlisting}[frame=none]
public final void onDataItemsModified(java.util.List dataItems)\end{lstlisting} %end signature
\begin{itemize}
\item{
{\bf  Description}

Needed by milo. Called when the data inside the server is modified
}
\item{
{\bf  Parameters}
  \begin{itemize}
   \item{
\texttt{dataItems} -- The modified data items}
  \end{itemize}
}%end item
\end{itemize}
}%end item
\divideents{onMonitoringModeChanged}
\item{ 
\index{onMonitoringModeChanged(List)}
{\bf  onMonitoringModeChanged}\\
\begin{lstlisting}[frame=none]
public final void onMonitoringModeChanged(java.util.List monitoredItems)\end{lstlisting} %end signature
\begin{itemize}
\item{
{\bf  Description}

Needed by milo. Called when the method changes how the client gets its data
}
\item{
{\bf  Parameters}
  \begin{itemize}
   \item{
\texttt{monitoredItems} -- The changed items}
  \end{itemize}
}%end item
\end{itemize}
}%end item
\divideents{read}
\item{ 
\index{read(ReadContext, Double, TimestampsToReturn, List)}
{\bf  read}\\
\begin{lstlisting}[frame=none]
public final void read(ReadContext context,java.lang.Double maxAge,TimestampsToReturn timestamps,java.util.List readValueIds)\end{lstlisting} %end signature
\begin{itemize}
\item{
{\bf  Description}

Needed by milo. Allows reading a value
}
\item{
{\bf  Parameters}
  \begin{itemize}
   \item{
\texttt{context} -- The context to write the value back}
   \item{
\texttt{maxAge} -- Maximum age of the values to return}
   \item{
\texttt{timestamps} -- The value from which time to return}
   \item{
\texttt{readValueIds} -- The ids of the values that should be read}
  \end{itemize}
}%end item
\end{itemize}
}%end item
\divideents{updateValue}
\item{ 
\index{updateValue(String, DataValue)}
{\bf  updateValue}\\
\begin{lstlisting}[frame=none]
protected final void updateValue(java.lang.String path,DataValue value)\end{lstlisting} %end signature
\begin{itemize}
\item{
{\bf  Description}

Update the value of a variable
}
\item{
{\bf  Parameters}
  \begin{itemize}
   \item{
\texttt{path} -- The path of the variable to update}
   \item{
\texttt{value} -- Name that is displayed to the user}
  \end{itemize}
}%end item
\end{itemize}
}%end item
\divideents{write}
\item{ 
\index{write(WriteContext, List)}
{\bf  write}\\
\begin{lstlisting}[frame=none]
public final void write(WriteContext context,java.util.List values)\end{lstlisting} %end signature
\begin{itemize}
\item{
{\bf  Description}

Needed by milo. Writes values to the current server
}
\item{
{\bf  Parameters}
  \begin{itemize}
   \item{
\texttt{context} -- The calling context}
   \item{
\texttt{values} -- The values to write}
  \end{itemize}
}%end item
\end{itemize}
}%end item
\end{itemize}
}
}
\section{\label{edu.kit.pse.osip.core.opcua.server.UAServerWrapper}\index{UAServerWrapper}Class UAServerWrapper}{
\rule[1em]{\hsize}{4pt}\vskip -1em
\vskip .1in 
Simplifies the interface of Milo. Automatically adds a namespace and provides methods to directly manage the namespace, because multiple namespaces are not needed in our case. Allows to access variables based on their path instead of using the NodeId.\vskip .1in 
\subsection{Declaration}{
\begin{lstlisting}[frame=none]
public abstract class UAServerWrapper
 extends java.lang.Object\end{lstlisting}
\subsection{All known subclasses}{AbstractTankServer\small{\refdefined{edu.kit.pse.osip.simulation.controller.AbstractTankServer}}, TankServer\small{\refdefined{edu.kit.pse.osip.simulation.controller.TankServer}}, MixTankServer\small{\refdefined{edu.kit.pse.osip.simulation.controller.MixTankServer}}}
\subsection{Fields}{
\rule[1em]{\hsize}{2pt}
\begin{itemize}
\item{
\index{namespace}
\label{edu.kit.pse.osip.core.opcua.server.UAServerWrapper.namespace}\texttt{public UANamespaceWrapper\ {\bf  namespace}}
}
\end{itemize}
}
\subsection{Constructors}{
\rule[1em]{\hsize}{2pt}\vskip -2em
\vskip -2em
\begin{itemize}
\item{ 
\index{UAServerWrapper(String, int)}
{\bf  UAServerWrapper}\\
\begin{lstlisting}[frame=none]
public UAServerWrapper(java.lang.String namespaceName,int port)\end{lstlisting} %end signature
\begin{itemize}
\item{
{\bf  Description}

Wraps a milo server to simplify handling
}
\item{
{\bf  Parameters}
  \begin{itemize}
   \item{
\texttt{namespaceName} -- The name of the namespace that is automatically generated}
   \item{
\texttt{port} -- The port to listen to}
  \end{itemize}
}%end item
\end{itemize}
}%end item
\end{itemize}
}
\subsection{Methods}{
\rule[1em]{\hsize}{2pt}\vskip -2em
\vskip -2em
\begin{itemize}
\item{ 
\index{addFolder(String, String)}
{\bf  addFolder}\\
\begin{lstlisting}[frame=none]
protected final void addFolder(java.lang.String path,java.lang.String displayName)\end{lstlisting} %end signature
\begin{itemize}
\item{
{\bf  Description}

Adds a folder to the default namespace
}
\item{
{\bf  Parameters}
  \begin{itemize}
   \item{
\texttt{path} -- Slash seperated path of the folder}
   \item{
\texttt{displayName} -- Name of the folder that is displayed to users}
  \end{itemize}
}%end item
\end{itemize}
}%end item
\divideents{addVariable}
\item{ 
\index{addVariable(String, String)}
{\bf  addVariable}\\
\begin{lstlisting}[frame=none]
protected final void addVariable(java.lang.String path,java.lang.String displayName)\end{lstlisting} %end signature
\begin{itemize}
\item{
{\bf  Description}

Adds a variable to the default namespace
}
\item{
{\bf  Parameters}
  \begin{itemize}
   \item{
\texttt{path} -- Slash seperated path of the folder}
   \item{
\texttt{displayName} -- Name of the variable that is displayed to users}
  \end{itemize}
}%end item
\end{itemize}
}%end item
\divideents{setVariable}
\item{ 
\index{setVariable(String, DataValue)}
{\bf  setVariable}\\
\begin{lstlisting}[frame=none]
protected final void setVariable(java.lang.String path,DataValue value)\end{lstlisting} %end signature
\begin{itemize}
\item{
{\bf  Description}

Sets the value of a variable in the default namespace
}
\item{
{\bf  Parameters}
  \begin{itemize}
   \item{
\texttt{path} -- Slash seperated path of the folder}
   \item{
\texttt{value} -- Value of the variable}
  \end{itemize}
}%end item
\end{itemize}
}%end item
\divideents{start}
\item{ 
\index{start()}
{\bf  start}\\
\begin{lstlisting}[frame=none]
public final void start()\end{lstlisting} %end signature
\begin{itemize}
\item{
{\bf  Description}

Starts the server
}
\item{{\bf  Throws}
  \begin{itemize}
   \item{\vskip -.6ex \texttt{InterruptedException,} -- ExecutionException, ConnectException If Milo has problems connecting to the remote machine}
  \end{itemize}
}%end item
\end{itemize}
}%end item
\divideents{stop}
\item{ 
\index{stop()}
{\bf  stop}\\
\begin{lstlisting}[frame=none]
public final void stop()\end{lstlisting} %end signature
\begin{itemize}
\item{
{\bf  Description}

Stops the server. Can not be restarted afterwards.
}
\end{itemize}
}%end item
\end{itemize}
}
}
}
\chapter{Package edu.kit.pse.osip.core.utils.language}{
\label{edu.kit.pse.osip.core.utils.language}\section{\label{edu.kit.pse.osip.core.utils.language.Translator}\index{Translator}Class Translator}{
\rule[1em]{\hsize}{4pt}\vskip -1em
\vskip .1in 
This class is the interface to load menu texts in different languages.\vskip .1in 
\subsection{Declaration}{
\begin{lstlisting}[frame=none]
public class Translator
 extends java.lang.Object\end{lstlisting}
\subsection{Fields}{
\rule[1em]{\hsize}{2pt}
\begin{itemize}
\item{
\index{bundle}
\label{edu.kit.pse.osip.core.utils.language.Translator.bundle}\texttt{public java.util.ResourceBundle\ {\bf  bundle}}
}
\item{
\index{locale}
\label{edu.kit.pse.osip.core.utils.language.Translator.locale}\texttt{public java.util.Locale\ {\bf  locale}}
}
\end{itemize}
}
\subsection{Methods}{
\rule[1em]{\hsize}{2pt}\vskip -2em
\vskip -2em
\begin{itemize}
\item{ 
\index{getInstance()}
{\bf  getInstance}\\
\begin{lstlisting}[frame=none]
public static final Translator getInstance()\end{lstlisting} %end signature
\begin{itemize}
\item{
{\bf  Description}

Gets the single Instance of the Translator. It is newly created at the first call of the method.
}
\item{{\bf  Returns} -- 
The single Instance of Translator. 
}%end item
\end{itemize}
}%end item
\divideents{getString}
\item{ 
\index{getString(String)}
{\bf  getString}\\
\begin{lstlisting}[frame=none]
public final java.lang.String getString(java.lang.String key)\end{lstlisting} %end signature
\begin{itemize}
\item{
{\bf  Description}

Gets the word that is associated with key in the current locale.
}
\item{
{\bf  Parameters}
  \begin{itemize}
   \item{
\texttt{key} -- The key to use for translation lookup}
  \end{itemize}
}%end item
\item{{\bf  Returns} -- 
The translation for key. 
}%end item
\end{itemize}
}%end item
\divideents{setLocale}
\item{ 
\index{setLocale(Locale)}
{\bf  setLocale}\\
\begin{lstlisting}[frame=none]
public final void setLocale(java.util.Locale locale)\end{lstlisting} %end signature
\begin{itemize}
\item{
{\bf  Description}

Sets the locale to be used when translating
}
\item{
{\bf  Parameters}
  \begin{itemize}
   \item{
\texttt{locale} -- The locale to set}
  \end{itemize}
}%end item
\end{itemize}
}%end item
\end{itemize}
}
}
}
\chapter{Package edu.kit.pse.osip.core.utils.formatting}{
\label{edu.kit.pse.osip.core.utils.formatting}\section{\label{edu.kit.pse.osip.core.utils.formatting.FormatChecker}\index{FormatChecker}Class FormatChecker}{
\rule[1em]{\hsize}{4pt}\vskip -1em
\vskip .1in 
This class provides static methods to check input formats.\vskip .1in 
\subsection{Declaration}{
\begin{lstlisting}[frame=none]
public class FormatChecker
 extends java.lang.Object\end{lstlisting}
\subsection{Constructors}{
\rule[1em]{\hsize}{2pt}\vskip -2em
\vskip -2em
\begin{itemize}
\item{ 
\index{FormatChecker()}
{\bf  FormatChecker}\\
\begin{lstlisting}[frame=none]
public FormatChecker()\end{lstlisting} %end signature
}%end item
\end{itemize}
}
\subsection{Methods}{
\rule[1em]{\hsize}{2pt}\vskip -2em
\vskip -2em
\begin{itemize}
\item{ 
\index{checkHost(String)}
{\bf  checkHost}\\
\begin{lstlisting}[frame=none]
public static final void checkHost(java.lang.String host)\end{lstlisting} %end signature
\begin{itemize}
\item{
{\bf  Description}

Checks whether string is a valid IP address or hostname.
}
\item{
{\bf  Parameters}
  \begin{itemize}
   \item{
\texttt{host} -- }
  \end{itemize}
}%end item
\item{{\bf  Throws}
  \begin{itemize}
   \item{\vskip -.6ex \texttt{edu.kit.pse.osip.core.utils.formatting.InvalidHostException} -- Thrown if the given host is not valid}
  \end{itemize}
}%end item
\end{itemize}
}%end item
\divideents{checkPercentage}
\item{ 
\index{checkPercentage(String)}
{\bf  checkPercentage}\\
\begin{lstlisting}[frame=none]
public static final int checkPercentage(java.lang.String percentage)\end{lstlisting} %end signature
\begin{itemize}
\item{
{\bf  Description}

Parses a percentage value and checks whether is valid.
}
\item{
{\bf  Parameters}
  \begin{itemize}
   \item{
\texttt{percentage} -- The percentage to parse}
  \end{itemize}
}%end item
\item{{\bf  Returns} -- 
percentage parsed into an int. 
}%end item
\item{{\bf  Throws}
  \begin{itemize}
   \item{\vskip -.6ex \texttt{edu.kit.pse.osip.core.utils.formatting.InvalidPercentageException} -- Thrown, if percentage does not represent an int or if percentage is not in the range from 0 to 100.}
  \end{itemize}
}%end item
\end{itemize}
}%end item
\divideents{checkPort}
\item{ 
\index{checkPort(String)}
{\bf  checkPort}\\
\begin{lstlisting}[frame=none]
public static final int checkPort(java.lang.String port)\end{lstlisting} %end signature
\begin{itemize}
\item{
{\bf  Description}

Parses port into an int.
}
\item{
{\bf  Parameters}
  \begin{itemize}
   \item{
\texttt{port} -- The port to parse}
  \end{itemize}
}%end item
\item{{\bf  Returns} -- 
The port number 
}%end item
\item{{\bf  Throws}
  \begin{itemize}
   \item{\vskip -.6ex \texttt{edu.kit.pse.osip.core.utils.formatting.InvalidPortException} -- Thrown, if port is not either an int between 1024 and 61000 or the string is empty, contains characters etc.}
  \end{itemize}
}%end item
\end{itemize}
}%end item
\end{itemize}
}
}
\section{\label{edu.kit.pse.osip.core.utils.formatting.InvalidHostException}\index{InvalidHostException}Exception InvalidHostException}{
\rule[1em]{\hsize}{4pt}\vskip -1em
\vskip .1in 
This exception signifies that an IP or a hostname is invalid.\vskip .1in 
\subsection{Declaration}{
\begin{lstlisting}[frame=none]
public class InvalidHostException
 extends java.lang.IllegalArgumentException\end{lstlisting}
\subsection{Constructors}{
\rule[1em]{\hsize}{2pt}\vskip -2em
\vskip -2em
\begin{itemize}
\item{ 
\index{InvalidHostException(String, String)}
{\bf  InvalidHostException}\\
\begin{lstlisting}[frame=none]
public InvalidHostException(java.lang.String tried,java.lang.String reason)\end{lstlisting} %end signature
\begin{itemize}
\item{
{\bf  Description}

Creates a new InvalidHostException
}
\item{
{\bf  Parameters}
  \begin{itemize}
   \item{
\texttt{tried} -- The value that was tried to parse}
   \item{
\texttt{reason} -- The reason for the failture}
  \end{itemize}
}%end item
\end{itemize}
}%end item
\end{itemize}
}
}
\section{\label{edu.kit.pse.osip.core.utils.formatting.InvalidPercentageException}\index{InvalidPercentageException}Exception InvalidPercentageException}{
\rule[1em]{\hsize}{4pt}\vskip -1em
\vskip .1in 
This exception signifies that a percentage is invalid.\vskip .1in 
\subsection{Declaration}{
\begin{lstlisting}[frame=none]
public class InvalidPercentageException
 extends java.lang.IllegalArgumentException\end{lstlisting}
\subsection{Constructors}{
\rule[1em]{\hsize}{2pt}\vskip -2em
\vskip -2em
\begin{itemize}
\item{ 
\index{InvalidPercentageException(String, String)}
{\bf  InvalidPercentageException}\\
\begin{lstlisting}[frame=none]
public InvalidPercentageException(java.lang.String tried,java.lang.String reason)\end{lstlisting} %end signature
\begin{itemize}
\item{
{\bf  Description}

Creates a new InvalidPercentageException
}
\item{
{\bf  Parameters}
  \begin{itemize}
   \item{
\texttt{tried} -- The string that was tried to be parsed}
   \item{
\texttt{reason} -- Explains why the check failed}
  \end{itemize}
}%end item
\end{itemize}
}%end item
\end{itemize}
}
}
\section{\label{edu.kit.pse.osip.core.utils.formatting.InvalidPortException}\index{InvalidPortException}Exception InvalidPortException}{
\rule[1em]{\hsize}{4pt}\vskip -1em
\vskip .1in 
This exception signifies that a port was invalid.\vskip .1in 
\subsection{Declaration}{
\begin{lstlisting}[frame=none]
public class InvalidPortException
 extends java.lang.IllegalArgumentException\end{lstlisting}
\subsection{Constructors}{
\rule[1em]{\hsize}{2pt}\vskip -2em
\vskip -2em
\begin{itemize}
\item{ 
\index{InvalidPortException(String, String)}
{\bf  InvalidPortException}\\
\begin{lstlisting}[frame=none]
public InvalidPortException(java.lang.String tried,java.lang.String reason)\end{lstlisting} %end signature
\begin{itemize}
\item{
{\bf  Description}

Creates a new InvalidPortException
}
\item{
{\bf  Parameters}
  \begin{itemize}
   \item{
\texttt{tried} -- The string that was tried}
   \item{
\texttt{reason} -- The reason why parsing went wrong}
  \end{itemize}
}%end item
\end{itemize}
}%end item
\end{itemize}
}
}
}
\chapter{Package edu.kit.pse.osip.core.model.behavior}{
\label{edu.kit.pse.osip.core.model.behavior}\section{\label{edu.kit.pse.osip.core.model.behavior.AlarmBehavior}\index{AlarmBehavior}Class AlarmBehavior}{
\rule[1em]{\hsize}{4pt}\vskip -1em
\vskip .1in 
This specifies whether an alarm is triggered if the value is below or greather than the threshold.\vskip .1in 
\subsection{Declaration}{
\begin{lstlisting}[frame=none]
public final class AlarmBehavior
 extends java.lang.Enum\end{lstlisting}
\subsection{Fields}{
\rule[1em]{\hsize}{2pt}
\begin{itemize}
\item{
\index{GREATHER\_THAN}
\label{edu.kit.pse.osip.core.model.behavior.AlarmBehavior.GREATHER_THAN}\texttt{public static final AlarmBehavior\ {\bf  GREATHER\_THAN}}
\begin{itemize}
\item{\vskip -.9ex 
Alarm is triggered when the observed value gets greater than the given value.}
\end{itemize}
}
\item{
\index{SMALLER\_THAN}
\label{edu.kit.pse.osip.core.model.behavior.AlarmBehavior.SMALLER_THAN}\texttt{public static final AlarmBehavior\ {\bf  SMALLER\_THAN}}
\begin{itemize}
\item{\vskip -.9ex 
Alarm is triggered when the observed value gets smaller than the given value.}
\end{itemize}
}
\end{itemize}
}
\subsection{Methods}{
\rule[1em]{\hsize}{2pt}\vskip -2em
\vskip -2em
\begin{itemize}
\item{ 
\index{valueOf(String)}
{\bf  valueOf}\\
\begin{lstlisting}[frame=none]
public static AlarmBehavior valueOf(java.lang.String name)\end{lstlisting} %end signature
}%end item
\divideents{values}
\item{ 
\index{values()}
{\bf  values}\\
\begin{lstlisting}[frame=none]
public static AlarmBehavior[] values()\end{lstlisting} %end signature
}%end item
\end{itemize}
}
}
\section{\label{edu.kit.pse.osip.core.model.behavior.FillAlarm}\index{FillAlarm}Class FillAlarm}{
\rule[1em]{\hsize}{4pt}\vskip -1em
\vskip .1in 
An alarm which monitors whether the fill level breaks a given threshold.\vskip .1in 
\subsection{Declaration}{
\begin{lstlisting}[frame=none]
public class FillAlarm
 extends edu.kit.pse.osip.core.model.behavior.TankAlarm\end{lstlisting}
\subsection{Constructors}{
\rule[1em]{\hsize}{2pt}\vskip -2em
\vskip -2em
\begin{itemize}
\item{ 
\index{FillAlarm(Tank, Float, AlarmBehavior)}
{\bf  FillAlarm}\\
\begin{lstlisting}[frame=none]
public FillAlarm(edu.kit.pse.osip.core.model.base.Tank tank,java.lang.Float threshold,AlarmBehavior behavior)\end{lstlisting} %end signature
\begin{itemize}
\item{
{\bf  Description}

Constructs a new FillAlarm.
}
\item{
{\bf  Parameters}
  \begin{itemize}
   \item{
\texttt{tank} -- The tank to monitor.}
   \item{
\texttt{threshold} -- The fill level threshold in \%.}
   \item{
\texttt{behavior} -- Whether the alarm should trigger if thefill level is above or below the threshold.}
  \end{itemize}
}%end item
\end{itemize}
}%end item
\end{itemize}
}
\subsection{Methods}{
\rule[1em]{\hsize}{2pt}\vskip -2em
\vskip -2em
\begin{itemize}
\item{ 
\index{getNotifiedValue()}
{\bf  getNotifiedValue}\\
\begin{lstlisting}[frame=none]
protected final java.lang.Float getNotifiedValue()\end{lstlisting} %end signature
\begin{itemize}
\item{
{\bf  Description}

Returns the fill level.
}
\item{{\bf  Returns} -- 
the fill level. 
}%end item
\end{itemize}
}%end item
\end{itemize}
}
}
\section{\label{edu.kit.pse.osip.core.model.behavior.Scenario}\index{Scenario}Class Scenario}{
\rule[1em]{\hsize}{4pt}\vskip -1em
\vskip .1in 
A sequence of automated changes to parameters of the simulation. It is used to have changes in the simulation (so it looks more interesting) without the need to adjust parameters all the time.\vskip .1in 
\subsection{Declaration}{
\begin{lstlisting}[frame=none]
public class Scenario
 extends java.util.Observable\end{lstlisting}
\subsection{Constructors}{
\rule[1em]{\hsize}{2pt}\vskip -2em
\vskip -2em
\begin{itemize}
\item{ 
\index{Scenario()}
{\bf  Scenario}\\
\begin{lstlisting}[frame=none]
public Scenario()\end{lstlisting} %end signature
}%end item
\end{itemize}
}
\subsection{Methods}{
\rule[1em]{\hsize}{2pt}\vskip -2em
\vskip -2em
\begin{itemize}
\item{ 
\index{addPause(int)}
{\bf  addPause}\\
\begin{lstlisting}[frame=none]
public final void addPause(int length)\end{lstlisting} %end signature
\begin{itemize}
\item{
{\bf  Description}

Add a pause after the last command or last pause.
}
\item{
{\bf  Parameters}
  \begin{itemize}
   \item{
\texttt{length} -- The length of the pause in ms.}
  \end{itemize}
}%end item
\end{itemize}
}%end item
\divideents{appendRunnable}
\item{ 
\index{appendRunnable(Consumer)}
{\bf  appendRunnable}\\
\begin{lstlisting}[frame=none]
public final void appendRunnable(java.util.function.Consumer runnable)\end{lstlisting} %end signature
\begin{itemize}
\item{
{\bf  Description}

Add a command to the Scenario. It gets executed after the last command or pause.
}
\item{
{\bf  Parameters}
  \begin{itemize}
   \item{
\texttt{runnable} -- The command to run.}
  \end{itemize}
}%end item
\end{itemize}
}%end item
\divideents{cancelScenario}
\item{ 
\index{cancelScenario()}
{\bf  cancelScenario}\\
\begin{lstlisting}[frame=none]
public final void cancelScenario()\end{lstlisting} %end signature
\begin{itemize}
\item{
{\bf  Description}

Stop the Scenario.
}
\end{itemize}
}%end item
\divideents{isRunning}
\item{ 
\index{isRunning()}
{\bf  isRunning}\\
\begin{lstlisting}[frame=none]
public final boolean isRunning()\end{lstlisting} %end signature
\begin{itemize}
\item{
{\bf  Description}

Report whether the Scenario is currently running.
}
\item{{\bf  Returns} -- 
true if the Scenario is running, false otherwise. 
}%end item
\end{itemize}
}%end item
\divideents{setProductionSite}
\item{ 
\index{setProductionSite(ProductionSite)}
{\bf  setProductionSite}\\
\begin{lstlisting}[frame=none]
public final void setProductionSite(edu.kit.pse.osip.core.model.base.ProductionSite productionSite)\end{lstlisting} %end signature
\begin{itemize}
\item{
{\bf  Description}

Set a production site which is needed to execute the runnables. This needs to be done before starting the scenario.
}
\item{
{\bf  Parameters}
  \begin{itemize}
   \item{
\texttt{productionSite} -- The production site which gets set.}
  \end{itemize}
}%end item
\end{itemize}
}%end item
\divideents{startScenario}
\item{ 
\index{startScenario()}
{\bf  startScenario}\\
\begin{lstlisting}[frame=none]
public final void startScenario()\end{lstlisting} %end signature
\begin{itemize}
\item{
{\bf  Description}

Start the Scenario.
}
\item{{\bf  Throws}
  \begin{itemize}
   \item{\vskip -.6ex \texttt{java.lang.IllegalStateException} -- if the production site isn't set (@see setProductionSite).}
  \end{itemize}
}%end item
\end{itemize}
}%end item
\end{itemize}
}
}
\section{\label{edu.kit.pse.osip.core.model.behavior.TankAlarm}\index{TankAlarm}Class TankAlarm}{
\rule[1em]{\hsize}{4pt}\vskip -1em
\vskip .1in 
An abstract class which monitors some attribute of an AbstractTank.\vskip .1in 
\subsection{Declaration}{
\begin{lstlisting}[frame=none]
public abstract class TankAlarm
 extends java.util.Observable implements java.util.Observer\end{lstlisting}
\subsection{All known subclasses}{FillAlarm\small{\refdefined{edu.kit.pse.osip.core.model.behavior.FillAlarm}}, TemperatureAlarm\small{\refdefined{edu.kit.pse.osip.core.model.behavior.TemperatureAlarm}}}
\subsection{Fields}{
\rule[1em]{\hsize}{2pt}
\begin{itemize}
\item{
\index{tank}
\label{edu.kit.pse.osip.core.model.behavior.TankAlarm.tank}\texttt{public edu.kit.pse.osip.core.model.base.AbstractTank\ {\bf  tank}}
}
\end{itemize}
}
\subsection{Constructors}{
\rule[1em]{\hsize}{2pt}\vskip -2em
\vskip -2em
\begin{itemize}
\item{ 
\index{TankAlarm(AbstractTank, T, AlarmBehavior)}
{\bf  TankAlarm}\\
\begin{lstlisting}[frame=none]
public TankAlarm(edu.kit.pse.osip.core.model.base.AbstractTank tank,java.lang.Object threshold,AlarmBehavior behavior)\end{lstlisting} %end signature
\begin{itemize}
\item{
{\bf  Description}

Creates a new TankAlarm.
}
\item{
{\bf  Parameters}
  \begin{itemize}
   \item{
\texttt{tank} -- The tank which has the property to monitor.}
   \item{
\texttt{threshold} -- The threshold of the value.}
   \item{
\texttt{behavior} -- Whether the alarm should trigger if the value is bigger or smaller than the threshold.}
  \end{itemize}
}%end item
\end{itemize}
}%end item
\end{itemize}
}
\subsection{Methods}{
\rule[1em]{\hsize}{2pt}\vskip -2em
\vskip -2em
\begin{itemize}
\item{ 
\index{getNotifiedValue()}
{\bf  getNotifiedValue}\\
\begin{lstlisting}[frame=none]
protected abstract java.lang.Object getNotifiedValue()\end{lstlisting} %end signature
\begin{itemize}
\item{
{\bf  Description}

Get the value of the tank, which is monitored by the alarm. This should be overriden by subclasses.
}
\end{itemize}
}%end item
\divideents{isAlarmTriggered}
\item{ 
\index{isAlarmTriggered()}
{\bf  isAlarmTriggered}\\
\begin{lstlisting}[frame=none]
public final boolean isAlarmTriggered()\end{lstlisting} %end signature
\begin{itemize}
\item{
{\bf  Description}

Return whether the alarm is active at the moment.
}
\item{{\bf  Returns} -- 
true if the alarm is currently triggered. 
}%end item
\end{itemize}
}%end item
\divideents{update}
\item{ 
\index{update(Observable, Object)}
{\bf  update}\\
\begin{lstlisting}[frame=none]
public final void update(java.util.Observable observable,java.lang.Object object)\end{lstlisting} %end signature
\begin{itemize}
\item{
{\bf  Description}

The method called by the tank if something changes.
}
\item{
{\bf  Parameters}
  \begin{itemize}
   \item{
\texttt{observable} -- the observerable object.}
   \item{
\texttt{object} -- The changed value.}
  \end{itemize}
}%end item
\end{itemize}
}%end item
\end{itemize}
}
}
\section{\label{edu.kit.pse.osip.core.model.behavior.TemperatureAlarm}\index{TemperatureAlarm}Class TemperatureAlarm}{
\rule[1em]{\hsize}{4pt}\vskip -1em
\vskip .1in 
An Alarm which monitors whether the temperature breaks a given threshold.\vskip .1in 
\subsection{Declaration}{
\begin{lstlisting}[frame=none]
public class TemperatureAlarm
 extends edu.kit.pse.osip.core.model.behavior.TankAlarm\end{lstlisting}
\subsection{Constructors}{
\rule[1em]{\hsize}{2pt}\vskip -2em
\vskip -2em
\begin{itemize}
\item{ 
\index{TemperatureAlarm(Tank, Float, AlarmBehavior)}
{\bf  TemperatureAlarm}\\
\begin{lstlisting}[frame=none]
public TemperatureAlarm(edu.kit.pse.osip.core.model.base.Tank tank,java.lang.Float threshold,AlarmBehavior behavior)\end{lstlisting} %end signature
\begin{itemize}
\item{
{\bf  Description}

Creates a new TemperatureAlarm.
}
\item{
{\bf  Parameters}
  \begin{itemize}
   \item{
\texttt{tank} -- The tank to monitor the temperature.}
   \item{
\texttt{threshold} -- The threshold in °K.}
   \item{
\texttt{behavior} -- Whether the alarm should trigger if the temperature is above or below to threshold.}
  \end{itemize}
}%end item
\end{itemize}
}%end item
\end{itemize}
}
\subsection{Methods}{
\rule[1em]{\hsize}{2pt}\vskip -2em
\vskip -2em
\begin{itemize}
\item{ 
\index{getNotifiedValue()}
{\bf  getNotifiedValue}\\
\begin{lstlisting}[frame=none]
protected final java.lang.Float getNotifiedValue()\end{lstlisting} %end signature
\begin{itemize}
\item{
{\bf  Description}

Returns the temperature of the tank
}
\item{{\bf  Returns} -- 
the temperature 
}%end item
\end{itemize}
}%end item
\end{itemize}
}
}
}
\chapter{Package edu.kit.pse.osip.core.model.simulation}{
\label{edu.kit.pse.osip.core.model.simulation}\section{\label{edu.kit.pse.osip.core.model.simulation.MixingStrategy}\index{MixingStrategy@\textit{ MixingStrategy}}Interface MixingStrategy}{
\rule[1em]{\hsize}{4pt}\vskip -1em
\vskip .1in 
A strategy to mix liquids. This includes the mixing of colors and the mixing of temperatures.\vskip .1in 
\subsection{Declaration}{
\begin{lstlisting}[frame=none]
public interface MixingStrategy
\end{lstlisting}
\subsection{All known subinterfaces}{SubtractiveMixingStrategy\small{\refdefined{edu.kit.pse.osip.core.model.simulation.SubtractiveMixingStrategy}}}
\subsection{All classes known to implement interface}{SubtractiveMixingStrategy\small{\refdefined{edu.kit.pse.osip.core.model.simulation.SubtractiveMixingStrategy}}}
\subsection{Methods}{
\rule[1em]{\hsize}{2pt}\vskip -2em
\vskip -2em
\begin{itemize}
\item{ 
\index{mixLiquids(LinkedList)}
{\bf  mixLiquids}\\
\begin{lstlisting}[frame=none]
edu.kit.pse.osip.core.model.base.Liquid mixLiquids(java.util.LinkedList inflow)\end{lstlisting} %end signature
\begin{itemize}
\item{
{\bf  Description}

Mixes the given liquids and generates a new one. Volume equals the sum of all others. Colors and temperatures get mixed pro rata.
}
\item{
{\bf  Parameters}
  \begin{itemize}
   \item{
\texttt{inflow} -- The different liquids to mix.}
  \end{itemize}
}%end item
\item{{\bf  Returns} -- 
a single Liquid element containing a mixture of the given liquids. 
}%end item
\end{itemize}
}%end item
\end{itemize}
}
}
\section{\label{edu.kit.pse.osip.core.model.simulation.MixTankSimulation}\index{MixTankSimulation}Class MixTankSimulation}{
\rule[1em]{\hsize}{4pt}\vskip -1em
\vskip .1in 
The model of a mixtank, allowing to put in and take out liquid. Besides this addition, it is a normal MixTank.\vskip .1in 
\subsection{Declaration}{
\begin{lstlisting}[frame=none]
public class MixTankSimulation
 extends edu.kit.pse.osip.core.model.base.MixTank\end{lstlisting}
\subsection{Fields}{
\rule[1em]{\hsize}{2pt}
\begin{itemize}
\item{
\index{mixingStrategy}
\label{edu.kit.pse.osip.core.model.simulation.MixTankSimulation.mixingStrategy}\texttt{public MixingStrategy\ {\bf  mixingStrategy}}
}
\end{itemize}
}
\subsection{Constructors}{
\rule[1em]{\hsize}{2pt}\vskip -2em
\vskip -2em
\begin{itemize}
\item{ 
\index{MixTankSimulation(float, Liquid, Pipe)}
{\bf  MixTankSimulation}\\
\begin{lstlisting}[frame=none]
public MixTankSimulation(float capacity,edu.kit.pse.osip.core.model.base.Liquid liquid,edu.kit.pse.osip.core.model.base.Pipe outPipe)\end{lstlisting} %end signature
\begin{itemize}
\item{
{\bf  Description}

Create a new mixtank.
}
\item{
{\bf  Parameters}
  \begin{itemize}
   \item{
\texttt{capacity} -- The size of the tank in cm³.}
   \item{
\texttt{liquid} -- The default content of the tank.}
   \item{
\texttt{outPipe} -- The outgoing pipe.}
  \end{itemize}
}%end item
\end{itemize}
}%end item
\end{itemize}
}
\subsection{Methods}{
\rule[1em]{\hsize}{2pt}\vskip -2em
\vskip -2em
\begin{itemize}
\item{ 
\index{putIn(Liquid)}
{\bf  putIn}\\
\begin{lstlisting}[frame=none]
public final void putIn(edu.kit.pse.osip.core.model.base.Liquid input)\end{lstlisting} %end signature
\begin{itemize}
\item{
{\bf  Description}

Add new liquid to the tank. The tank mixes the input with its current content and sets its values.
}
\item{
{\bf  Parameters}
  \begin{itemize}
   \item{
\texttt{input} -- The liquid to put in.}
  \end{itemize}
}%end item
\end{itemize}
}%end item
\divideents{takeOut}
\item{ 
\index{takeOut(float)}
{\bf  takeOut}\\
\begin{lstlisting}[frame=none]
public final edu.kit.pse.osip.core.model.base.Liquid takeOut(float amount)\end{lstlisting} %end signature
\begin{itemize}
\item{
{\bf  Description}

Takes liquid out of the tank.
}
\item{
{\bf  Parameters}
  \begin{itemize}
   \item{
\texttt{amount} -- The amount of liquid to take out.}
  \end{itemize}
}%end item
\item{{\bf  Returns} -- 
the liquid that was taken out. 
}%end item
\end{itemize}
}%end item
\end{itemize}
}
}
\section{\label{edu.kit.pse.osip.core.model.simulation.ProductionSiteSimulation}\index{ProductionSiteSimulation}Class ProductionSiteSimulation}{
\rule[1em]{\hsize}{4pt}\vskip -1em
\vskip .1in 
This is a ProductionSite which returns the simulated tanks of the package "simulation" instead of the non-simulated tanks of package "base".\vskip .1in 
\subsection{Declaration}{
\begin{lstlisting}[frame=none]
public class ProductionSiteSimulation
 extends edu.kit.pse.osip.core.model.base.ProductionSite\end{lstlisting}
\subsection{Constructors}{
\rule[1em]{\hsize}{2pt}\vskip -2em
\vskip -2em
\begin{itemize}
\item{ 
\index{ProductionSiteSimulation()}
{\bf  ProductionSiteSimulation}\\
\begin{lstlisting}[frame=none]
public ProductionSiteSimulation()\end{lstlisting} %end signature
}%end item
\end{itemize}
}
\subsection{Methods}{
\rule[1em]{\hsize}{2pt}\vskip -2em
\vskip -2em
\begin{itemize}
\item{ 
\index{getMixTank()}
{\bf  getMixTank}\\
\begin{lstlisting}[frame=none]
public final MixTankSimulation getMixTank()\end{lstlisting} %end signature
\begin{itemize}
\item{
{\bf  Description}

Get the mixtank which is extended to be simulatable.
}
\item{{\bf  Returns} -- 
the requested mixtank that is simulatable. 
}%end item
\end{itemize}
}%end item
\divideents{getUpperTank}
\item{ 
\index{getUpperTank(TankSelector)}
{\bf  getUpperTank}\\
\begin{lstlisting}[frame=none]
public final TankSimulation getUpperTank(edu.kit.pse.osip.core.model.base.TankSelector tank)\end{lstlisting} %end signature
\begin{itemize}
\item{
{\bf  Description}

Get a tank which is extended to be simulatable.
}
\item{
{\bf  Parameters}
  \begin{itemize}
   \item{
\texttt{tank} -- Specifies the tank.}
  \end{itemize}
}%end item
\item{{\bf  Returns} -- 
the requested tank that is simulatable. 
}%end item
\end{itemize}
}%end item
\end{itemize}
}
}
\section{\label{edu.kit.pse.osip.core.model.simulation.SubtractiveMixingStrategy}\index{SubtractiveMixingStrategy}Class SubtractiveMixingStrategy}{
\rule[1em]{\hsize}{4pt}\vskip -1em
\vskip .1in 
Mixing liquids using subtractive color mixture and richmann's formula for temperature mixing.\vskip .1in 
\subsection{Declaration}{
\begin{lstlisting}[frame=none]
public class SubtractiveMixingStrategy
 extends java.lang.Object implements MixingStrategy\end{lstlisting}
\subsection{Constructors}{
\rule[1em]{\hsize}{2pt}\vskip -2em
\vskip -2em
\begin{itemize}
\item{ 
\index{SubtractiveMixingStrategy()}
{\bf  SubtractiveMixingStrategy}\\
\begin{lstlisting}[frame=none]
public SubtractiveMixingStrategy()\end{lstlisting} %end signature
}%end item
\end{itemize}
}
\subsection{Methods}{
\rule[1em]{\hsize}{2pt}\vskip -2em
\vskip -2em
\begin{itemize}
\item{ 
\index{mixLiquids(LinkedList)}
{\bf  mixLiquids}\\
\begin{lstlisting}[frame=none]
public final edu.kit.pse.osip.core.model.base.Liquid mixLiquids(java.util.LinkedList inflow)\end{lstlisting} %end signature
\begin{itemize}
\item{
{\bf  Description}

Mixes the given liquids and generates a new one. Colors are mixed using a substractive strategy and the temperatures are mixed using the richmann's formula.
}
\item{
{\bf  Parameters}
  \begin{itemize}
   \item{
\texttt{inflow} -- Liquids to mix.}
  \end{itemize}
}%end item
\item{{\bf  Returns} -- 
a single Liquid element containing a mixture of the given liquids 
}%end item
\end{itemize}
}%end item
\end{itemize}
}
}
\section{\label{edu.kit.pse.osip.core.model.simulation.TankSimulation}\index{TankSimulation}Class TankSimulation}{
\rule[1em]{\hsize}{4pt}\vskip -1em
\vskip .1in 
The model of a tank, allowing to put in and take out liquid. Besides this addition, it is a normal tank.\vskip .1in 
\subsection{Declaration}{
\begin{lstlisting}[frame=none]
public class TankSimulation
 extends edu.kit.pse.osip.core.model.base.Tank\end{lstlisting}
\subsection{Fields}{
\rule[1em]{\hsize}{2pt}
\begin{itemize}
\item{
\index{mixingStrategy}
\label{edu.kit.pse.osip.core.model.simulation.TankSimulation.mixingStrategy}\texttt{public MixingStrategy\ {\bf  mixingStrategy}}
}
\end{itemize}
}
\subsection{Constructors}{
\rule[1em]{\hsize}{2pt}\vskip -2em
\vskip -2em
\begin{itemize}
\item{ 
\index{TankSimulation(float, TankSelector, Liquid, Pipe, Pipe)}
{\bf  TankSimulation}\\
\begin{lstlisting}[frame=none]
public TankSimulation(float capacity,edu.kit.pse.osip.core.model.base.TankSelector tankSelector,edu.kit.pse.osip.core.model.base.Liquid liquid,edu.kit.pse.osip.core.model.base.Pipe outPipe,edu.kit.pse.osip.core.model.base.Pipe inPipe)\end{lstlisting} %end signature
\begin{itemize}
\item{
{\bf  Description}

Create a new tank.
}
\item{
{\bf  Parameters}
  \begin{itemize}
   \item{
\texttt{capacity} -- The size of the tank in cm³.}
   \item{
\texttt{tankSelector} -- Specifies which tank it is. This is useful if you have access to the tank and need to get a unique identifier.}
   \item{
\texttt{liquid} -- The default content of the tank.}
   \item{
\texttt{outPipe} -- The outgoing pipe.}
   \item{
\texttt{inPipe} -- The incoming pipe.}
  \end{itemize}
}%end item
\end{itemize}
}%end item
\end{itemize}
}
\subsection{Methods}{
\rule[1em]{\hsize}{2pt}\vskip -2em
\vskip -2em
\begin{itemize}
\item{ 
\index{putIn(Liquid)}
{\bf  putIn}\\
\begin{lstlisting}[frame=none]
public final void putIn(edu.kit.pse.osip.core.model.base.Liquid input)\end{lstlisting} %end signature
\begin{itemize}
\item{
{\bf  Description}

Add new liquid to the tank. The tank mixes the input with its current content and sets its values.
}
\item{
{\bf  Parameters}
  \begin{itemize}
   \item{
\texttt{input} -- The liquid to put in.}
  \end{itemize}
}%end item
\end{itemize}
}%end item
\divideents{takeOut}
\item{ 
\index{takeOut(float)}
{\bf  takeOut}\\
\begin{lstlisting}[frame=none]
public final edu.kit.pse.osip.core.model.base.Liquid takeOut(float amount)\end{lstlisting} %end signature
\begin{itemize}
\item{
{\bf  Description}

Takes liquid out of the tank.
}
\item{
{\bf  Parameters}
  \begin{itemize}
   \item{
\texttt{amount} -- The amount of liquid to take out.}
  \end{itemize}
}%end item
\item{{\bf  Returns} -- 
the liquid that was taken out. 
}%end item
\end{itemize}
}%end item
\end{itemize}
}
}
}
\chapter{Package edu.kit.pse.osip.core.model.base}{
\label{edu.kit.pse.osip.core.model.base}\section{\label{edu.kit.pse.osip.core.model.base.AbstractTank}\index{AbstractTank}Class AbstractTank}{
\rule[1em]{\hsize}{4pt}\vskip -1em
\vskip .1in 
The model of a tank. It is the base class of the mixtank and the upper tanks, so it has some shared functionality which is avaiable in the mixtank and the upper tanks.\vskip .1in 
\subsection{Declaration}{
\begin{lstlisting}[frame=none]
public abstract class AbstractTank
 extends java.util.Observable\end{lstlisting}
\subsection{All known subclasses}{TankSimulation\small{\refdefined{edu.kit.pse.osip.core.model.simulation.TankSimulation}}, MixTankSimulation\small{\refdefined{edu.kit.pse.osip.core.model.simulation.MixTankSimulation}}, Tank\small{\refdefined{edu.kit.pse.osip.core.model.base.Tank}}, MixTank\small{\refdefined{edu.kit.pse.osip.core.model.base.MixTank}}}
\subsection{Fields}{
\rule[1em]{\hsize}{2pt}
\begin{itemize}
\item{
\index{liquid}
\label{edu.kit.pse.osip.core.model.base.AbstractTank.liquid}\texttt{public Liquid\ {\bf  liquid}}
}
\end{itemize}
}
\subsection{Constructors}{
\rule[1em]{\hsize}{2pt}\vskip -2em
\vskip -2em
\begin{itemize}
\item{ 
\index{AbstractTank()}
{\bf  AbstractTank}\\
\begin{lstlisting}[frame=none]
public AbstractTank()\end{lstlisting} %end signature
}%end item
\end{itemize}
}
\subsection{Methods}{
\rule[1em]{\hsize}{2pt}\vskip -2em
\vskip -2em
\begin{itemize}
\item{ 
\index{getFillLevel()}
{\bf  getFillLevel}\\
\begin{lstlisting}[frame=none]
public final float getFillLevel()\end{lstlisting} %end signature
\begin{itemize}
\item{
{\bf  Description}

Get the fill level of the tank in \%.
}
\item{{\bf  Returns} -- 
the fill level in \%. 
}%end item
\end{itemize}
}%end item
\divideents{getLiquid}
\item{ 
\index{getLiquid()}
{\bf  getLiquid}\\
\begin{lstlisting}[frame=none]
public final Liquid getLiquid()\end{lstlisting} %end signature
\begin{itemize}
\item{
{\bf  Description}

Get the liquid of the tank.
}
\item{{\bf  Returns} -- 
the liquid of the tank. 
}%end item
\end{itemize}
}%end item
\divideents{getOutPipe}
\item{ 
\index{getOutPipe()}
{\bf  getOutPipe}\\
\begin{lstlisting}[frame=none]
public final Pipe getOutPipe()\end{lstlisting} %end signature
\begin{itemize}
\item{
{\bf  Description}

Get the outgoing pipe of the tank.
}
\item{{\bf  Returns} -- 
the outgoing pipe. 
}%end item
\end{itemize}
}%end item
\divideents{getTankSelector}
\item{ 
\index{getTankSelector()}
{\bf  getTankSelector}\\
\begin{lstlisting}[frame=none]
public final TankSelector getTankSelector()\end{lstlisting} %end signature
\begin{itemize}
\item{
{\bf  Description}

Get the tank selector. This is useful if you have access to the tank and need to get a unique identifier.
}
\item{{\bf  Returns} -- 
the tank selecor. 
}%end item
\end{itemize}
}%end item
\divideents{setLiquid}
\item{ 
\index{setLiquid(Liquid)}
{\bf  setLiquid}\\
\begin{lstlisting}[frame=none]
public final void setLiquid(Liquid liquid)\end{lstlisting} %end signature
\begin{itemize}
\item{
{\bf  Description}

Set the liquid of the tank. It doesn't throw an exception if you put in more liquid than possible, so you can detect overflow conditions.
}
\item{
{\bf  Parameters}
  \begin{itemize}
   \item{
\texttt{liquid} -- The liquid which should be in the tank afterwards.}
  \end{itemize}
}%end item
\end{itemize}
}%end item
\divideents{Tank}
\item{ 
\index{Tank(float, TankSelector, Liquid, Pipe)}
{\bf  Tank}\\
\begin{lstlisting}[frame=none]
public final void Tank(float capacity,TankSelector tankSelector,Liquid liquid,Pipe outPipe)\end{lstlisting} %end signature
\begin{itemize}
\item{
{\bf  Description}

Create a new tank.
}
\item{
{\bf  Parameters}
  \begin{itemize}
   \item{
\texttt{capacity} -- The size of the tank in cm³.}
   \item{
\texttt{tankSelector} -- Specifies which tank it is. This is useful if you have access to the tank and need to get a unique identifier.}
   \item{
\texttt{liquid} -- The default liquid of the tank.}
   \item{
\texttt{outPipe} -- The outgoing pipe.}
  \end{itemize}
}%end item
\end{itemize}
}%end item
\end{itemize}
}
}
\section{\label{edu.kit.pse.osip.core.model.base.Color}\index{Color}Class Color}{
\rule[1em]{\hsize}{4pt}\vskip -1em
\vskip .1in 
A color in the CMY (cyan, magenta, yellow) color space.\vskip .1in 
\subsection{Declaration}{
\begin{lstlisting}[frame=none]
public class Color
 extends java.lang.Object\end{lstlisting}
\subsection{Constructors}{
\rule[1em]{\hsize}{2pt}\vskip -2em
\vskip -2em
\begin{itemize}
\item{ 
\index{Color(short, short, short)}
{\bf  Color}\\
\begin{lstlisting}[frame=none]
public Color(short cyan,short yellow,short magenta)\end{lstlisting} %end signature
\begin{itemize}
\item{
{\bf  Description}

Construct a new Color object.
}
\item{
{\bf  Parameters}
  \begin{itemize}
   \item{
\texttt{cyan} -- The percentage of the cyan color. It needs to be between 0 and 100.}
   \item{
\texttt{yellow} -- The percentage of the yellow color. It needs to be between 0 and 100.}
   \item{
\texttt{magenta} -- The percentage of the magenta color. It needs to be between 0 and 100.}
  \end{itemize}
}%end item
\item{{\bf  Throws}
  \begin{itemize}
   \item{\vskip -.6ex \texttt{java.lang.IllegalArgumentException} -- if cyan, yellow or magenta is smaller than 0 or greather than 255.}
  \end{itemize}
}%end item
\end{itemize}
}%end item
\end{itemize}
}
\subsection{Methods}{
\rule[1em]{\hsize}{2pt}\vskip -2em
\vskip -2em
\begin{itemize}
\item{ 
\index{getB()}
{\bf  getB}\\
\begin{lstlisting}[frame=none]
public final short getB()\end{lstlisting} %end signature
\begin{itemize}
\item{
{\bf  Description}

Returns the b value when converting this color to RGB.
}
\item{{\bf  Returns} -- 
The b value. 
}%end item
\end{itemize}
}%end item
\divideents{getCyan}
\item{ 
\index{getCyan()}
{\bf  getCyan}\\
\begin{lstlisting}[frame=none]
public final short getCyan()\end{lstlisting} %end signature
\begin{itemize}
\item{
{\bf  Description}

Get the cyan percentage.
}
\item{{\bf  Returns} -- 
the percentage of the cyan color. 
}%end item
\end{itemize}
}%end item
\divideents{getG}
\item{ 
\index{getG()}
{\bf  getG}\\
\begin{lstlisting}[frame=none]
public final short getG()\end{lstlisting} %end signature
\begin{itemize}
\item{
{\bf  Description}

Returns the g value when converting this color to RGB.
}
\item{{\bf  Returns} -- 
The g value. 
}%end item
\end{itemize}
}%end item
\divideents{getMagenta}
\item{ 
\index{getMagenta()}
{\bf  getMagenta}\\
\begin{lstlisting}[frame=none]
public final short getMagenta()\end{lstlisting} %end signature
\begin{itemize}
\item{
{\bf  Description}

Get the magenta percentage.
}
\item{{\bf  Returns} -- 
the percentage of the magenta color. 
}%end item
\end{itemize}
}%end item
\divideents{getR}
\item{ 
\index{getR()}
{\bf  getR}\\
\begin{lstlisting}[frame=none]
public final short getR()\end{lstlisting} %end signature
\begin{itemize}
\item{
{\bf  Description}

Returns the r value when converting this color to RGB.
}
\item{{\bf  Returns} -- 
The r value. 
}%end item
\end{itemize}
}%end item
\divideents{getYellow}
\item{ 
\index{getYellow()}
{\bf  getYellow}\\
\begin{lstlisting}[frame=none]
public final short getYellow()\end{lstlisting} %end signature
\begin{itemize}
\item{
{\bf  Description}

Get the yellow percentage.
}
\item{{\bf  Returns} -- 
the percentage of the yellow color. 
}%end item
\end{itemize}
}%end item
\divideents{mix}
\item{ 
\index{mix(Color, Color, float)}
{\bf  mix}\\
\begin{lstlisting}[frame=none]
public static final Color mix(Color color1,Color color2,float ratio)\end{lstlisting} %end signature
\begin{itemize}
\item{
{\bf  Description}

Mix two colors.
}
\item{
{\bf  Parameters}
  \begin{itemize}
   \item{
\texttt{color1} -- The first color to mix.}
   \item{
\texttt{color2} -- The second color to mix.}
   \item{
\texttt{ratio} -- The ratio of color1/color2, e.g. "1" means that both colors have the same intensity.}
  \end{itemize}
}%end item
\item{{\bf  Returns} -- 
the new color. 
}%end item
\item{{\bf  Throws}
  \begin{itemize}
   \item{\vskip -.6ex \texttt{java.lang.IllegalArgumentException} -- if ratio is = 0.}
  \end{itemize}
}%end item
\end{itemize}
}%end item
\divideents{setCyan}
\item{ 
\index{setCyan(short)}
{\bf  setCyan}\\
\begin{lstlisting}[frame=none]
public final void setCyan(short cyan)\end{lstlisting} %end signature
\begin{itemize}
\item{
{\bf  Description}

Set the cyan percentage.
}
\item{
{\bf  Parameters}
  \begin{itemize}
   \item{
\texttt{cyan} -- The percentage of the cyan color. It needs to be between 0 and 100.}
  \end{itemize}
}%end item
\item{{\bf  Throws}
  \begin{itemize}
   \item{\vskip -.6ex \texttt{java.lang.IllegalArgumentException} -- if cyan is smaller than 0 or greather than 255.}
  \end{itemize}
}%end item
\end{itemize}
}%end item
\divideents{setMagenta}
\item{ 
\index{setMagenta(short)}
{\bf  setMagenta}\\
\begin{lstlisting}[frame=none]
public final void setMagenta(short magenta)\end{lstlisting} %end signature
\begin{itemize}
\item{
{\bf  Description}

Set the magenta percentage.
}
\item{
{\bf  Parameters}
  \begin{itemize}
   \item{
\texttt{magenta} -- The percentage of the magenta color. It needs to be between 0 and 100.}
  \end{itemize}
}%end item
\item{{\bf  Throws}
  \begin{itemize}
   \item{\vskip -.6ex \texttt{java.lang.IllegalArgumentException} -- if magenta is smaller than 0 or greather than 255.}
  \end{itemize}
}%end item
\end{itemize}
}%end item
\divideents{setYellow}
\item{ 
\index{setYellow(short)}
{\bf  setYellow}\\
\begin{lstlisting}[frame=none]
public final void setYellow(short yellow)\end{lstlisting} %end signature
\begin{itemize}
\item{
{\bf  Description}

Set the yellow percentage.
}
\item{
{\bf  Parameters}
  \begin{itemize}
   \item{
\texttt{yellow} -- The percentage of the yellow color. It needs to be between 0 and 100.}
  \end{itemize}
}%end item
\item{{\bf  Throws}
  \begin{itemize}
   \item{\vskip -.6ex \texttt{java.lang.IllegalArgumentException} -- if yellow is smaller than 0 or greather than 255.}
  \end{itemize}
}%end item
\end{itemize}
}%end item
\end{itemize}
}
}
\section{\label{edu.kit.pse.osip.core.model.base.Liquid}\index{Liquid}Class Liquid}{
\rule[1em]{\hsize}{4pt}\vskip -1em
\vskip .1in 
A readonly data type for liquids. This is the liquid which flows through the tanks and pipes. It has attributes for the color, the temperature and the amount.\vskip .1in 
\subsection{Declaration}{
\begin{lstlisting}[frame=none]
public class Liquid
 extends java.lang.Object\end{lstlisting}
\subsection{Constructors}{
\rule[1em]{\hsize}{2pt}\vskip -2em
\vskip -2em
\begin{itemize}
\item{ 
\index{Liquid(float, float, Color)}
{\bf  Liquid}\\
\begin{lstlisting}[frame=none]
public Liquid(float amount,float temperature,Color color)\end{lstlisting} %end signature
\begin{itemize}
\item{
{\bf  Description}

Construct a new Liquid object.
}
\item{
{\bf  Parameters}
  \begin{itemize}
   \item{
\texttt{amount} -- The amout in cm³.}
   \item{
\texttt{temperature} -- The temperature in °K.}
   \item{
\texttt{color} -- The color of the liquid.}
  \end{itemize}
}%end item
\end{itemize}
}%end item
\end{itemize}
}
\subsection{Methods}{
\rule[1em]{\hsize}{2pt}\vskip -2em
\vskip -2em
\begin{itemize}
\item{ 
\index{getAmount()}
{\bf  getAmount}\\
\begin{lstlisting}[frame=none]
public final float getAmount()\end{lstlisting} %end signature
\begin{itemize}
\item{
{\bf  Description}

Get the amount of the liquid.
}
\item{{\bf  Returns} -- 
the amount. 
}%end item
\end{itemize}
}%end item
\divideents{getColor}
\item{ 
\index{getColor()}
{\bf  getColor}\\
\begin{lstlisting}[frame=none]
public final Color getColor()\end{lstlisting} %end signature
\begin{itemize}
\item{
{\bf  Description}

Get the color of the Liquid.
}
\item{{\bf  Returns} -- 
the color. 
}%end item
\end{itemize}
}%end item
\divideents{getTemperature}
\item{ 
\index{getTemperature()}
{\bf  getTemperature}\\
\begin{lstlisting}[frame=none]
public final float getTemperature()\end{lstlisting} %end signature
\begin{itemize}
\item{
{\bf  Description}

Get the temperature of the liquid.
}
\item{{\bf  Returns} -- 
the temperature. 
}%end item
\end{itemize}
}%end item
\end{itemize}
}
}
\section{\label{edu.kit.pse.osip.core.model.base.MixTank}\index{MixTank}Class MixTank}{
\rule[1em]{\hsize}{4pt}\vskip -1em
\vskip .1in 
This is a mixtank. The difference to AbstractTank is the presence of the motor, which mixes the liquids.\vskip .1in 
\subsection{Declaration}{
\begin{lstlisting}[frame=none]
public class MixTank
 extends edu.kit.pse.osip.core.model.base.AbstractTank\end{lstlisting}
\subsection{All known subclasses}{MixTankSimulation\small{\refdefined{edu.kit.pse.osip.core.model.simulation.MixTankSimulation}}}
\subsection{Fields}{
\rule[1em]{\hsize}{2pt}
\begin{itemize}
\item{
\index{motor}
\label{edu.kit.pse.osip.core.model.base.MixTank.motor}\texttt{public Motor\ {\bf  motor}}
}
\end{itemize}
}
\subsection{Constructors}{
\rule[1em]{\hsize}{2pt}\vskip -2em
\vskip -2em
\begin{itemize}
\item{ 
\index{MixTank(float, Liquid, Pipe)}
{\bf  MixTank}\\
\begin{lstlisting}[frame=none]
public MixTank(float capacity,Liquid liquid,Pipe outPipe)\end{lstlisting} %end signature
\begin{itemize}
\item{
{\bf  Description}

Create a new mixtank.
}
\item{
{\bf  Parameters}
  \begin{itemize}
   \item{
\texttt{capacity} -- The capacity of the tank in cm³.}
   \item{
\texttt{liquid} -- The default content of the tank.}
   \item{
\texttt{outPipe} -- The outgoing pipe.}
  \end{itemize}
}%end item
\end{itemize}
}%end item
\end{itemize}
}
\subsection{Methods}{
\rule[1em]{\hsize}{2pt}\vskip -2em
\vskip -2em
\begin{itemize}
\item{ 
\index{getMotor()}
{\bf  getMotor}\\
\begin{lstlisting}[frame=none]
public final Motor getMotor()\end{lstlisting} %end signature
\begin{itemize}
\item{
{\bf  Description}

Get the motor attached to the tank
}
\item{{\bf  Returns} -- 
the motor 
}%end item
\end{itemize}
}%end item
\end{itemize}
}
}
\section{\label{edu.kit.pse.osip.core.model.base.Motor}\index{Motor}Class Motor}{
\rule[1em]{\hsize}{4pt}\vskip -1em
\vskip .1in 
This is the motor which is installed in the mixtank.\vskip .1in 
\subsection{Declaration}{
\begin{lstlisting}[frame=none]
public class Motor
 extends java.util.Observable\end{lstlisting}
\subsection{Constructors}{
\rule[1em]{\hsize}{2pt}\vskip -2em
\vskip -2em
\begin{itemize}
\item{ 
\index{Motor()}
{\bf  Motor}\\
\begin{lstlisting}[frame=none]
public Motor()\end{lstlisting} %end signature
}%end item
\end{itemize}
}
\subsection{Methods}{
\rule[1em]{\hsize}{2pt}\vskip -2em
\vskip -2em
\begin{itemize}
\item{ 
\index{getRPM()}
{\bf  getRPM}\\
\begin{lstlisting}[frame=none]
public final int getRPM()\end{lstlisting} %end signature
\begin{itemize}
\item{
{\bf  Description}

Get the RPM.
}
\item{{\bf  Returns} -- 
the RPM. 
}%end item
\end{itemize}
}%end item
\divideents{setRPM}
\item{ 
\index{setRPM(int)}
{\bf  setRPM}\\
\begin{lstlisting}[frame=none]
public final void setRPM(int rpm)\end{lstlisting} %end signature
\begin{itemize}
\item{
{\bf  Description}

Set the RPM of the motor.
}
\item{
{\bf  Parameters}
  \begin{itemize}
   \item{
\texttt{rpm} -- The target RPM.}
  \end{itemize}
}%end item
\item{{\bf  Throws}
  \begin{itemize}
   \item{\vskip -.6ex \texttt{java.lang.IllegalArgumentException} -- if rpm is negative or greater than 3000.}
  \end{itemize}
}%end item
\end{itemize}
}%end item
\end{itemize}
}
}
\section{\label{edu.kit.pse.osip.core.model.base.Pipe}\index{Pipe}Class Pipe}{
\rule[1em]{\hsize}{4pt}\vskip -1em
\vskip .1in 
A pipe, where you can insert and take out liquid. It is a queue, so if you put in liquid, you can take it out in @see length steps. It has a valve attached, so you can limit the throughput. If you limit the throughput or put in less liquid than possible, the liquid of takeOut will be smaller when the put in liquid reaches the other end. You can put in the liquid of \$SIM\_STEP\_SIZE milimeter of the pipe at once.\vskip .1in 
\subsection{Declaration}{
\begin{lstlisting}[frame=none]
public class Pipe
 extends java.util.Observable\end{lstlisting}
\subsection{Constructors}{
\rule[1em]{\hsize}{2pt}\vskip -2em
\vskip -2em
\begin{itemize}
\item{ 
\index{Pipe(float, int)}
{\bf  Pipe}\\
\begin{lstlisting}[frame=none]
public Pipe(float corsssection,int length)\end{lstlisting} %end signature
\begin{itemize}
\item{
{\bf  Description}

Construct a Pipe.
}
\item{
{\bf  Parameters}
  \begin{itemize}
   \item{
\texttt{corsssection} -- Crosssection of the pipe in cm².}
   \item{
\texttt{length} -- Length in cm.}
  \end{itemize}
}%end item
\end{itemize}
}%end item
\end{itemize}
}
\subsection{Methods}{
\rule[1em]{\hsize}{2pt}\vskip -2em
\vskip -2em
\begin{itemize}
\item{ 
\index{getMaxInput()}
{\bf  getMaxInput}\\
\begin{lstlisting}[frame=none]
public final float getMaxInput()\end{lstlisting} %end signature
\begin{itemize}
\item{
{\bf  Description}

This tells you the maximal amount of liquid in cm³, which you can put into the pipe. It is calculated as: crosssection*\$SIM\_STEP\_SIZE*threshold (\$SIM\_STEP\_SIZE, because you fill one \$SIM\_STEP\_SIZE with every call to putIn).
}
\item{{\bf  Returns} -- 
the maximum amout of liquid you can put into the pipe with one call to putIn(). 
}%end item
\end{itemize}
}%end item
\divideents{getValveThreshold}
\item{ 
\index{getValveThreshold()}
{\bf  getValveThreshold}\\
\begin{lstlisting}[frame=none]
public final byte getValveThreshold()\end{lstlisting} %end signature
\begin{itemize}
\item{
{\bf  Description}

Get the valve threshold in \%.
}
\item{{\bf  Returns} -- 
the threshold, which is between 0 and 100. 
}%end item
\end{itemize}
}%end item
\divideents{putIn}
\item{ 
\index{putIn(Liquid)}
{\bf  putIn}\\
\begin{lstlisting}[frame=none]
public final Liquid putIn(Liquid liquid)\end{lstlisting} %end signature
\begin{itemize}
\item{
{\bf  Description}

Insert liquid into one \$SIM\_STEP\_SIZE of the pipe and take out the liquid at the other side, which gets pushed out.
}
\item{
{\bf  Parameters}
  \begin{itemize}
   \item{
\texttt{liquid} -- The liquid to put into the pipe.}
  \end{itemize}
}%end item
\item{{\bf  Returns} -- 
the liquid which was pushed out at the other side. 
}%end item
\item{{\bf  Throws}
  \begin{itemize}
   \item{\vskip -.6ex \texttt{edu.kit.pse.osip.core.model.base.OverfullLiquidContainerException} -- if the pipe is full and you need to take out some liquit first, or if you try to put more liquid into the pipe, than getMaxInput tells you.}
  \end{itemize}
}%end item
\end{itemize}
}%end item
\divideents{setValveThreshold}
\item{ 
\index{setValveThreshold(byte)}
{\bf  setValveThreshold}\\
\begin{lstlisting}[frame=none]
public final void setValveThreshold(byte threshold)\end{lstlisting} %end signature
\begin{itemize}
\item{
{\bf  Description}

Set the threshold in \%.
}
\item{
{\bf  Parameters}
  \begin{itemize}
   \item{
\texttt{threshold} -- The threshold. It needs to be between 0 and 100.}
  \end{itemize}
}%end item
\item{{\bf  Throws}
  \begin{itemize}
   \item{\vskip -.6ex \texttt{java.lang.IllegalArgumentException} -- if threshold is not between 0 and 100.}
  \end{itemize}
}%end item
\end{itemize}
}%end item
\end{itemize}
}
}
\section{\label{edu.kit.pse.osip.core.model.base.ProductionSite}\index{ProductionSite}Class ProductionSite}{
\rule[1em]{\hsize}{4pt}\vskip -1em
\vskip .1in 
Group all tanks in the production site together. This is the entrance point of the model, because you can get every tank and, through the tanks, every pipe in the production site.\vskip .1in 
\subsection{Declaration}{
\begin{lstlisting}[frame=none]
public class ProductionSite
 extends java.lang.Object\end{lstlisting}
\subsection{All known subclasses}{ProductionSiteSimulation\small{\refdefined{edu.kit.pse.osip.core.model.simulation.ProductionSiteSimulation}}}
\subsection{Fields}{
\rule[1em]{\hsize}{2pt}
\begin{itemize}
\item{
\index{mixTank}
\label{edu.kit.pse.osip.core.model.base.ProductionSite.mixTank}\texttt{public MixTank\ {\bf  mixTank}}
}
\item{
\index{tank}
\label{edu.kit.pse.osip.core.model.base.ProductionSite.tank}\texttt{public Tank\ {\bf  tank}}
}
\end{itemize}
}
\subsection{Constructors}{
\rule[1em]{\hsize}{2pt}\vskip -2em
\vskip -2em
\begin{itemize}
\item{ 
\index{ProductionSite()}
{\bf  ProductionSite}\\
\begin{lstlisting}[frame=none]
public ProductionSite()\end{lstlisting} %end signature
}%end item
\end{itemize}
}
\subsection{Methods}{
\rule[1em]{\hsize}{2pt}\vskip -2em
\vskip -2em
\begin{itemize}
\item{ 
\index{getMixTank()}
{\bf  getMixTank}\\
\begin{lstlisting}[frame=none]
public MixTank getMixTank()\end{lstlisting} %end signature
\begin{itemize}
\item{
{\bf  Description}

Get the mixtank.
}
\item{{\bf  Returns} -- 
the mixtank of the production site. 
}%end item
\end{itemize}
}%end item
\divideents{getUpperTank}
\item{ 
\index{getUpperTank(TankSelector)}
{\bf  getUpperTank}\\
\begin{lstlisting}[frame=none]
public Tank getUpperTank(TankSelector tank)\end{lstlisting} %end signature
\begin{itemize}
\item{
{\bf  Description}

Get one of the upper tanks.
}
\item{
{\bf  Parameters}
  \begin{itemize}
   \item{
\texttt{tank} -- Specifies the tank.}
  \end{itemize}
}%end item
\item{{\bf  Returns} -- 
the requested tank. 
}%end item
\end{itemize}
}%end item
\divideents{reset}
\item{ 
\index{reset()}
{\bf  reset}\\
\begin{lstlisting}[frame=none]
public final void reset()\end{lstlisting} %end signature
\begin{itemize}
\item{
{\bf  Description}

Reset the whole production site to its default values: Every tank with 50\%\ infill, valves putting the site to a stable state.
}
\end{itemize}
}%end item
\end{itemize}
}
}
\section{\label{edu.kit.pse.osip.core.model.base.Tank}\index{Tank}Class Tank}{
\rule[1em]{\hsize}{4pt}\vskip -1em
\vskip .1in 
This is the model of an upper tank. The difference to AbstractTank is the pipe which puts liquid into the tank (a mixtank does not have this pipe).\vskip .1in 
\subsection{Declaration}{
\begin{lstlisting}[frame=none]
public class Tank
 extends edu.kit.pse.osip.core.model.base.AbstractTank\end{lstlisting}
\subsection{All known subclasses}{TankSimulation\small{\refdefined{edu.kit.pse.osip.core.model.simulation.TankSimulation}}}
\subsection{Constructors}{
\rule[1em]{\hsize}{2pt}\vskip -2em
\vskip -2em
\begin{itemize}
\item{ 
\index{Tank(float, TankSelector, Liquid, Pipe, Pipe)}
{\bf  Tank}\\
\begin{lstlisting}[frame=none]
public Tank(float capacity,TankSelector tankSelector,Liquid liquid,Pipe outPipe,Pipe inPipe)\end{lstlisting} %end signature
\begin{itemize}
\item{
{\bf  Description}

Constructs a new Tank.
}
\item{
{\bf  Parameters}
  \begin{itemize}
   \item{
\texttt{capacity} -- The capacity of a tank in cm³.}
   \item{
\texttt{tankSelector} -- Specifies which tank it is. This is useful if you have access to the tank and need to get a unique identifier.}
   \item{
\texttt{liquid} -- The default content of the tank.}
   \item{
\texttt{outPipe} -- The outgoing pipe.}
   \item{
\texttt{inPipe} -- The incoming pipe.}
  \end{itemize}
}%end item
\end{itemize}
}%end item
\end{itemize}
}
\subsection{Methods}{
\rule[1em]{\hsize}{2pt}\vskip -2em
\vskip -2em
\begin{itemize}
\item{ 
\index{getInPipe()}
{\bf  getInPipe}\\
\begin{lstlisting}[frame=none]
public final Pipe getInPipe()\end{lstlisting} %end signature
\begin{itemize}
\item{
{\bf  Description}

Get the pipe which puts liquid into the tank.
}
\item{{\bf  Returns} -- 
the ingoing pipe. 
}%end item
\end{itemize}
}%end item
\end{itemize}
}
}
\section{\label{edu.kit.pse.osip.core.model.base.TankSelector}\index{TankSelector}Class TankSelector}{
\rule[1em]{\hsize}{4pt}\vskip -1em
\vskip .1in 
This enumeration defines all tanks including their default color value.\vskip .1in 
\subsection{Declaration}{
\begin{lstlisting}[frame=none]
public final class TankSelector
 extends java.lang.Enum\end{lstlisting}
\subsection{Fields}{
\rule[1em]{\hsize}{2pt}
\begin{itemize}
\item{
\index{MIX}
\label{edu.kit.pse.osip.core.model.base.TankSelector.MIX}\texttt{public static final TankSelector\ {\bf  MIX}}
\begin{itemize}
\item{\vskip -.9ex 
the mixtank}
\end{itemize}
}
\item{
\index{YELLOW}
\label{edu.kit.pse.osip.core.model.base.TankSelector.YELLOW}\texttt{public static final TankSelector\ {\bf  YELLOW}}
\begin{itemize}
\item{\vskip -.9ex 
The yellow tank}
\end{itemize}
}
\item{
\index{CYAN}
\label{edu.kit.pse.osip.core.model.base.TankSelector.CYAN}\texttt{public static final TankSelector\ {\bf  CYAN}}
\begin{itemize}
\item{\vskip -.9ex 
The cyan colored tank}
\end{itemize}
}
\item{
\index{MAGENTA}
\label{edu.kit.pse.osip.core.model.base.TankSelector.MAGENTA}\texttt{public static final TankSelector\ {\bf  MAGENTA}}
\begin{itemize}
\item{\vskip -.9ex 
The magenta colored tank}
\end{itemize}
}
\end{itemize}
}
\subsection{Methods}{
\rule[1em]{\hsize}{2pt}\vskip -2em
\vskip -2em
\begin{itemize}
\item{ 
\index{getInitialColor()}
{\bf  getInitialColor}\\
\begin{lstlisting}[frame=none]
public Color getInitialColor()\end{lstlisting} %end signature
\begin{itemize}
\item{
{\bf  Description}

Get the initial color of a tank.
}
\item{{\bf  Returns} -- 
The initial color of a tank. 
}%end item
\end{itemize}
}%end item
\divideents{getUpperTankCount}
\item{ 
\index{getUpperTankCount()}
{\bf  getUpperTankCount}\\
\begin{lstlisting}[frame=none]
public int getUpperTankCount()\end{lstlisting} %end signature
\begin{itemize}
\item{
{\bf  Description}

Get the number of upper (non-mix) tanks in the production site. So the total number of tanks is getUpperTankCount()+1 because of the mixtank.
}
\item{{\bf  Returns} -- 
the number of upper tanks. 
}%end item
\end{itemize}
}%end item
\divideents{valueOf}
\item{ 
\index{valueOf(String)}
{\bf  valueOf}\\
\begin{lstlisting}[frame=none]
public static TankSelector valueOf(java.lang.String name)\end{lstlisting} %end signature
}%end item
\divideents{values}
\item{ 
\index{values()}
{\bf  values}\\
\begin{lstlisting}[frame=none]
public static TankSelector[] values()\end{lstlisting} %end signature
}%end item
\divideents{valuesWithoutMix}
\item{ 
\index{valuesWithoutMix()}
{\bf  valuesWithoutMix}\\
\begin{lstlisting}[frame=none]
public TankSelector[] valuesWithoutMix()\end{lstlisting} %end signature
\begin{itemize}
\item{
{\bf  Description}

Return an array of all tanks, except for the mixtank.
}
\item{{\bf  Returns} -- 
Array of all tanks, except for the mixtank. 
}%end item
\end{itemize}
}%end item
\end{itemize}
}
}
\section{\label{edu.kit.pse.osip.core.model.base.OverfullLiquidContainerException}\index{OverfullLiquidContainerException}Exception OverfullLiquidContainerException}{
\rule[1em]{\hsize}{4pt}\vskip -1em
\vskip .1in 
Thrown to indicate that a too big liquid got inserted into a pipe or a tank.\vskip .1in 
\subsection{Declaration}{
\begin{lstlisting}[frame=none]
public class OverfullLiquidContainerException
 extends java.lang.IllegalArgumentException\end{lstlisting}
\subsection{Constructors}{
\rule[1em]{\hsize}{2pt}\vskip -2em
\vskip -2em
\begin{itemize}
\item{ 
\index{OverfullLiquidContainerException(String, float, float)}
{\bf  OverfullLiquidContainerException}\\
\begin{lstlisting}[frame=none]
public OverfullLiquidContainerException(java.lang.String message,float maximumAmount,float triedAmount)\end{lstlisting} %end signature
\begin{itemize}
\item{
{\bf  Description}

Create a new OverfullLiquidContainerException
}
\item{
{\bf  Parameters}
  \begin{itemize}
   \item{
\texttt{message} -- The error message.}
   \item{
\texttt{maximumAmount} -- How much one can insert at maximum.}
   \item{
\texttt{triedAmount} -- How much was tried to be inserted. This should be more than maximumAmount.}
  \end{itemize}
}%end item
\item{{\bf  Throws}
  \begin{itemize}
   \item{\vskip -.6ex \texttt{java.lang.IllegalArgumentException} -- if maximumAmount or triedAmount is 0.}
  \end{itemize}
}%end item
\end{itemize}
}%end item
\end{itemize}
}
\subsection{Methods}{
\rule[1em]{\hsize}{2pt}\vskip -2em
\vskip -2em
\begin{itemize}
\item{ 
\index{getMaximumAmount()}
{\bf  getMaximumAmount}\\
\begin{lstlisting}[frame=none]
public final float getMaximumAmount()\end{lstlisting} %end signature
\begin{itemize}
\item{
{\bf  Description}

Returns the maximum amount
}
\item{{\bf  Returns} -- 
the maximum amount. 
}%end item
\end{itemize}
}%end item
\divideents{getTriedAmount}
\item{ 
\index{getTriedAmount()}
{\bf  getTriedAmount}\\
\begin{lstlisting}[frame=none]
public final float getTriedAmount()\end{lstlisting} %end signature
\begin{itemize}
\item{
{\bf  Description}

Returns the amount that was tried to input.
}
\item{{\bf  Returns} -- 
the tried amount. 
}%end item
\end{itemize}
}%end item
\end{itemize}
}
}
}
\chapter{Package edu.kit.pse.osip.simulation.controller}{
\label{edu.kit.pse.osip.simulation.controller}\section{\label{edu.kit.pse.osip.simulation.controller.FlowRateListener}\index{FlowRateListener@\textit{ FlowRateListener}}Interface FlowRateListener}{
\rule[1em]{\hsize}{4pt}\vskip -1em
\vskip .1in 
This Listener listens for changes in the in- and outflow slider.\vskip .1in 
\subsection{Declaration}{
\begin{lstlisting}[frame=none]
public interface FlowRateListener
\end{lstlisting}
\subsection{Methods}{
\rule[1em]{\hsize}{2pt}\vskip -2em
\vskip -2em
\begin{itemize}
\item{ 
\index{onFlowRateUpdated(TankSelector)}
{\bf  onFlowRateUpdated}\\
\begin{lstlisting}[frame=none]
void onFlowRateUpdated(edu.kit.pse.osip.core.model.base.TankSelector tank)\end{lstlisting} %end signature
\begin{itemize}
\item{
{\bf  Description}

Alerts the Controller that the flow changed.
}
\item{
{\bf  Parameters}
  \begin{itemize}
   \item{
\texttt{tank} -- The tank that was updated}
  \end{itemize}
}%end item
\end{itemize}
}%end item
\end{itemize}
}
}
\section{\label{edu.kit.pse.osip.simulation.controller.MotorListener}\index{MotorListener@\textit{ MotorListener}}Interface MotorListener}{
\rule[1em]{\hsize}{4pt}\vskip -1em
\vskip .1in 
This Listener listens for changes in the motor speed slider.\vskip .1in 
\subsection{Declaration}{
\begin{lstlisting}[frame=none]
public interface MotorListener
\end{lstlisting}
\subsection{Methods}{
\rule[1em]{\hsize}{2pt}\vskip -2em
\vskip -2em
\begin{itemize}
\item{ 
\index{onSpeedUpdated()}
{\bf  onSpeedUpdated}\\
\begin{lstlisting}[frame=none]
void onSpeedUpdated()\end{lstlisting} %end signature
\begin{itemize}
\item{
{\bf  Description}

Alerts the Controller that the speed changed.
}
\end{itemize}
}%end item
\end{itemize}
}
}
\section{\label{edu.kit.pse.osip.simulation.controller.SettingsChangedListener}\index{SettingsChangedListener@\textit{ SettingsChangedListener}}Interface SettingsChangedListener}{
\rule[1em]{\hsize}{4pt}\vskip -1em
\vskip .1in 
This Listener listens for changes in the settings\vskip .1in 
\subsection{Declaration}{
\begin{lstlisting}[frame=none]
public interface SettingsChangedListener
\end{lstlisting}
\subsection{Methods}{
\rule[1em]{\hsize}{2pt}\vskip -2em
\vskip -2em
\begin{itemize}
\item{ 
\index{onSettingsChanged()}
{\bf  onSettingsChanged}\\
\begin{lstlisting}[frame=none]
void onSettingsChanged()\end{lstlisting} %end signature
\begin{itemize}
\item{
{\bf  Description}

Alerts the Controller that the settings have been changed.
}
\end{itemize}
}%end item
\end{itemize}
}
}
\section{\label{edu.kit.pse.osip.simulation.controller.SimulationControlWindow}\index{SimulationControlWindow@\textit{ SimulationControlWindow}}Interface SimulationControlWindow}{
\rule[1em]{\hsize}{4pt}\vskip -1em
\vskip .1in 
Provides abstraction from the view\vskip .1in 
\subsection{Declaration}{
\begin{lstlisting}[frame=none]
public interface SimulationControlWindow
\end{lstlisting}
\subsection{All known subinterfaces}{SimulationControlWindow\small{\refdefined{edu.kit.pse.osip.simulation.view.control.SimulationControlWindow}}}
\subsection{All classes known to implement interface}{SimulationControlWindow\small{\refdefined{edu.kit.pse.osip.simulation.view.control.SimulationControlWindow}}}
\subsection{Methods}{
\rule[1em]{\hsize}{2pt}\vskip -2em
\vskip -2em
\begin{itemize}
\item{ 
\index{getInFlow(TankSelector)}
{\bf  getInFlow}\\
\begin{lstlisting}[frame=none]
int getInFlow(edu.kit.pse.osip.core.model.base.TankSelector tank)\end{lstlisting} %end signature
\begin{itemize}
\item{
{\bf  Description}

Gets the value of inFlow of tank.
}
\item{
{\bf  Parameters}
  \begin{itemize}
   \item{
\texttt{tank} -- The tank to get the inflow from}
  \end{itemize}
}%end item
\item{{\bf  Returns} -- 
The value of inFlow of tank 
}%end item
\end{itemize}
}%end item
\divideents{getMotorSpeed}
\item{ 
\index{getMotorSpeed()}
{\bf  getMotorSpeed}\\
\begin{lstlisting}[frame=none]
int getMotorSpeed()\end{lstlisting} %end signature
\begin{itemize}
\item{
{\bf  Description}

Returns the motor speed of the mixtank
}
\item{{\bf  Returns} -- 
the motor speed of the mixtank 
}%end item
\end{itemize}
}%end item
\divideents{getOutFlow}
\item{ 
\index{getOutFlow(TankSelector)}
{\bf  getOutFlow}\\
\begin{lstlisting}[frame=none]
int getOutFlow(edu.kit.pse.osip.core.model.base.TankSelector tank)\end{lstlisting} %end signature
\begin{itemize}
\item{
{\bf  Description}

Gets the value of outFlow of the tank.
}
\item{
{\bf  Parameters}
  \begin{itemize}
   \item{
\texttt{tank} -- The tank to get the outflow from}
  \end{itemize}
}%end item
\item{{\bf  Returns} -- 
The value of outFlow of the tank 
}%end item
\end{itemize}
}%end item
\divideents{getTemperature}
\item{ 
\index{getTemperature(TankSelector)}
{\bf  getTemperature}\\
\begin{lstlisting}[frame=none]
int getTemperature(edu.kit.pse.osip.core.model.base.TankSelector tank)\end{lstlisting} %end signature
\begin{itemize}
\item{
{\bf  Description}

Gets the value of temp of the tank
}
\item{
{\bf  Parameters}
  \begin{itemize}
   \item{
\texttt{tank} -- The tank to get the temperature from}
  \end{itemize}
}%end item
\item{{\bf  Returns} -- 
The value of temp the tank. 
}%end item
\end{itemize}
}%end item
\divideents{setFlowRateListener}
\item{ 
\index{setFlowRateListener(FlowRateListener)}
{\bf  setFlowRateListener}\\
\begin{lstlisting}[frame=none]
void setFlowRateListener(FlowRateListener listener)\end{lstlisting} %end signature
\begin{itemize}
\item{
{\bf  Description}

Sets the listener that is notified of changes in the flow rate. Listner gets the TankSelector of the actually modified tank
}
\item{
{\bf  Parameters}
  \begin{itemize}
   \item{
\texttt{listener} -- The listener to be called on changes}
  \end{itemize}
}%end item
\end{itemize}
}%end item
\divideents{setMotorListener}
\item{ 
\index{setMotorListener(MotorListener)}
{\bf  setMotorListener}\\
\begin{lstlisting}[frame=none]
void setMotorListener(MotorListener listener)\end{lstlisting} %end signature
\begin{itemize}
\item{
{\bf  Description}

Sets the listener that is notified of changes in motor speed.
}
\item{
{\bf  Parameters}
  \begin{itemize}
   \item{
\texttt{listener} -- The listener to be called on changes}
  \end{itemize}
}%end item
\end{itemize}
}%end item
\divideents{setTemperatureListener}
\item{ 
\index{setTemperatureListener(TemperatureListener)}
{\bf  setTemperatureListener}\\
\begin{lstlisting}[frame=none]
void setTemperatureListener(TemperatureListener listener)\end{lstlisting} %end signature
\begin{itemize}
\item{
{\bf  Description}

Sets the listener that is notified of changes in the temperature. Listner gets the TankSelector of the actually modified tank
}
\item{
{\bf  Parameters}
  \begin{itemize}
   \item{
\texttt{listener} -- The listener to be called on changes}
  \end{itemize}
}%end item
\end{itemize}
}%end item
\end{itemize}
}
}
\section{\label{edu.kit.pse.osip.simulation.controller.SimulationSettingsWindow}\index{SimulationSettingsWindow@\textit{ SimulationSettingsWindow}}Interface SimulationSettingsWindow}{
\rule[1em]{\hsize}{4pt}\vskip -1em
\vskip .1in 
Provides abstraction from the view\vskip .1in 
\subsection{Declaration}{
\begin{lstlisting}[frame=none]
public interface SimulationSettingsWindow
\end{lstlisting}
\subsection{All known subinterfaces}{SimulationSettingsWindow\small{\refdefined{edu.kit.pse.osip.simulation.view.settings.SimulationSettingsWindow}}}
\subsection{All classes known to implement interface}{SimulationSettingsWindow\small{\refdefined{edu.kit.pse.osip.simulation.view.settings.SimulationSettingsWindow}}}
\subsection{Methods}{
\rule[1em]{\hsize}{2pt}\vskip -2em
\vskip -2em
\begin{itemize}
\item{ 
\index{getJitter()}
{\bf  getJitter}\\
\begin{lstlisting}[frame=none]
int getJitter()\end{lstlisting} %end signature
\begin{itemize}
\item{
{\bf  Description}

Gets he current value of the jitter-scrollbar.
}
\item{{\bf  Returns} -- 
The current value of the jitter-scrollbar. 
}%end item
\end{itemize}
}%end item
\divideents{getPort}
\item{ 
\index{getPort(TankSelector)}
{\bf  getPort}\\
\begin{lstlisting}[frame=none]
int getPort(edu.kit.pse.osip.core.model.base.TankSelector tank)\end{lstlisting} %end signature
\begin{itemize}
\item{
{\bf  Description}

Gets the port number in Porttextfield id.
}
\item{
{\bf  Parameters}
  \begin{itemize}
   \item{
\texttt{tank} -- }
  \end{itemize}
}%end item
\item{{\bf  Returns} -- 
The port number in Porttextfield id. 
}%end item
\item{{\bf  Throws}
  \begin{itemize}
   \item{\vskip -.6ex \texttt{edu.kit.pse.osip.core.utils.formatting.InvalidPortException} -- Thrown if the current value in port is not valid (see FormatChecker.parsePort(String port).}
  \end{itemize}
}%end item
\end{itemize}
}%end item
\divideents{setSettingsChangedListener}
\item{ 
\index{setSettingsChangedListener(SettingsChangedListener)}
{\bf  setSettingsChangedListener}\\
\begin{lstlisting}[frame=none]
void setSettingsChangedListener(SettingsChangedListener listener)\end{lstlisting} %end signature
\begin{itemize}
\item{
{\bf  Description}

Sets the listener that gets notified as soon as the settings change
}
\item{
{\bf  Parameters}
  \begin{itemize}
   \item{
\texttt{listener} -- }
  \end{itemize}
}%end item
\end{itemize}
}%end item
\end{itemize}
}
}
\section{\label{edu.kit.pse.osip.simulation.controller.SimulationViewInterface}\index{SimulationViewInterface@\textit{ SimulationViewInterface}}Interface SimulationViewInterface}{
\rule[1em]{\hsize}{4pt}\vskip -1em
\vskip .1in 
Provides abstraction from the view\vskip .1in 
\subsection{Declaration}{
\begin{lstlisting}[frame=none]
public interface SimulationViewInterface
\end{lstlisting}
\subsection{All known subinterfaces}{SimulationMainWindow\small{\refdefined{edu.kit.pse.osip.simulation.view.main.SimulationMainWindow}}}
\subsection{All classes known to implement interface}{SimulationMainWindow\small{\refdefined{edu.kit.pse.osip.simulation.view.main.SimulationMainWindow}}}
\subsection{Methods}{
\rule[1em]{\hsize}{2pt}\vskip -2em
\vskip -2em
\begin{itemize}
\item{ 
\index{showOverflow()}
{\bf  showOverflow}\\
\begin{lstlisting}[frame=none]
void showOverflow()\end{lstlisting} %end signature
\begin{itemize}
\item{
{\bf  Description}

The simulation is replaced by the OverflowOverlay.
}
\end{itemize}
}%end item
\divideents{start}
\item{ 
\index{start(javafx.stage.Stage)}
{\bf  start}\\
\begin{lstlisting}[frame=none]
void start(javafx.stage.Stage primaryStage)\end{lstlisting} %end signature
\begin{itemize}
\item{
{\bf  Description}

Draw the simulation view to the stage
}
\item{
{\bf  Parameters}
  \begin{itemize}
   \item{
\texttt{primaryStage} -- The stage that is provided by JavaFx}
  \end{itemize}
}%end item
\end{itemize}
}%end item
\end{itemize}
}
}
\section{\label{edu.kit.pse.osip.simulation.controller.TemperatureListener}\index{TemperatureListener@\textit{ TemperatureListener}}Interface TemperatureListener}{
\rule[1em]{\hsize}{4pt}\vskip -1em
\vskip .1in 
This Listener listens for changes in the temperature slider.\vskip .1in 
\subsection{Declaration}{
\begin{lstlisting}[frame=none]
public interface TemperatureListener
\end{lstlisting}
\subsection{Methods}{
\rule[1em]{\hsize}{2pt}\vskip -2em
\vskip -2em
\begin{itemize}
\item{ 
\index{onTempUpdated(TankSelector)}
{\bf  onTempUpdated}\\
\begin{lstlisting}[frame=none]
void onTempUpdated(edu.kit.pse.osip.core.model.base.TankSelector tank)\end{lstlisting} %end signature
\begin{itemize}
\item{
{\bf  Description}

Alerts the Controller that the temperature changed.
}
\item{
{\bf  Parameters}
  \begin{itemize}
   \item{
\texttt{tank} -- The tank that was updated}
  \end{itemize}
}%end item
\end{itemize}
}%end item
\end{itemize}
}
}
\section{\label{edu.kit.pse.osip.simulation.controller.AbstractTankServer}\index{AbstractTankServer}Class AbstractTankServer}{
\rule[1em]{\hsize}{4pt}\vskip -1em
\vskip .1in 
Allows setting the values inside the OPC UA server without having to think about NodeIds, namespaces or data types. Provides concrete methods for the variables used in our simulation.\vskip .1in 
\subsection{Declaration}{
\begin{lstlisting}[frame=none]
public abstract class AbstractTankServer
 extends edu.kit.pse.osip.core.opcua.server.UAServerWrapper\end{lstlisting}
\subsection{All known subclasses}{TankServer\small{\refdefined{edu.kit.pse.osip.simulation.controller.TankServer}}, MixTankServer\small{\refdefined{edu.kit.pse.osip.simulation.controller.MixTankServer}}}
\subsection{Constructors}{
\rule[1em]{\hsize}{2pt}\vskip -2em
\vskip -2em
\begin{itemize}
\item{ 
\index{AbstractTankServer(String, int)}
{\bf  AbstractTankServer}\\
\begin{lstlisting}[frame=none]
public AbstractTankServer(java.lang.String namespaceName,int port)\end{lstlisting} %end signature
\begin{itemize}
\item{
{\bf  Description}

Represents a tank inside an OPC UA server
}
\item{
{\bf  Parameters}
  \begin{itemize}
   \item{
\texttt{namespaceName} -- The name of the namespace}
   \item{
\texttt{port} -- The port on which the server should be available}
  \end{itemize}
}%end item
\end{itemize}
}%end item
\end{itemize}
}
\subsection{Methods}{
\rule[1em]{\hsize}{2pt}\vskip -2em
\vskip -2em
\begin{itemize}
\item{ 
\index{setColor(int)}
{\bf  setColor}\\
\begin{lstlisting}[frame=none]
public final void setColor(int color)\end{lstlisting} %end signature
\begin{itemize}
\item{
{\bf  Description}

Set the color of the liquid in this tank
}
\item{
{\bf  Parameters}
  \begin{itemize}
   \item{
\texttt{color} -- The color of the tank infill}
  \end{itemize}
}%end item
\end{itemize}
}%end item
\divideents{setFillLevel}
\item{ 
\index{setFillLevel(float)}
{\bf  setFillLevel}\\
\begin{lstlisting}[frame=none]
public final void setFillLevel(float filled)\end{lstlisting} %end signature
\begin{itemize}
\item{
{\bf  Description}

Set the fill level of the tank
}
\item{
{\bf  Parameters}
  \begin{itemize}
   \item{
\texttt{filled} -- The fill level of the tank}
  \end{itemize}
}%end item
\end{itemize}
}%end item
\divideents{setOutputFlowRate}
\item{ 
\index{setOutputFlowRate(float)}
{\bf  setOutputFlowRate}\\
\begin{lstlisting}[frame=none]
public final void setOutputFlowRate(float rate)\end{lstlisting} %end signature
\begin{itemize}
\item{
{\bf  Description}

Set the flow rate of the outgoing valve
}
\item{
{\bf  Parameters}
  \begin{itemize}
   \item{
\texttt{rate} -- The flow rate of the outgoing valve}
  \end{itemize}
}%end item
\end{itemize}
}%end item
\divideents{setOverflowAlarm}
\item{ 
\index{setOverflowAlarm(boolean)}
{\bf  setOverflowAlarm}\\
\begin{lstlisting}[frame=none]
public final void setOverflowAlarm(boolean overflow)\end{lstlisting} %end signature
\begin{itemize}
\item{
{\bf  Description}

Sets the overflow alarm state
}
\item{
{\bf  Parameters}
  \begin{itemize}
   \item{
\texttt{overflow} -- true if the tank is overflowing}
  \end{itemize}
}%end item
\end{itemize}
}%end item
\divideents{setOverheatAlarm}
\item{ 
\index{setOverheatAlarm(boolean)}
{\bf  setOverheatAlarm}\\
\begin{lstlisting}[frame=none]
public final void setOverheatAlarm(boolean overheat)\end{lstlisting} %end signature
\begin{itemize}
\item{
{\bf  Description}

Sets the overheat alarm state
}
\item{
{\bf  Parameters}
  \begin{itemize}
   \item{
\texttt{overheat} -- true if the tank is overheating}
  \end{itemize}
}%end item
\end{itemize}
}%end item
\divideents{setTemperature}
\item{ 
\index{setTemperature(float)}
{\bf  setTemperature}\\
\begin{lstlisting}[frame=none]
public final void setTemperature(float temperature)\end{lstlisting} %end signature
\begin{itemize}
\item{
{\bf  Description}

Set temperature of the contained liquid
}
\item{
{\bf  Parameters}
  \begin{itemize}
   \item{
\texttt{temperature} -- Temperature of the tank content}
  \end{itemize}
}%end item
\end{itemize}
}%end item
\divideents{setUndercoolAlarm}
\item{ 
\index{setUndercoolAlarm(boolean)}
{\bf  setUndercoolAlarm}\\
\begin{lstlisting}[frame=none]
public final void setUndercoolAlarm(boolean undercool)\end{lstlisting} %end signature
\begin{itemize}
\item{
{\bf  Description}

Sets the undercool alarm state
}
\item{
{\bf  Parameters}
  \begin{itemize}
   \item{
\texttt{undercool} -- true if the tank is undercooling}
  \end{itemize}
}%end item
\end{itemize}
}%end item
\divideents{setUnderflowAlarm}
\item{ 
\index{setUnderflowAlarm(boolean)}
{\bf  setUnderflowAlarm}\\
\begin{lstlisting}[frame=none]
public final void setUnderflowAlarm(boolean underflow)\end{lstlisting} %end signature
\begin{itemize}
\item{
{\bf  Description}

Sets the underflow alarm state
}
\item{
{\bf  Parameters}
  \begin{itemize}
   \item{
\texttt{underflow} -- true if the tank is underflowing}
  \end{itemize}
}%end item
\end{itemize}
}%end item
\end{itemize}
}
}
\section{\label{edu.kit.pse.osip.simulation.controller.MainClass}\index{MainClass}Class MainClass}{
\rule[1em]{\hsize}{4pt}\vskip -1em
\vskip .1in 
Main entry point for the whole simulation\vskip .1in 
\subsection{Declaration}{
\begin{lstlisting}[frame=none]
public class MainClass
 extends java.lang.Object\end{lstlisting}
\subsection{Methods}{
\rule[1em]{\hsize}{2pt}\vskip -2em
\vskip -2em
\begin{itemize}
\item{ 
\index{main(String\lbrack \rbrack )}
{\bf  main}\\
\begin{lstlisting}[frame=none]
public static final void main(java.lang.String[] args)\end{lstlisting} %end signature
\begin{itemize}
\item{
{\bf  Description}

Main entry point into the jar
}
\item{
{\bf  Parameters}
  \begin{itemize}
   \item{
\texttt{args} -- Command line arguments}
  \end{itemize}
}%end item
\end{itemize}
}%end item
\end{itemize}
}
}
\section{\label{edu.kit.pse.osip.simulation.controller.MenuAboutButtonHandler}\index{MenuAboutButtonHandler}Class MenuAboutButtonHandler}{
\rule[1em]{\hsize}{4pt}\vskip -1em
\vskip .1in 
Handles a click on the about menu button in the simulation view.\vskip .1in 
\subsection{Declaration}{
\begin{lstlisting}[frame=none]
public class MenuAboutButtonHandler
 extends java.lang.Object\end{lstlisting}
\subsection{Constructors}{
\rule[1em]{\hsize}{2pt}\vskip -2em
\vskip -2em
\begin{itemize}
\item{ 
\index{MenuAboutButtonHandler()}
{\bf  MenuAboutButtonHandler}\\
\begin{lstlisting}[frame=none]
public MenuAboutButtonHandler()\end{lstlisting} %end signature
}%end item
\end{itemize}
}
\subsection{Methods}{
\rule[1em]{\hsize}{2pt}\vskip -2em
\vskip -2em
\begin{itemize}
\item{ 
\index{handle(javafx.event.ActionEvent)}
{\bf  handle}\\
\begin{lstlisting}[frame=none]
public final void handle(javafx.event.ActionEvent event)\end{lstlisting} %end signature
\begin{itemize}
\item{
{\bf  Description}

Handles a click on the about menu button in the simulation view.
}
\item{
{\bf  Parameters}
  \begin{itemize}
   \item{
\texttt{event} -- The occured event.}
  \end{itemize}
}%end item
\end{itemize}
}%end item
\end{itemize}
}
}
\section{\label{edu.kit.pse.osip.simulation.controller.MenuControlButtonHandler}\index{MenuControlButtonHandler}Class MenuControlButtonHandler}{
\rule[1em]{\hsize}{4pt}\vskip -1em
\vskip .1in 
Handles a click on the control menu button\vskip .1in 
\subsection{Declaration}{
\begin{lstlisting}[frame=none]
public class MenuControlButtonHandler
 extends java.lang.Object\end{lstlisting}
\subsection{Constructors}{
\rule[1em]{\hsize}{2pt}\vskip -2em
\vskip -2em
\begin{itemize}
\item{ 
\index{MenuControlButtonHandler(SimulationControlWindow)}
{\bf  MenuControlButtonHandler}\\
\begin{lstlisting}[frame=none]
protected MenuControlButtonHandler(edu.kit.pse.osip.simulation.view.control.SimulationControlWindow controlWindow)\end{lstlisting} %end signature
\begin{itemize}
\item{
{\bf  Description}

Creates a new handler.
}
\item{
{\bf  Parameters}
  \begin{itemize}
   \item{
\texttt{controlWindow} -- The current control window}
  \end{itemize}
}%end item
\end{itemize}
}%end item
\end{itemize}
}
\subsection{Methods}{
\rule[1em]{\hsize}{2pt}\vskip -2em
\vskip -2em
\begin{itemize}
\item{ 
\index{handle(javafx.event.ActionEvent)}
{\bf  handle}\\
\begin{lstlisting}[frame=none]
public final void handle(javafx.event.ActionEvent event)\end{lstlisting} %end signature
\begin{itemize}
\item{
{\bf  Description}

Handles a click on the settings menu button in the monitoring view.
}
\item{
{\bf  Parameters}
  \begin{itemize}
   \item{
\texttt{event} -- The occured event.}
  \end{itemize}
}%end item
\end{itemize}
}%end item
\end{itemize}
}
}
\section{\label{edu.kit.pse.osip.simulation.controller.MenuHelpButtonHandler}\index{MenuHelpButtonHandler}Class MenuHelpButtonHandler}{
\rule[1em]{\hsize}{4pt}\vskip -1em
\vskip .1in 
Handles a click on the help menu button in the simulation view.\vskip .1in 
\subsection{Declaration}{
\begin{lstlisting}[frame=none]
public class MenuHelpButtonHandler
 extends java.lang.Object\end{lstlisting}
\subsection{Constructors}{
\rule[1em]{\hsize}{2pt}\vskip -2em
\vskip -2em
\begin{itemize}
\item{ 
\index{MenuHelpButtonHandler()}
{\bf  MenuHelpButtonHandler}\\
\begin{lstlisting}[frame=none]
public MenuHelpButtonHandler()\end{lstlisting} %end signature
}%end item
\end{itemize}
}
\subsection{Methods}{
\rule[1em]{\hsize}{2pt}\vskip -2em
\vskip -2em
\begin{itemize}
\item{ 
\index{handle(javafx.event.ActionEvent)}
{\bf  handle}\\
\begin{lstlisting}[frame=none]
public final void handle(javafx.event.ActionEvent event)\end{lstlisting} %end signature
\begin{itemize}
\item{
{\bf  Description}

Handles a click on the help menu button in the simulation view.
}
\item{
{\bf  Parameters}
  \begin{itemize}
   \item{
\texttt{event} -- The ocurred event.}
  \end{itemize}
}%end item
\end{itemize}
}%end item
\end{itemize}
}
}
\section{\label{edu.kit.pse.osip.simulation.controller.MenuSettingsButtonHandler}\index{MenuSettingsButtonHandler}Class MenuSettingsButtonHandler}{
\rule[1em]{\hsize}{4pt}\vskip -1em
\vskip .1in 
Handles a click on the settings menu button\vskip .1in 
\subsection{Declaration}{
\begin{lstlisting}[frame=none]
public class MenuSettingsButtonHandler
 extends java.lang.Object\end{lstlisting}
\subsection{Constructors}{
\rule[1em]{\hsize}{2pt}\vskip -2em
\vskip -2em
\begin{itemize}
\item{ 
\index{MenuSettingsButtonHandler(SimulationSettingsWindow)}
{\bf  MenuSettingsButtonHandler}\\
\begin{lstlisting}[frame=none]
protected MenuSettingsButtonHandler(edu.kit.pse.osip.simulation.view.settings.SimulationSettingsWindow settingsWindow)\end{lstlisting} %end signature
\begin{itemize}
\item{
{\bf  Description}

Creates a new handler.
}
\item{
{\bf  Parameters}
  \begin{itemize}
   \item{
\texttt{settingsWindow} -- The current control window}
  \end{itemize}
}%end item
\end{itemize}
}%end item
\end{itemize}
}
\subsection{Methods}{
\rule[1em]{\hsize}{2pt}\vskip -2em
\vskip -2em
\begin{itemize}
\item{ 
\index{handle(javafx.event.ActionEvent)}
{\bf  handle}\\
\begin{lstlisting}[frame=none]
public final void handle(javafx.event.ActionEvent event)\end{lstlisting} %end signature
\begin{itemize}
\item{
{\bf  Description}

Handles a click on the settings menu button in the monitoring view.
}
\item{
{\bf  Parameters}
  \begin{itemize}
   \item{
\texttt{event} -- The occured event.}
  \end{itemize}
}%end item
\end{itemize}
}%end item
\end{itemize}
}
}
\section{\label{edu.kit.pse.osip.simulation.controller.MixTankServer}\index{MixTankServer}Class MixTankServer}{
\rule[1em]{\hsize}{4pt}\vskip -1em
\vskip .1in 
Server for the mixtank. Contain the motor, in addition to the variables provided in the parent class.\vskip .1in 
\subsection{Declaration}{
\begin{lstlisting}[frame=none]
public class MixTankServer
 extends edu.kit.pse.osip.simulation.controller.AbstractTankServer\end{lstlisting}
\subsection{Constructors}{
\rule[1em]{\hsize}{2pt}\vskip -2em
\vskip -2em
\begin{itemize}
\item{ 
\index{MixTankServer(int)}
{\bf  MixTankServer}\\
\begin{lstlisting}[frame=none]
public MixTankServer(int port)\end{lstlisting} %end signature
\begin{itemize}
\item{
{\bf  Description}

Creates a new server for a mixtank
}
\item{
{\bf  Parameters}
  \begin{itemize}
   \item{
\texttt{port} -- The port to start the server on}
  \end{itemize}
}%end item
\end{itemize}
}%end item
\end{itemize}
}
\subsection{Methods}{
\rule[1em]{\hsize}{2pt}\vskip -2em
\vskip -2em
\begin{itemize}
\item{ 
\index{setMotorSpeed(int)}
{\bf  setMotorSpeed}\\
\begin{lstlisting}[frame=none]
public final void setMotorSpeed(int speed)\end{lstlisting} %end signature
\begin{itemize}
\item{
{\bf  Description}

Sets the speed of the motor in rpm
}
\item{
{\bf  Parameters}
  \begin{itemize}
   \item{
\texttt{speed} -- The speed in rpm}
  \end{itemize}
}%end item
\end{itemize}
}%end item
\end{itemize}
}
}
\section{\label{edu.kit.pse.osip.simulation.controller.PhysicsSimulator}\index{PhysicsSimulator}Class PhysicsSimulator}{
\rule[1em]{\hsize}{4pt}\vskip -1em
\vskip .1in 
Simulator for a given plant.\vskip .1in 
\subsection{Declaration}{
\begin{lstlisting}[frame=none]
public class PhysicsSimulator
 extends java.lang.Object\end{lstlisting}
\subsection{Fields}{
\rule[1em]{\hsize}{2pt}
\begin{itemize}
\item{
\index{productionSite}
\label{edu.kit.pse.osip.simulation.controller.PhysicsSimulator.productionSite}\texttt{public edu.kit.pse.osip.core.model.simulation.ProductionSiteSimulation\ {\bf  productionSite}}
}
\end{itemize}
}
\subsection{Constructors}{
\rule[1em]{\hsize}{2pt}\vskip -2em
\vskip -2em
\begin{itemize}
\item{ 
\index{PhysicsSimulator(ProductionSiteSimulation)}
{\bf  PhysicsSimulator}\\
\begin{lstlisting}[frame=none]
public PhysicsSimulator(edu.kit.pse.osip.core.model.simulation.ProductionSiteSimulation productionSite)\end{lstlisting} %end signature
\begin{itemize}
\item{
{\bf  Description}

Initializes a new simulator for the given plant
}
\item{
{\bf  Parameters}
  \begin{itemize}
   \item{
\texttt{productionSite} -- The production site to simulate the values on}
  \end{itemize}
}%end item
\end{itemize}
}%end item
\end{itemize}
}
\subsection{Methods}{
\rule[1em]{\hsize}{2pt}\vskip -2em
\vskip -2em
\begin{itemize}
\item{ 
\index{setInputTemperature(TankSelector, float)}
{\bf  setInputTemperature}\\
\begin{lstlisting}[frame=none]
public final void setInputTemperature(edu.kit.pse.osip.core.model.base.TankSelector tank,float temperature)\end{lstlisting} %end signature
\begin{itemize}
\item{
{\bf  Description}

Set the temperature of the liquid which gets into the upper tanks
}
\item{
{\bf  Parameters}
  \begin{itemize}
   \item{
\texttt{tank} -- The tank to set the temperature}
   \item{
\texttt{temperature} -- The temperature}
  \end{itemize}
}%end item
\end{itemize}
}%end item
\divideents{tick}
\item{ 
\index{tick()}
{\bf  tick}\\
\begin{lstlisting}[frame=none]
public final void tick()\end{lstlisting} %end signature
\begin{itemize}
\item{
{\bf  Description}

Executes one simulation step. Needs to be called regularly.
}
\end{itemize}
}%end item
\end{itemize}
}
}
\section{\label{edu.kit.pse.osip.simulation.controller.SimulationController}\index{SimulationController}Class SimulationController}{
\rule[1em]{\hsize}{4pt}\vskip -1em
\vskip .1in 
Manages servers and controls view actions.\vskip .1in 
\subsection{Declaration}{
\begin{lstlisting}[frame=none]
public class SimulationController
 extends javafx.application.Application implements java.util.Observer\end{lstlisting}
\subsection{Fields}{
\rule[1em]{\hsize}{2pt}
\begin{itemize}
\item{
\index{currentSimulationView}
\label{edu.kit.pse.osip.simulation.controller.SimulationController.currentSimulationView}\texttt{public SimulationViewInterface\ {\bf  currentSimulationView}}
}
\item{
\index{productionSite}
\label{edu.kit.pse.osip.simulation.controller.SimulationController.productionSite}\texttt{public edu.kit.pse.osip.core.model.simulation.ProductionSiteSimulation\ {\bf  productionSite}}
}
\item{
\index{simulator}
\label{edu.kit.pse.osip.simulation.controller.SimulationController.simulator}\texttt{public PhysicsSimulator\ {\bf  simulator}}
}
\item{
\index{inputServer}
\label{edu.kit.pse.osip.simulation.controller.SimulationController.inputServer}\texttt{public TankServer\ {\bf  inputServer}}
}
\item{
\index{mixServer}
\label{edu.kit.pse.osip.simulation.controller.SimulationController.mixServer}\texttt{public MixTankServer\ {\bf  mixServer}}
}
\end{itemize}
}
\subsection{Constructors}{
\rule[1em]{\hsize}{2pt}\vskip -2em
\vskip -2em
\begin{itemize}
\item{ 
\index{SimulationController()}
{\bf  SimulationController}\\
\begin{lstlisting}[frame=none]
public SimulationController()\end{lstlisting} %end signature
\begin{itemize}
\item{
{\bf  Description}

Responsible for controlling the display windows and simulating the production
}
\end{itemize}
}%end item
\end{itemize}
}
\subsection{Methods}{
\rule[1em]{\hsize}{2pt}\vskip -2em
\vskip -2em
\begin{itemize}
\item{ 
\index{start(javafx.stage.Stage)}
{\bf  start}\\
\begin{lstlisting}[frame=none]
public final void start(javafx.stage.Stage primaryStage)\end{lstlisting} %end signature
\begin{itemize}
\item{
{\bf  Description}

Called bx JavaFx to start drawing the UI
}
\item{
{\bf  Parameters}
  \begin{itemize}
   \item{
\texttt{primaryStage} -- The stage to draw the main window on}
  \end{itemize}
}%end item
\end{itemize}
}%end item
\divideents{startMainLoop}
\item{ 
\index{startMainLoop()}
{\bf  startMainLoop}\\
\begin{lstlisting}[frame=none]
public final void startMainLoop()\end{lstlisting} %end signature
\begin{itemize}
\item{
{\bf  Description}

Show windows and start loop that updates the values
}
\end{itemize}
}%end item
\divideents{stop}
\item{ 
\index{stop()}
{\bf  stop}\\
\begin{lstlisting}[frame=none]
public final void stop()\end{lstlisting} %end signature
\begin{itemize}
\item{
{\bf  Description}

Called when the last window is closed
}
\end{itemize}
}%end item
\divideents{update}
\item{ 
\index{update(Observable, Object)}
{\bf  update}\\
\begin{lstlisting}[frame=none]
public final void update(java.util.Observable observable,java.lang.Object object)\end{lstlisting} %end signature
\begin{itemize}
\item{
{\bf  Description}

One of the observed objects changed its state
}
\item{
{\bf  Parameters}
  \begin{itemize}
   \item{
\texttt{observable} -- Observed object}
   \item{
\texttt{object} -- New value}
  \end{itemize}
}%end item
\end{itemize}
}%end item
\end{itemize}
}
}
\section{\label{edu.kit.pse.osip.simulation.controller.TankServer}\index{TankServer}Class TankServer}{
\rule[1em]{\hsize}{4pt}\vskip -1em
\vskip .1in 
Server for the upper tanks. They contain the input valve, in addition to the variables provided in the parent class.\vskip .1in 
\subsection{Declaration}{
\begin{lstlisting}[frame=none]
public class TankServer
 extends edu.kit.pse.osip.simulation.controller.AbstractTankServer\end{lstlisting}
\subsection{Constructors}{
\rule[1em]{\hsize}{2pt}\vskip -2em
\vskip -2em
\begin{itemize}
\item{ 
\index{TankServer(int)}
{\bf  TankServer}\\
\begin{lstlisting}[frame=none]
public TankServer(int port)\end{lstlisting} %end signature
\begin{itemize}
\item{
{\bf  Description}

Creates a new server for a tank
}
\item{
{\bf  Parameters}
  \begin{itemize}
   \item{
\texttt{port} -- The port to add the server}
  \end{itemize}
}%end item
\end{itemize}
}%end item
\end{itemize}
}
\subsection{Methods}{
\rule[1em]{\hsize}{2pt}\vskip -2em
\vskip -2em
\begin{itemize}
\item{ 
\index{setInputFlowRate(float)}
{\bf  setInputFlowRate}\\
\begin{lstlisting}[frame=none]
public final void setInputFlowRate(float flowRate)\end{lstlisting} %end signature
\begin{itemize}
\item{
{\bf  Description}

Sets the flow rate of the incoming valve
}
\item{
{\bf  Parameters}
  \begin{itemize}
   \item{
\texttt{flowRate} -- The flow rate of the input tanks}
  \end{itemize}
}%end item
\end{itemize}
}%end item
\end{itemize}
}
}
}
\chapter{Package edu.kit.pse.osip.simulation.view.control}{
\label{edu.kit.pse.osip.simulation.view.control}\section{\label{edu.kit.pse.osip.simulation.view.control.AbstractTankTab}\index{AbstractTankTab}Class AbstractTankTab}{
\rule[1em]{\hsize}{4pt}\vskip -1em
\vskip .1in 
This class contains the controls for a single tank in the simulation.\vskip .1in 
\subsection{Declaration}{
\begin{lstlisting}[frame=none]
public abstract class AbstractTankTab
 extends javafx.scene.control.TabPane\end{lstlisting}
\subsection{All known subclasses}{MixTankTab\small{\refdefined{edu.kit.pse.osip.simulation.view.control.MixTankTab}}, TankTab\small{\refdefined{edu.kit.pse.osip.simulation.view.control.TankTab}}}
\subsection{Constructors}{
\rule[1em]{\hsize}{2pt}\vskip -2em
\vskip -2em
\begin{itemize}
\item{ 
\index{AbstractTankTab()}
{\bf  AbstractTankTab}\\
\begin{lstlisting}[frame=none]
public AbstractTankTab()\end{lstlisting} %end signature
}%end item
\end{itemize}
}
\subsection{Methods}{
\rule[1em]{\hsize}{2pt}\vskip -2em
\vskip -2em
\begin{itemize}
\item{ 
\index{getOutFlow()}
{\bf  getOutFlow}\\
\begin{lstlisting}[frame=none]
public final int getOutFlow()\end{lstlisting} %end signature
\begin{itemize}
\item{
{\bf  Description}

Gets the value of outFlow.
}
\item{{\bf  Returns} -- 
The value of outFlow. 
}%end item
\end{itemize}
}%end item
\end{itemize}
}
}
\section{\label{edu.kit.pse.osip.simulation.view.control.MixTankTab}\index{MixTankTab}Class MixTankTab}{
\rule[1em]{\hsize}{4pt}\vskip -1em
\vskip .1in 
This class has controls specific to the tanks in which several inputs are mixed.\vskip .1in 
\subsection{Declaration}{
\begin{lstlisting}[frame=none]
public class MixTankTab
 extends edu.kit.pse.osip.simulation.view.control.AbstractTankTab\end{lstlisting}
\subsection{Constructors}{
\rule[1em]{\hsize}{2pt}\vskip -2em
\vskip -2em
\begin{itemize}
\item{ 
\index{MixTankTab()}
{\bf  MixTankTab}\\
\begin{lstlisting}[frame=none]
public MixTankTab()\end{lstlisting} %end signature
}%end item
\end{itemize}
}
\subsection{Methods}{
\rule[1em]{\hsize}{2pt}\vskip -2em
\vskip -2em
\begin{itemize}
\item{ 
\index{getMotorRPM()}
{\bf  getMotorRPM}\\
\begin{lstlisting}[frame=none]
public final int getMotorRPM()\end{lstlisting} %end signature
\begin{itemize}
\item{
{\bf  Description}

Gets the value of the motor rpm slider
}
\item{{\bf  Returns} -- 
The rpm of the motor in the mixtank 
}%end item
\end{itemize}
}%end item
\end{itemize}
}
}
\section{\label{edu.kit.pse.osip.simulation.view.control.SimulationControlWindow}\index{SimulationControlWindow}Class SimulationControlWindow}{
\rule[1em]{\hsize}{4pt}\vskip -1em
\vskip .1in 
This is the main window for controlling the OSIP simulation. It provides scroll bars in different tabs for all tanks to adjust their values, like outflow, temperature and motor speed.\vskip .1in 
\subsection{Declaration}{
\begin{lstlisting}[frame=none]
public class SimulationControlWindow
 extends javafx.stage.Stage implements edu.kit.pse.osip.simulation.controller.SimulationControlWindow\end{lstlisting}
\subsection{Fields}{
\rule[1em]{\hsize}{2pt}
\begin{itemize}
\item{
\index{mixTankTab}
\label{edu.kit.pse.osip.simulation.view.control.SimulationControlWindow.mixTankTab}\texttt{public MixTankTab\ {\bf  mixTankTab}}
}
\item{
\index{tankTabs}
\label{edu.kit.pse.osip.simulation.view.control.SimulationControlWindow.tankTabs}\texttt{public TankTab\ {\bf  tankTabs}}
}
\item{
\index{temperatureListener}
\label{edu.kit.pse.osip.simulation.view.control.SimulationControlWindow.temperatureListener}\texttt{public edu.kit.pse.osip.simulation.controller.TemperatureListener\ {\bf  temperatureListener}}
\begin{itemize}
\item{\vskip -.9ex 
The listener that is called when the temperature was changed}
\end{itemize}
}
\item{
\index{flowRateListener}
\label{edu.kit.pse.osip.simulation.view.control.SimulationControlWindow.flowRateListener}\texttt{public edu.kit.pse.osip.simulation.controller.FlowRateListener\ {\bf  flowRateListener}}
\begin{itemize}
\item{\vskip -.9ex 
The listener that is called when the flow rate was changed}
\end{itemize}
}
\item{
\index{motorListener}
\label{edu.kit.pse.osip.simulation.view.control.SimulationControlWindow.motorListener}\texttt{public edu.kit.pse.osip.simulation.controller.MotorListener\ {\bf  motorListener}}
\begin{itemize}
\item{\vskip -.9ex 
The listener that is called when the motor speed was changed}
\end{itemize}
}
\end{itemize}
}
\subsection{Constructors}{
\rule[1em]{\hsize}{2pt}\vskip -2em
\vskip -2em
\begin{itemize}
\item{ 
\index{SimulationControlWindow(ProductionSite)}
{\bf  SimulationControlWindow}\\
\begin{lstlisting}[frame=none]
public SimulationControlWindow(edu.kit.pse.osip.core.model.base.ProductionSite productionSite)\end{lstlisting} %end signature
\begin{itemize}
\item{
{\bf  Description}

Constructs a new SImulationConstrolWindow
}
\item{
{\bf  Parameters}
  \begin{itemize}
   \item{
\texttt{productionSite} -- The ProducitonSite, to get all the tanks}
  \end{itemize}
}%end item
\end{itemize}
}%end item
\end{itemize}
}
\subsection{Methods}{
\rule[1em]{\hsize}{2pt}\vskip -2em
\vskip -2em
\begin{itemize}
\item{ 
\index{getInFlow(TankSelector)}
{\bf  getInFlow}\\
\begin{lstlisting}[frame=none]
public final int getInFlow(edu.kit.pse.osip.core.model.base.TankSelector tank)\end{lstlisting} %end signature
\begin{itemize}
\item{
{\bf  Description}

Gets the value of inFlow of tank.
}
\item{
{\bf  Parameters}
  \begin{itemize}
   \item{
\texttt{tank} -- The tank to get the inflow from}
  \end{itemize}
}%end item
\item{{\bf  Returns} -- 
The value of inFlow of tank 
}%end item
\end{itemize}
}%end item
\divideents{getMotorSpeed}
\item{ 
\index{getMotorSpeed()}
{\bf  getMotorSpeed}\\
\begin{lstlisting}[frame=none]
public final int getMotorSpeed()\end{lstlisting} %end signature
\begin{itemize}
\item{
{\bf  Description}

Returns the motor speed of the mixtank
}
\item{{\bf  Returns} -- 
the motor speed of the mixtank 
}%end item
\end{itemize}
}%end item
\divideents{getOutFlow}
\item{ 
\index{getOutFlow(TankSelector)}
{\bf  getOutFlow}\\
\begin{lstlisting}[frame=none]
public final int getOutFlow(edu.kit.pse.osip.core.model.base.TankSelector tank)\end{lstlisting} %end signature
\begin{itemize}
\item{
{\bf  Description}

Gets the value of outFlow of the tank.
}
\item{
{\bf  Parameters}
  \begin{itemize}
   \item{
\texttt{tank} -- The tank to get the outflow from}
  \end{itemize}
}%end item
\item{{\bf  Returns} -- 
The value of outFlow of the tank 
}%end item
\end{itemize}
}%end item
\divideents{getTemperature}
\item{ 
\index{getTemperature(TankSelector)}
{\bf  getTemperature}\\
\begin{lstlisting}[frame=none]
public final int getTemperature(edu.kit.pse.osip.core.model.base.TankSelector tank)\end{lstlisting} %end signature
\begin{itemize}
\item{
{\bf  Description}

Gets the value of temp of the tank
}
\item{
{\bf  Parameters}
  \begin{itemize}
   \item{
\texttt{tank} -- The tank to get the temperature from}
  \end{itemize}
}%end item
\item{{\bf  Returns} -- 
The value of temp the tank. 
}%end item
\end{itemize}
}%end item
\divideents{setFlowRateListener}
\item{ 
\index{setFlowRateListener(FlowRateListener)}
{\bf  setFlowRateListener}\\
\begin{lstlisting}[frame=none]
public final void setFlowRateListener(edu.kit.pse.osip.simulation.controller.FlowRateListener listener)\end{lstlisting} %end signature
\begin{itemize}
\item{
{\bf  Description}

Sets the listener that is notified of changes in the flow rate. Listner gets the TankSelector of the actually modified tank
}
\item{
{\bf  Parameters}
  \begin{itemize}
   \item{
\texttt{listener} -- The listener to be called on changes}
  \end{itemize}
}%end item
\end{itemize}
}%end item
\divideents{setMotorListener}
\item{ 
\index{setMotorListener(MotorListener)}
{\bf  setMotorListener}\\
\begin{lstlisting}[frame=none]
public final void setMotorListener(edu.kit.pse.osip.simulation.controller.MotorListener listener)\end{lstlisting} %end signature
\begin{itemize}
\item{
{\bf  Description}

Sets the listener that is notified of changes in motor speed.
}
\item{
{\bf  Parameters}
  \begin{itemize}
   \item{
\texttt{listener} -- The listener to be called on changes}
  \end{itemize}
}%end item
\end{itemize}
}%end item
\divideents{setTemperatureListener}
\item{ 
\index{setTemperatureListener(TemperatureListener)}
{\bf  setTemperatureListener}\\
\begin{lstlisting}[frame=none]
public final void setTemperatureListener(edu.kit.pse.osip.simulation.controller.TemperatureListener listener)\end{lstlisting} %end signature
\begin{itemize}
\item{
{\bf  Description}

Sets the listener that is notified of changes in the temperature. Listener gets the TankSelector of the actually modified tank
}
\item{
{\bf  Parameters}
  \begin{itemize}
   \item{
\texttt{listener} -- The listener to be called on changes}
  \end{itemize}
}%end item
\end{itemize}
}%end item
\end{itemize}
}
}
\section{\label{edu.kit.pse.osip.simulation.view.control.TankTab}\index{TankTab}Class TankTab}{
\rule[1em]{\hsize}{4pt}\vskip -1em
\vskip .1in 
This class has controls specific to the input tanks.\vskip .1in 
\subsection{Declaration}{
\begin{lstlisting}[frame=none]
public class TankTab
 extends edu.kit.pse.osip.simulation.view.control.AbstractTankTab\end{lstlisting}
\subsection{Constructors}{
\rule[1em]{\hsize}{2pt}\vskip -2em
\vskip -2em
\begin{itemize}
\item{ 
\index{TankTab()}
{\bf  TankTab}\\
\begin{lstlisting}[frame=none]
public TankTab()\end{lstlisting} %end signature
}%end item
\end{itemize}
}
\subsection{Methods}{
\rule[1em]{\hsize}{2pt}\vskip -2em
\vskip -2em
\begin{itemize}
\item{ 
\index{getInFlow()}
{\bf  getInFlow}\\
\begin{lstlisting}[frame=none]
public final int getInFlow()\end{lstlisting} %end signature
\begin{itemize}
\item{
{\bf  Description}

Gets the value of inFlow.
}
\item{{\bf  Returns} -- 
The value of inFlow. 
}%end item
\end{itemize}
}%end item
\divideents{getTemperature}
\item{ 
\index{getTemperature()}
{\bf  getTemperature}\\
\begin{lstlisting}[frame=none]
public final int getTemperature()\end{lstlisting} %end signature
\begin{itemize}
\item{
{\bf  Description}

Gets the value of temperature.
}
\item{{\bf  Returns} -- 
The value of temperature. 
}%end item
\end{itemize}
}%end item
\end{itemize}
}
}
}
\chapter{Package edu.kit.pse.osip.simulation.view.main}{
\label{edu.kit.pse.osip.simulation.view.main}\section{\label{edu.kit.pse.osip.simulation.view.main.Drawer}\index{Drawer@\textit{ Drawer}}Interface Drawer}{
\rule[1em]{\hsize}{4pt}\vskip -1em
\vskip .1in 
The interface from which all other GUI elements originate. It specifies the draw() method with which the element draws itself onto the screen according to its implementation.\vskip .1in 
\subsection{Declaration}{
\begin{lstlisting}[frame=none]
public interface Drawer
\end{lstlisting}
\subsection{All known subinterfaces}{FillSensorDrawer\small{\refdefined{edu.kit.pse.osip.simulation.view.main.FillSensorDrawer}}, RotatingControlDrawer\small{\refdefined{edu.kit.pse.osip.simulation.view.main.RotatingControlDrawer}}, TankDrawer\small{\refdefined{edu.kit.pse.osip.simulation.view.main.TankDrawer}}, PipeDrawer\small{\refdefined{edu.kit.pse.osip.simulation.view.main.PipeDrawer}}, MotorDrawer\small{\refdefined{edu.kit.pse.osip.simulation.view.main.MotorDrawer}}, MixTankDrawer\small{\refdefined{edu.kit.pse.osip.simulation.view.main.MixTankDrawer}}, AbstractTankDrawer\small{\refdefined{edu.kit.pse.osip.simulation.view.main.AbstractTankDrawer}}, ObjectDrawer\small{\refdefined{edu.kit.pse.osip.simulation.view.main.ObjectDrawer}}, TemperatureSensorDrawer\small{\refdefined{edu.kit.pse.osip.simulation.view.main.TemperatureSensorDrawer}}, ValveDrawer\small{\refdefined{edu.kit.pse.osip.simulation.view.main.ValveDrawer}}}
\subsection{All classes known to implement interface}{PipeDrawer\small{\refdefined{edu.kit.pse.osip.simulation.view.main.PipeDrawer}}, ObjectDrawer\small{\refdefined{edu.kit.pse.osip.simulation.view.main.ObjectDrawer}}}
\subsection{Methods}{
\rule[1em]{\hsize}{2pt}\vskip -2em
\vskip -2em
\begin{itemize}
\item{ 
\index{draw(GraphicsContext)}
{\bf  draw}\\
\begin{lstlisting}[frame=none]
void draw(GraphicsContext context)\end{lstlisting} %end signature
\begin{itemize}
\item{
{\bf  Description}

The Drawer draws itself onto the GraphicsContext at its position.
}
\item{
{\bf  Parameters}
  \begin{itemize}
   \item{
\texttt{context} -- The context that the object draws itself onto}
  \end{itemize}
}%end item
\end{itemize}
}%end item
\end{itemize}
}
}
\section{\label{edu.kit.pse.osip.simulation.view.main.AbstractTankDrawer}\index{AbstractTankDrawer}Class AbstractTankDrawer}{
\rule[1em]{\hsize}{4pt}\vskip -1em
\vskip .1in 
This class visualizes a tank holding a colored liquid. It knows its position as well as the color of the content. The changing part of the visualization are the fill level of the tank and, possibly, the color of the liquid.\vskip .1in 
\subsection{Declaration}{
\begin{lstlisting}[frame=none]
public abstract class AbstractTankDrawer
 extends edu.kit.pse.osip.simulation.view.main.ObjectDrawer\end{lstlisting}
\subsection{All known subclasses}{TankDrawer\small{\refdefined{edu.kit.pse.osip.simulation.view.main.TankDrawer}}, MixTankDrawer\small{\refdefined{edu.kit.pse.osip.simulation.view.main.MixTankDrawer}}}
\subsection{Fields}{
\rule[1em]{\hsize}{2pt}
\begin{itemize}
\item{
\index{tank}
\label{edu.kit.pse.osip.simulation.view.main.AbstractTankDrawer.tank}\texttt{public edu.kit.pse.osip.core.model.base.AbstractTank\ {\bf  tank}}
}
\item{
\index{temperatureSensor}
\label{edu.kit.pse.osip.simulation.view.main.AbstractTankDrawer.temperatureSensor}\texttt{public TemperatureSensorDrawer\ {\bf  temperatureSensor}}
}
\item{
\index{fillSensor}
\label{edu.kit.pse.osip.simulation.view.main.AbstractTankDrawer.fillSensor}\texttt{public FillSensorDrawer\ {\bf  fillSensor}}
}
\end{itemize}
}
\subsection{Constructors}{
\rule[1em]{\hsize}{2pt}\vskip -2em
\vskip -2em
\begin{itemize}
\item{ 
\index{AbstractTankDrawer(Point2D, AbstractTank)}
{\bf  AbstractTankDrawer}\\
\begin{lstlisting}[frame=none]
public AbstractTankDrawer(Point2D pos,edu.kit.pse.osip.core.model.base.AbstractTank tank)\end{lstlisting} %end signature
\begin{itemize}
\item{
{\bf  Description}

Sets the position by using super(pos) and sets the tank
}
\item{
{\bf  Parameters}
  \begin{itemize}
   \item{
\texttt{pos} -- The upper left corner of the tank}
   \item{
\texttt{tank} -- The tank to get the attributes from}
  \end{itemize}
}%end item
\end{itemize}
}%end item
\end{itemize}
}
\subsection{Methods}{
\rule[1em]{\hsize}{2pt}\vskip -2em
\vskip -2em
\begin{itemize}
\item{ 
\index{draw(GraphicsContext)}
{\bf  draw}\\
\begin{lstlisting}[frame=none]
public final void draw(GraphicsContext context)\end{lstlisting} %end signature
\begin{itemize}
\item{
{\bf  Description}

Contains the main calls necessary to draw the tank. Uses the abstract method drawSensors() for detail.
}
\item{
{\bf  Parameters}
  \begin{itemize}
   \item{
\texttt{context} -- }
  \end{itemize}
}%end item
\end{itemize}
}%end item
\divideents{drawSensors}
\item{ 
\index{drawSensors(GraphicsContext)}
{\bf  drawSensors}\\
\begin{lstlisting}[frame=none]
public abstract void drawSensors(GraphicsContext context)\end{lstlisting} %end signature
\begin{itemize}
\item{
{\bf  Description}

Used by draw(). Adds some detail to the tank depending on tank type.
}
\item{
{\bf  Parameters}
  \begin{itemize}
   \item{
\texttt{context} -- }
  \end{itemize}
}%end item
\end{itemize}
}%end item
\divideents{getHeight}
\item{ 
\index{getHeight()}
{\bf  getHeight}\\
\begin{lstlisting}[frame=none]
public final double getHeight()\end{lstlisting} %end signature
\begin{itemize}
\item{
{\bf  Description}

Gets the height of this tank
}
\item{{\bf  Returns} -- 
the height 
}%end item
\end{itemize}
}%end item
\divideents{getWidth}
\item{ 
\index{getWidth()}
{\bf  getWidth}\\
\begin{lstlisting}[frame=none]
public final double getWidth()\end{lstlisting} %end signature
\begin{itemize}
\item{
{\bf  Description}

Gets the width of this tank
}
\item{{\bf  Returns} -- 
the width 
}%end item
\end{itemize}
}%end item
\divideents{setHeight}
\item{ 
\index{setHeight(double)}
{\bf  setHeight}\\
\begin{lstlisting}[frame=none]
public final void setHeight(double height)\end{lstlisting} %end signature
\begin{itemize}
\item{
{\bf  Description}

Sets the height of this tank
}
\item{
{\bf  Parameters}
  \begin{itemize}
   \item{
\texttt{height} -- The height of the tank}
  \end{itemize}
}%end item
\end{itemize}
}%end item
\divideents{setWidth}
\item{ 
\index{setWidth(double)}
{\bf  setWidth}\\
\begin{lstlisting}[frame=none]
public final void setWidth(double width)\end{lstlisting} %end signature
\begin{itemize}
\item{
{\bf  Description}

Sets the width of this tank
}
\item{
{\bf  Parameters}
  \begin{itemize}
   \item{
\texttt{width} -- The width of the tank}
  \end{itemize}
}%end item
\end{itemize}
}%end item
\end{itemize}
}
}
\section{\label{edu.kit.pse.osip.simulation.view.main.FillSensorDrawer}\index{FillSensorDrawer}Class FillSensorDrawer}{
\rule[1em]{\hsize}{4pt}\vskip -1em
\vskip .1in 
This class visualizes a fill level sensor.\vskip .1in 
\subsection{Declaration}{
\begin{lstlisting}[frame=none]
public class FillSensorDrawer
 extends edu.kit.pse.osip.simulation.view.main.ObjectDrawer\end{lstlisting}
\subsection{Constructors}{
\rule[1em]{\hsize}{2pt}\vskip -2em
\vskip -2em
\begin{itemize}
\item{ 
\index{FillSensorDrawer(Point2D)}
{\bf  FillSensorDrawer}\\
\begin{lstlisting}[frame=none]
public FillSensorDrawer(Point2D pos)\end{lstlisting} %end signature
\begin{itemize}
\item{
{\bf  Description}

Generates a new drawer for fill sensors
}
\item{
{\bf  Parameters}
  \begin{itemize}
   \item{
\texttt{pos} -- The center of the drawer}
  \end{itemize}
}%end item
\end{itemize}
}%end item
\end{itemize}
}
\subsection{Methods}{
\rule[1em]{\hsize}{2pt}\vskip -2em
\vskip -2em
\begin{itemize}
\item{ 
\index{draw(GraphicsContext)}
{\bf  draw}\\
\begin{lstlisting}[frame=none]
public final void draw(GraphicsContext context)\end{lstlisting} %end signature
\begin{itemize}
\item{
{\bf  Description}

The Drawer draws itself onto the GraphicsContext at its position.
}
\item{
{\bf  Parameters}
  \begin{itemize}
   \item{
\texttt{context} -- The context that the object draws itself onto}
  \end{itemize}
}%end item
\end{itemize}
}%end item
\end{itemize}
}
}
\section{\label{edu.kit.pse.osip.simulation.view.main.MixTankDrawer}\index{MixTankDrawer}Class MixTankDrawer}{
\rule[1em]{\hsize}{4pt}\vskip -1em
\vskip .1in 
The class visualizes a tank that is connected to several inputs. Due to this the fill level as well as the color of the contained liquid might change with time.\vskip .1in 
\subsection{Declaration}{
\begin{lstlisting}[frame=none]
public class MixTankDrawer
 extends edu.kit.pse.osip.simulation.view.main.AbstractTankDrawer\end{lstlisting}
\subsection{Constructors}{
\rule[1em]{\hsize}{2pt}\vskip -2em
\vskip -2em
\begin{itemize}
\item{ 
\index{MixTankDrawer(Point2D, MixTank)}
{\bf  MixTankDrawer}\\
\begin{lstlisting}[frame=none]
public MixTankDrawer(Point2D pos,edu.kit.pse.osip.core.model.base.MixTank tank)\end{lstlisting} %end signature
\begin{itemize}
\item{
{\bf  Description}

Creates a new MixTankDrawer
}
\item{
{\bf  Parameters}
  \begin{itemize}
   \item{
\texttt{pos} -- The upper left corner of the tank}
   \item{
\texttt{tank} -- The tank to get the attributes from}
  \end{itemize}
}%end item
\end{itemize}
}%end item
\end{itemize}
}
\subsection{Methods}{
\rule[1em]{\hsize}{2pt}\vskip -2em
\vskip -2em
\begin{itemize}
\item{ 
\index{drawSensors(GraphicsContext)}
{\bf  drawSensors}\\
\begin{lstlisting}[frame=none]
public final void drawSensors(GraphicsContext context)\end{lstlisting} %end signature
\begin{itemize}
\item{
{\bf  Description}

Add temperature- and fillsensor as well as a motor visualization to the tank.
}
\item{
{\bf  Parameters}
  \begin{itemize}
   \item{
\texttt{context} -- }
  \end{itemize}
}%end item
\end{itemize}
}%end item
\end{itemize}
}
}
\section{\label{edu.kit.pse.osip.simulation.view.main.MotorDrawer}\index{MotorDrawer}Class MotorDrawer}{
\rule[1em]{\hsize}{4pt}\vskip -1em
\vskip .1in 
This class visualizes the mixing motor in the MixTank.\vskip .1in 
\subsection{Declaration}{
\begin{lstlisting}[frame=none]
public class MotorDrawer
 extends edu.kit.pse.osip.simulation.view.main.RotatingControlDrawer\end{lstlisting}
\subsection{Constructors}{
\rule[1em]{\hsize}{2pt}\vskip -2em
\vskip -2em
\begin{itemize}
\item{ 
\index{MotorDrawer(Point2D, Motor)}
{\bf  MotorDrawer}\\
\begin{lstlisting}[frame=none]
public MotorDrawer(Point2D pos,edu.kit.pse.osip.core.model.base.Motor motor)\end{lstlisting} %end signature
\begin{itemize}
\item{
{\bf  Description}

Generates a new drawer object for motors
}
\item{
{\bf  Parameters}
  \begin{itemize}
   \item{
\texttt{pos} -- The center of the drawer}
   \item{
\texttt{motor} -- The motor to draw}
  \end{itemize}
}%end item
\end{itemize}
}%end item
\end{itemize}
}
\subsection{Methods}{
\rule[1em]{\hsize}{2pt}\vskip -2em
\vskip -2em
\begin{itemize}
\item{ 
\index{draw(GraphicsContext)}
{\bf  draw}\\
\begin{lstlisting}[frame=none]
public final void draw(GraphicsContext context)\end{lstlisting} %end signature
\begin{itemize}
\item{
{\bf  Description}

The Drawer draws itself onto the GraphicsContext at its position.
}
\item{
{\bf  Parameters}
  \begin{itemize}
   \item{
\texttt{context} -- The context that the object draws itself onto}
  \end{itemize}
}%end item
\end{itemize}
}%end item
\end{itemize}
}
}
\section{\label{edu.kit.pse.osip.simulation.view.main.ObjectDrawer}\index{ObjectDrawer}Class ObjectDrawer}{
\rule[1em]{\hsize}{4pt}\vskip -1em
\vskip .1in 
The parent class for all non-moving objects of the simulation. The only data it needs is a position where to draw itself.\vskip .1in 
\subsection{Declaration}{
\begin{lstlisting}[frame=none]
public abstract class ObjectDrawer
 extends java.lang.Object implements Drawer\end{lstlisting}
\subsection{All known subclasses}{FillSensorDrawer\small{\refdefined{edu.kit.pse.osip.simulation.view.main.FillSensorDrawer}}, RotatingControlDrawer\small{\refdefined{edu.kit.pse.osip.simulation.view.main.RotatingControlDrawer}}, TankDrawer\small{\refdefined{edu.kit.pse.osip.simulation.view.main.TankDrawer}}, MotorDrawer\small{\refdefined{edu.kit.pse.osip.simulation.view.main.MotorDrawer}}, MixTankDrawer\small{\refdefined{edu.kit.pse.osip.simulation.view.main.MixTankDrawer}}, AbstractTankDrawer\small{\refdefined{edu.kit.pse.osip.simulation.view.main.AbstractTankDrawer}}, TemperatureSensorDrawer\small{\refdefined{edu.kit.pse.osip.simulation.view.main.TemperatureSensorDrawer}}, ValveDrawer\small{\refdefined{edu.kit.pse.osip.simulation.view.main.ValveDrawer}}}
\subsection{Fields}{
\rule[1em]{\hsize}{2pt}
\begin{itemize}
\item{
\index{position}
\label{edu.kit.pse.osip.simulation.view.main.ObjectDrawer.position}\texttt{public Point2D\ {\bf  position}}
}
\end{itemize}
}
\subsection{Constructors}{
\rule[1em]{\hsize}{2pt}\vskip -2em
\vskip -2em
\begin{itemize}
\item{ 
\index{ObjectDrawer(Point2D)}
{\bf  ObjectDrawer}\\
\begin{lstlisting}[frame=none]
public ObjectDrawer(Point2D pos)\end{lstlisting} %end signature
\begin{itemize}
\item{
{\bf  Description}

Initiates the ObjectDrawer, setting its position to pos.
}
\item{
{\bf  Parameters}
  \begin{itemize}
   \item{
\texttt{pos} -- }
  \end{itemize}
}%end item
\end{itemize}
}%end item
\end{itemize}
}
\subsection{Methods}{
\rule[1em]{\hsize}{2pt}\vskip -2em
\vskip -2em
\begin{itemize}
\item{ 
\index{draw(GraphicsContext)}
{\bf  draw}\\
\begin{lstlisting}[frame=none]
public abstract void draw(GraphicsContext context)\end{lstlisting} %end signature
\begin{itemize}
\item{
{\bf  Description}

The Drawer draws itself onto the GraphicsContext at its position.
}
\item{
{\bf  Parameters}
  \begin{itemize}
   \item{
\texttt{context} -- The context that the object draws itself onto}
  \end{itemize}
}%end item
\end{itemize}
}%end item
\divideents{getPosition}
\item{ 
\index{getPosition()}
{\bf  getPosition}\\
\begin{lstlisting}[frame=none]
public final Point2D getPosition()\end{lstlisting} %end signature
\begin{itemize}
\item{
{\bf  Description}

Returns the position of this element
}
\item{{\bf  Returns} -- 
The position 
}%end item
\end{itemize}
}%end item
\divideents{setPosition}
\item{ 
\index{setPosition(Point2D)}
{\bf  setPosition}\\
\begin{lstlisting}[frame=none]
public final void setPosition(Point2D position)\end{lstlisting} %end signature
\begin{itemize}
\item{
{\bf  Description}

Sets the position of this element
}
\item{
{\bf  Parameters}
  \begin{itemize}
   \item{
\texttt{position} -- The upper left corner}
  \end{itemize}
}%end item
\end{itemize}
}%end item
\end{itemize}
}
}
\section{\label{edu.kit.pse.osip.simulation.view.main.PipeDrawer}\index{PipeDrawer}Class PipeDrawer}{
\rule[1em]{\hsize}{4pt}\vskip -1em
\vskip .1in 
This class visualizes a pipe connecting two tanks. It is specified by the waypoints during which the pipe leads. It needs at least two waypoints to exist.\vskip .1in 
\subsection{Declaration}{
\begin{lstlisting}[frame=none]
public class PipeDrawer
 extends java.lang.Object implements Drawer\end{lstlisting}
\subsection{Fields}{
\rule[1em]{\hsize}{2pt}
\begin{itemize}
\item{
\index{position}
\label{edu.kit.pse.osip.simulation.view.main.PipeDrawer.position}\texttt{public Point2D\ {\bf  position}}
}
\end{itemize}
}
\subsection{Constructors}{
\rule[1em]{\hsize}{2pt}\vskip -2em
\vskip -2em
\begin{itemize}
\item{ 
\index{PipeDrawer(Point2D\lbrack \rbrack )}
{\bf  PipeDrawer}\\
\begin{lstlisting}[frame=none]
public PipeDrawer(Point2D[] waypoints)\end{lstlisting} %end signature
\begin{itemize}
\item{
{\bf  Description}

Create a new pipe along the waypoints 1 to n.
}
\item{
{\bf  Parameters}
  \begin{itemize}
   \item{
\texttt{waypoints} -- The points that the pipe goes along}
  \end{itemize}
}%end item
\end{itemize}
}%end item
\end{itemize}
}
\subsection{Methods}{
\rule[1em]{\hsize}{2pt}\vskip -2em
\vskip -2em
\begin{itemize}
\item{ 
\index{draw(GraphicsContext)}
{\bf  draw}\\
\begin{lstlisting}[frame=none]
public final void draw(GraphicsContext context)\end{lstlisting} %end signature
\begin{itemize}
\item{
{\bf  Description}

The Drawer draws itself onto the GraphicsContext at its position.
}
\item{
{\bf  Parameters}
  \begin{itemize}
   \item{
\texttt{context} -- The context that the object draws itself onto}
  \end{itemize}
}%end item
\end{itemize}
}%end item
\end{itemize}
}
}
\section{\label{edu.kit.pse.osip.simulation.view.main.Point2D}\index{Point2D}Class Point2D}{
\rule[1em]{\hsize}{4pt}\vskip -1em
\vskip .1in 
This class represents a simple Point in 2D space. It is a simple wrapper for the data and does not support arithmetic operations on points.\vskip .1in 
\subsection{Declaration}{
\begin{lstlisting}[frame=none]
public class Point2D
 extends java.lang.Object\end{lstlisting}
\subsection{Constructors}{
\rule[1em]{\hsize}{2pt}\vskip -2em
\vskip -2em
\begin{itemize}
\item{ 
\index{Point2D(double, double)}
{\bf  Point2D}\\
\begin{lstlisting}[frame=none]
public Point2D(double x,double y)\end{lstlisting} %end signature
\begin{itemize}
\item{
{\bf  Description}

Creates a new point on a 2D surface
}
\item{
{\bf  Parameters}
  \begin{itemize}
   \item{
\texttt{x} -- }
   \item{
\texttt{y} -- }
  \end{itemize}
}%end item
\end{itemize}
}%end item
\end{itemize}
}
\subsection{Methods}{
\rule[1em]{\hsize}{2pt}\vskip -2em
\vskip -2em
\begin{itemize}
\item{ 
\index{getX()}
{\bf  getX}\\
\begin{lstlisting}[frame=none]
public final double getX()\end{lstlisting} %end signature
\begin{itemize}
\item{
{\bf  Description}

Gets the value of x.
}
\item{{\bf  Returns} -- 
The value of x. 
}%end item
\end{itemize}
}%end item
\divideents{getY}
\item{ 
\index{getY()}
{\bf  getY}\\
\begin{lstlisting}[frame=none]
public final double getY()\end{lstlisting} %end signature
\begin{itemize}
\item{
{\bf  Description}

Gets the value of y.
}
\item{{\bf  Returns} -- 
The value of y. 
}%end item
\end{itemize}
}%end item
\divideents{setX}
\item{ 
\index{setX(double)}
{\bf  setX}\\
\begin{lstlisting}[frame=none]
public final void setX(double x)\end{lstlisting} %end signature
\begin{itemize}
\item{
{\bf  Description}

Sets the value of x.
}
\item{
{\bf  Parameters}
  \begin{itemize}
   \item{
\texttt{x} -- The x coordinate}
  \end{itemize}
}%end item
\end{itemize}
}%end item
\divideents{setY}
\item{ 
\index{setY(double)}
{\bf  setY}\\
\begin{lstlisting}[frame=none]
public final void setY(double y)\end{lstlisting} %end signature
\begin{itemize}
\item{
{\bf  Description}

Sets the value of y.
}
\item{
{\bf  Parameters}
  \begin{itemize}
   \item{
\texttt{y} -- The y coordinate}
  \end{itemize}
}%end item
\end{itemize}
}%end item
\end{itemize}
}
}
\section{\label{edu.kit.pse.osip.simulation.view.main.RotatingControlDrawer}\index{RotatingControlDrawer}Class RotatingControlDrawer}{
\rule[1em]{\hsize}{4pt}\vskip -1em
\vskip .1in 
The parent for all rotating gui elements. It needs a position and a rotational speed. It provides private methods for rotation of the shapes making up the visualization, depending on the given speed.\vskip .1in 
\subsection{Declaration}{
\begin{lstlisting}[frame=none]
public abstract class RotatingControlDrawer
 extends edu.kit.pse.osip.simulation.view.main.ObjectDrawer\end{lstlisting}
\subsection{All known subclasses}{MotorDrawer\small{\refdefined{edu.kit.pse.osip.simulation.view.main.MotorDrawer}}, ValveDrawer\small{\refdefined{edu.kit.pse.osip.simulation.view.main.ValveDrawer}}}
\subsection{Constructors}{
\rule[1em]{\hsize}{2pt}\vskip -2em
\vskip -2em
\begin{itemize}
\item{ 
\index{RotatingControlDrawer(Point2D, int)}
{\bf  RotatingControlDrawer}\\
\begin{lstlisting}[frame=none]
public RotatingControlDrawer(Point2D pos,int speed)\end{lstlisting} %end signature
\begin{itemize}
\item{
{\bf  Description}

Sets the position by using super(pos) and the rotation speed.
}
\item{
{\bf  Parameters}
  \begin{itemize}
   \item{
\texttt{pos} -- The upper left corner}
   \item{
\texttt{speed} -- The speed in rpm}
  \end{itemize}
}%end item
\end{itemize}
}%end item
\end{itemize}
}
}
\section{\label{edu.kit.pse.osip.simulation.view.main.SimulationMainWindow}\index{SimulationMainWindow}Class SimulationMainWindow}{
\rule[1em]{\hsize}{4pt}\vskip -1em
\vskip .1in 
The main window for visualizing the OSIP simulation. It regularly updates itself with current information from the model and posesses an update() method for alarms. If an overflow occurs in the model it is be displayed by an overlay.\vskip .1in 
\subsection{Declaration}{
\begin{lstlisting}[frame=none]
public class SimulationMainWindow
 extends java.lang.Object implements edu.kit.pse.osip.simulation.controller.SimulationViewInterface, java.util.Observer\end{lstlisting}
\subsection{Fields}{
\rule[1em]{\hsize}{2pt}
\begin{itemize}
\item{
\index{productionSite}
\label{edu.kit.pse.osip.simulation.view.main.SimulationMainWindow.productionSite}\texttt{public edu.kit.pse.osip.core.model.base.ProductionSite\ {\bf  productionSite}}
}
\item{
\index{element}
\label{edu.kit.pse.osip.simulation.view.main.SimulationMainWindow.element}\texttt{public Drawer\ {\bf  element}}
}
\end{itemize}
}
\subsection{Constructors}{
\rule[1em]{\hsize}{2pt}\vskip -2em
\vskip -2em
\begin{itemize}
\item{ 
\index{SimulationMainWindow(ProductionSite)}
{\bf  SimulationMainWindow}\\
\begin{lstlisting}[frame=none]
public SimulationMainWindow(edu.kit.pse.osip.core.model.base.ProductionSite productionSite)\end{lstlisting} %end signature
\begin{itemize}
\item{
{\bf  Description}

Creates a new SimulationMainWindow
}
\item{
{\bf  Parameters}
  \begin{itemize}
   \item{
\texttt{productionSite} -- The ProductionSite object, so that the view can access the model}
  \end{itemize}
}%end item
\end{itemize}
}%end item
\end{itemize}
}
\subsection{Methods}{
\rule[1em]{\hsize}{2pt}\vskip -2em
\vskip -2em
\begin{itemize}
\item{ 
\index{setAboutButtonHandler(MenuAboutButtonHandler)}
{\bf  setAboutButtonHandler}\\
\begin{lstlisting}[frame=none]
public final void setAboutButtonHandler(edu.kit.pse.osip.monitoring.controller.MenuAboutButtonHandler aboutButtonHandler)\end{lstlisting} %end signature
\begin{itemize}
\item{
{\bf  Description}

Sets the handler for pressing the about entry in the menu
}
\item{
{\bf  Parameters}
  \begin{itemize}
   \item{
\texttt{aboutButtonHandler} -- The handler to be called when the about button is pressed}
  \end{itemize}
}%end item
\end{itemize}
}%end item
\divideents{setControlButtonHandler}
\item{ 
\index{setControlButtonHandler(MenuControlButtonHandler)}
{\bf  setControlButtonHandler}\\
\begin{lstlisting}[frame=none]
public final void setControlButtonHandler(edu.kit.pse.osip.simulation.controller.MenuControlButtonHandler controlButtonHandler)\end{lstlisting} %end signature
\begin{itemize}
\item{
{\bf  Description}

Sets the handler for pressing the control entry in the menu
}
\item{
{\bf  Parameters}
  \begin{itemize}
   \item{
\texttt{controlButtonHandler} -- The handler to execute}
  \end{itemize}
}%end item
\end{itemize}
}%end item
\divideents{setHelpButtonHandler}
\item{ 
\index{setHelpButtonHandler(MenuHelpButtonHandler)}
{\bf  setHelpButtonHandler}\\
\begin{lstlisting}[frame=none]
public final void setHelpButtonHandler(edu.kit.pse.osip.monitoring.controller.MenuHelpButtonHandler helpButtonHandler)\end{lstlisting} %end signature
\begin{itemize}
\item{
{\bf  Description}

Sets the handler for pressing the help entry in the menu
}
\item{
{\bf  Parameters}
  \begin{itemize}
   \item{
\texttt{helpButtonHandler} -- The handler to be called when the help button is pressed}
  \end{itemize}
}%end item
\end{itemize}
}%end item
\divideents{setSettingsButtonHandler}
\item{ 
\index{setSettingsButtonHandler(MenuSettingsButtonHandler)}
{\bf  setSettingsButtonHandler}\\
\begin{lstlisting}[frame=none]
public final void setSettingsButtonHandler(edu.kit.pse.osip.monitoring.controller.MenuSettingsButtonHandler settingsButtonHandler)\end{lstlisting} %end signature
\begin{itemize}
\item{
{\bf  Description}

Sets the handler for pressing the settings entry in the menu
}
\item{
{\bf  Parameters}
  \begin{itemize}
   \item{
\texttt{settingsButtonHandler} -- The handler to be called when the settings button is pressed}
  \end{itemize}
}%end item
\end{itemize}
}%end item
\divideents{showOverflow}
\item{ 
\index{showOverflow()}
{\bf  showOverflow}\\
\begin{lstlisting}[frame=none]
public final void showOverflow()\end{lstlisting} %end signature
\begin{itemize}
\item{
{\bf  Description}

The simulation is replaced by the OverflowOverlay.
}
\end{itemize}
}%end item
\divideents{start}
\item{ 
\index{start(javafx.stage.Stage)}
{\bf  start}\\
\begin{lstlisting}[frame=none]
public final void start(javafx.stage.Stage primaryStage)\end{lstlisting} %end signature
\begin{itemize}
\item{
{\bf  Description}

The stage that is provided by JavaFx
}
\item{
{\bf  Parameters}
  \begin{itemize}
   \item{
\texttt{primaryStage} -- The stage to draw the window on}
  \end{itemize}
}%end item
\end{itemize}
}%end item
\divideents{update}
\item{ 
\index{update(Observable, Object)}
{\bf  update}\\
\begin{lstlisting}[frame=none]
public final void update(java.util.Observable observable,java.lang.Object object)\end{lstlisting} %end signature
\begin{itemize}
\item{
{\bf  Description}

The observed object received an update
}
\item{
{\bf  Parameters}
  \begin{itemize}
   \item{
\texttt{observable} -- The observed object}
   \item{
\texttt{object} -- The new value}
  \end{itemize}
}%end item
\end{itemize}
}%end item
\end{itemize}
}
}
\section{\label{edu.kit.pse.osip.simulation.view.main.TankDrawer}\index{TankDrawer}Class TankDrawer}{
\rule[1em]{\hsize}{4pt}\vskip -1em
\vskip .1in 
The class visualizes the tanks during which liquid first enters the simulation. As they have only one input only the fill level is variable, the color remains fixed.\vskip .1in 
\subsection{Declaration}{
\begin{lstlisting}[frame=none]
public class TankDrawer
 extends edu.kit.pse.osip.simulation.view.main.AbstractTankDrawer\end{lstlisting}
\subsection{Constructors}{
\rule[1em]{\hsize}{2pt}\vskip -2em
\vskip -2em
\begin{itemize}
\item{ 
\index{TankDrawer(Point2D, Tank)}
{\bf  TankDrawer}\\
\begin{lstlisting}[frame=none]
public TankDrawer(Point2D pos,edu.kit.pse.osip.core.model.base.Tank tank)\end{lstlisting} %end signature
\begin{itemize}
\item{
{\bf  Description}

Creates a new tank drawer
}
\item{
{\bf  Parameters}
  \begin{itemize}
   \item{
\texttt{pos} -- The upper left corner of the tank}
   \item{
\texttt{tank} -- The tank to get the attributes from}
  \end{itemize}
}%end item
\end{itemize}
}%end item
\end{itemize}
}
\subsection{Methods}{
\rule[1em]{\hsize}{2pt}\vskip -2em
\vskip -2em
\begin{itemize}
\item{ 
\index{drawSensors(GraphicsContext)}
{\bf  drawSensors}\\
\begin{lstlisting}[frame=none]
public final void drawSensors(GraphicsContext context)\end{lstlisting} %end signature
\begin{itemize}
\item{
{\bf  Description}

Add temperature- and fillsensor visualizations to the tank.
}
\item{
{\bf  Parameters}
  \begin{itemize}
   \item{
\texttt{context} -- }
  \end{itemize}
}%end item
\end{itemize}
}%end item
\end{itemize}
}
}
\section{\label{edu.kit.pse.osip.simulation.view.main.TemperatureSensorDrawer}\index{TemperatureSensorDrawer}Class TemperatureSensorDrawer}{
\rule[1em]{\hsize}{4pt}\vskip -1em
\vskip .1in 
This class visualizes a temperature sensor.\vskip .1in 
\subsection{Declaration}{
\begin{lstlisting}[frame=none]
public class TemperatureSensorDrawer
 extends edu.kit.pse.osip.simulation.view.main.ObjectDrawer\end{lstlisting}
\subsection{Constructors}{
\rule[1em]{\hsize}{2pt}\vskip -2em
\vskip -2em
\begin{itemize}
\item{ 
\index{TemperatureSensorDrawer(Point2D)}
{\bf  TemperatureSensorDrawer}\\
\begin{lstlisting}[frame=none]
public TemperatureSensorDrawer(Point2D pos)\end{lstlisting} %end signature
\begin{itemize}
\item{
{\bf  Description}

Generates a new drawer for temperature sensors
}
\item{
{\bf  Parameters}
  \begin{itemize}
   \item{
\texttt{pos} -- The center of the drawer}
  \end{itemize}
}%end item
\end{itemize}
}%end item
\end{itemize}
}
\subsection{Methods}{
\rule[1em]{\hsize}{2pt}\vskip -2em
\vskip -2em
\begin{itemize}
\item{ 
\index{draw(GraphicsContext)}
{\bf  draw}\\
\begin{lstlisting}[frame=none]
public final void draw(GraphicsContext context)\end{lstlisting} %end signature
\begin{itemize}
\item{
{\bf  Description}

The Drawer draws itself onto the GraphicsContext at its position.
}
\item{
{\bf  Parameters}
  \begin{itemize}
   \item{
\texttt{context} -- The context that the object draws itself onto}
  \end{itemize}
}%end item
\end{itemize}
}%end item
\end{itemize}
}
}
\section{\label{edu.kit.pse.osip.simulation.view.main.ValveDrawer}\index{ValveDrawer}Class ValveDrawer}{
\rule[1em]{\hsize}{4pt}\vskip -1em
\vskip .1in 
This class visualizes the valve that is part of every pipe.\vskip .1in 
\subsection{Declaration}{
\begin{lstlisting}[frame=none]
public class ValveDrawer
 extends edu.kit.pse.osip.simulation.view.main.RotatingControlDrawer\end{lstlisting}
\subsection{Fields}{
\rule[1em]{\hsize}{2pt}
\begin{itemize}
\item{
\index{pipe}
\label{edu.kit.pse.osip.simulation.view.main.ValveDrawer.pipe}\texttt{public PipeDrawer\ {\bf  pipe}}
}
\end{itemize}
}
\subsection{Constructors}{
\rule[1em]{\hsize}{2pt}\vskip -2em
\vskip -2em
\begin{itemize}
\item{ 
\index{ValveDrawer(Point2D, Pipe)}
{\bf  ValveDrawer}\\
\begin{lstlisting}[frame=none]
public ValveDrawer(Point2D pos,edu.kit.pse.osip.core.model.base.Pipe pipe)\end{lstlisting} %end signature
\begin{itemize}
\item{
{\bf  Description}

Generates a new drawer object for valves
}
\item{
{\bf  Parameters}
  \begin{itemize}
   \item{
\texttt{pos} -- The center of the drawer}
   \item{
\texttt{pipe} -- The pipe to which the valve is attached}
  \end{itemize}
}%end item
\end{itemize}
}%end item
\end{itemize}
}
\subsection{Methods}{
\rule[1em]{\hsize}{2pt}\vskip -2em
\vskip -2em
\begin{itemize}
\item{ 
\index{draw(GraphicsContext)}
{\bf  draw}\\
\begin{lstlisting}[frame=none]
public final void draw(GraphicsContext context)\end{lstlisting} %end signature
\begin{itemize}
\item{
{\bf  Description}

The Drawer draws itself onto the GraphicsContext at its position.
}
\item{
{\bf  Parameters}
  \begin{itemize}
   \item{
\texttt{context} -- The context that the object draws itself onto}
  \end{itemize}
}%end item
\end{itemize}
}%end item
\end{itemize}
}
}
\section{\label{edu.kit.pse.osip.simulation.view.main.InvalidWaypointsException}\index{InvalidWaypointsException}Exception InvalidWaypointsException}{
\rule[1em]{\hsize}{4pt}\vskip -1em
\vskip .1in 
This exception signifies that a list of waypoints is not valid, e.g. if the list is empty.\vskip .1in 
\subsection{Declaration}{
\begin{lstlisting}[frame=none]
public class InvalidWaypointsException
 extends java.lang.IllegalArgumentException\end{lstlisting}
\subsection{Constructors}{
\rule[1em]{\hsize}{2pt}\vskip -2em
\vskip -2em
\begin{itemize}
\item{ 
\index{InvalidWaypointsException(Point2D\lbrack \rbrack )}
{\bf  InvalidWaypointsException}\\
\begin{lstlisting}[frame=none]
public InvalidWaypointsException(Point2D[] waypoints)\end{lstlisting} %end signature
\begin{itemize}
\item{
{\bf  Description}

Generates the invalid waypoint exception
}
\item{
{\bf  Parameters}
  \begin{itemize}
   \item{
\texttt{waypoints} -- The invalid waypoints}
  \end{itemize}
}%end item
\end{itemize}
}%end item
\end{itemize}
}
}
}
\chapter{Package edu.kit.pse.osip.simulation.view.settings}{
\label{edu.kit.pse.osip.simulation.view.settings}\section{\label{edu.kit.pse.osip.simulation.view.settings.PortTextField}\index{PortTextField}Class PortTextField}{
\rule[1em]{\hsize}{4pt}\vskip -1em
\vskip .1in 
This textfield is for the input of port numbers. It checks its own inputs and only allows saving them if they are valid.\vskip .1in 
\subsection{Declaration}{
\begin{lstlisting}[frame=none]
public class PortTextField
 extends javafx.scene.control.TextField\end{lstlisting}
\subsection{Constructors}{
\rule[1em]{\hsize}{2pt}\vskip -2em
\vskip -2em
\begin{itemize}
\item{ 
\index{PortTextField()}
{\bf  PortTextField}\\
\begin{lstlisting}[frame=none]
public PortTextField()\end{lstlisting} %end signature
}%end item
\end{itemize}
}
\subsection{Methods}{
\rule[1em]{\hsize}{2pt}\vskip -2em
\vskip -2em
\begin{itemize}
\item{ 
\index{getPort()}
{\bf  getPort}\\
\begin{lstlisting}[frame=none]
public final int getPort()\end{lstlisting} %end signature
\begin{itemize}
\item{
{\bf  Description}

Gets the port number in the textField.
}
\item{{\bf  Returns} -- 
The port number in the textField. 
}%end item
\item{{\bf  Throws}
  \begin{itemize}
   \item{\vskip -.6ex \texttt{edu.kit.pse.osip.core.utils.formatting.InvalidPortException} -- Thrown if the current value in port is not valid (see FormatChecker.parsePort(String port).}
  \end{itemize}
}%end item
\end{itemize}
}%end item
\end{itemize}
}
}
\section{\label{edu.kit.pse.osip.simulation.view.settings.SimulationSettingsWindow}\index{SimulationSettingsWindow}Class SimulationSettingsWindow}{
\rule[1em]{\hsize}{4pt}\vskip -1em
\vskip .1in 
This class is the main window of the OSIP simulation settings.\vskip .1in 
\subsection{Declaration}{
\begin{lstlisting}[frame=none]
public class SimulationSettingsWindow
 extends javafx.stage.Stage implements edu.kit.pse.osip.simulation.controller.SimulationSettingsWindow\end{lstlisting}
\subsection{Constructors}{
\rule[1em]{\hsize}{2pt}\vskip -2em
\vskip -2em
\begin{itemize}
\item{ 
\index{SimulationSettingsWindow(ServerSettingsWrapper, ProductionSite)}
{\bf  SimulationSettingsWindow}\\
\begin{lstlisting}[frame=none]
public SimulationSettingsWindow(edu.kit.pse.osip.core.io.files.ServerSettingsWrapper settings,edu.kit.pse.osip.core.model.base.ProductionSite productionSite)\end{lstlisting} %end signature
\begin{itemize}
\item{
{\bf  Description}

Generates a new window that shows the simulation settings
}
\item{
{\bf  Parameters}
  \begin{itemize}
   \item{
\texttt{settings} -- }
   \item{
\texttt{productionSite} -- The ProductionSite, to get all the tanks}
  \end{itemize}
}%end item
\end{itemize}
}%end item
\end{itemize}
}
\subsection{Methods}{
\rule[1em]{\hsize}{2pt}\vskip -2em
\vskip -2em
\begin{itemize}
\item{ 
\index{getJitter()}
{\bf  getJitter}\\
\begin{lstlisting}[frame=none]
public final int getJitter()\end{lstlisting} %end signature
\begin{itemize}
\item{
{\bf  Description}

Gets he current value of the jitter-scrollbar.
}
\item{{\bf  Returns} -- 
The current value of the jitter-scrollbar. 
}%end item
\end{itemize}
}%end item
\divideents{getPort}
\item{ 
\index{getPort(TankSelector)}
{\bf  getPort}\\
\begin{lstlisting}[frame=none]
public final int getPort(edu.kit.pse.osip.core.model.base.TankSelector tank)\end{lstlisting} %end signature
\begin{itemize}
\item{
{\bf  Description}

Gets the port number in Porttextfield id.
}
\item{
{\bf  Parameters}
  \begin{itemize}
   \item{
\texttt{tank} -- }
  \end{itemize}
}%end item
\item{{\bf  Returns} -- 
The port number in Porttextfield id. 
}%end item
\item{{\bf  Throws}
  \begin{itemize}
   \item{\vskip -.6ex \texttt{edu.kit.pse.osip.core.utils.formatting.InvalidPortException} -- Thrown if the current value in port is not valid (see FormatChecker.parsePort(String port).}
  \end{itemize}
}%end item
\end{itemize}
}%end item
\divideents{setSettingsChangedListener}
\item{ 
\index{setSettingsChangedListener(SettingsChangedListener)}
{\bf  setSettingsChangedListener}\\
\begin{lstlisting}[frame=none]
public final void setSettingsChangedListener(edu.kit.pse.osip.simulation.controller.SettingsChangedListener listener)\end{lstlisting} %end signature
\begin{itemize}
\item{
{\bf  Description}

Sets the listener that gets notified as soon as the settings change
}
\item{
{\bf  Parameters}
  \begin{itemize}
   \item{
\texttt{listener} -- }
  \end{itemize}
}%end item
\end{itemize}
}%end item
\end{itemize}
}
}
}
\chapter{Package edu.kit.pse.osip.simulation.view.dialogs}{
\label{edu.kit.pse.osip.simulation.view.dialogs}\section{\label{edu.kit.pse.osip.simulation.view.dialogs.AboutDialog}\index{AboutDialog}Class AboutDialog}{
\rule[1em]{\hsize}{4pt}\vskip -1em
\vskip .1in 
This window shows information about the creators of OSIP.\vskip .1in 
\subsection{Declaration}{
\begin{lstlisting}[frame=none]
public class AboutDialog
 extends javafx.stage.Stage\end{lstlisting}
\subsection{Constructors}{
\rule[1em]{\hsize}{2pt}\vskip -2em
\vskip -2em
\begin{itemize}
\item{ 
\index{AboutDialog()}
{\bf  AboutDialog}\\
\begin{lstlisting}[frame=none]
public AboutDialog()\end{lstlisting} %end signature
}%end item
\end{itemize}
}
}
\section{\label{edu.kit.pse.osip.simulation.view.dialogs.HelpDialog}\index{HelpDialog}Class HelpDialog}{
\rule[1em]{\hsize}{4pt}\vskip -1em
\vskip .1in 
This window shows a short description of OSIP and how to use it.\vskip .1in 
\subsection{Declaration}{
\begin{lstlisting}[frame=none]
public class HelpDialog
 extends javafx.stage.Stage\end{lstlisting}
\subsection{Constructors}{
\rule[1em]{\hsize}{2pt}\vskip -2em
\vskip -2em
\begin{itemize}
\item{ 
\index{HelpDialog()}
{\bf  HelpDialog}\\
\begin{lstlisting}[frame=none]
public HelpDialog()\end{lstlisting} %end signature
}%end item
\end{itemize}
}
}
\section{\label{edu.kit.pse.osip.simulation.view.dialogs.OverflowDialog}\index{OverflowDialog}Class OverflowDialog}{
\rule[1em]{\hsize}{4pt}\vskip -1em
\vskip .1in 
This window informs the user, that a tank has an overflow.\vskip .1in 
\subsection{Declaration}{
\begin{lstlisting}[frame=none]
public class OverflowDialog
 extends java.lang.Object\end{lstlisting}
\subsection{Constructors}{
\rule[1em]{\hsize}{2pt}\vskip -2em
\vskip -2em
\begin{itemize}
\item{ 
\index{OverflowDialog()}
{\bf  OverflowDialog}\\
\begin{lstlisting}[frame=none]
public OverflowDialog()\end{lstlisting} %end signature
}%end item
\end{itemize}
}
\subsection{Methods}{
\rule[1em]{\hsize}{2pt}\vskip -2em
\vskip -2em
\begin{itemize}
\item{ 
\index{setTank(TankSelector)}
{\bf  setTank}\\
\begin{lstlisting}[frame=none]
public final void setTank(edu.kit.pse.osip.core.model.base.TankSelector tank)\end{lstlisting} %end signature
\begin{itemize}
\item{
{\bf  Description}

Set the tank which has an overflow
}
\item{
{\bf  Parameters}
  \begin{itemize}
   \item{
\texttt{tank} -- The tank which has an overflow}
  \end{itemize}
}%end item
\end{itemize}
}%end item
\end{itemize}
}
}
}
\printindex
\end{document}
