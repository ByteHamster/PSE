%!TEX root = Pflichtenheft.tex

\newglossaryentry{Dockerimage}
{
  name=Dockerimage,
  plural=Dockerimages,
  description={Ein Container mit Programmen, der unabhängig vom zugrunde liegenden Betriebssystem gleichbleibende Bedingungen herstellt}
}

\newglossaryentry{Fertigungssimulation}
{
  name=Fertigungssimulation,
  plural=Fertigungssimulationen,
  description={Darstellung der Industrieanlage und Simulieren des Verhaltens}
}

\newglossaryentry{GUI}
{
  name=GUI,
  plural=GUIs,
  description={Die grafische Benutzerumgebung des Programms, welche vom Bediener gesehen wird}
}

\newglossaryentry{Industrial Data Space}
{
  name=Industrial Data Space,
  plural=Industrial Data Spaces,
  description={Eine Infrastruktur zum Austausch von Daten in der Industrie}
}

\newglossaryentry{IP-Adresse}
{
  name=IP-Adresse,
  plural=IP-Adressen,
  description={Eine Adresse in Computernetzen. Sie basiert auf dem Internetprotokoll}
}

\newglossaryentry{Java Property Datei}
{
  name=Java Property Datei,
  plural=Java Property Dateien,
  description={Textdatei, mit der Java Programme konfiguriert werden können}
}

\newglossaryentry{Jitter}
{
  name=Jitter,
  description={Simulation von variierenden Werten, um näher an echten Anlagen zu liegen}
}

\newglossaryentry{Milo}
{
  name=Milo,
  description={Eine quelloffene Implementierung des \gls{OPC UA} Protokolls in Java. Milo wird von der Eclipse Foundation gest\"utzt. Code unter https://github.com/eclipse/milo}
}

\newglossaryentry{OPC UA}
{
  name=OPC UA,
  plural=OPC UA,
  description={Ein Protokoll zur Übertragung von Daten und Steuersignalen von Industrieanlagen}
}

\newglossaryentry{OPC UA Client}
{
  name=OPC UA Client,
  plural=OPC UA Clients,
  description={Fordert bei einem \gls{OPC UA Server} Daten an und empf\"angt von diesem Alarme. Kommuniziert per \gls{OPC UA}}
}

\newglossaryentry{OPC UA Server}
{
  name=OPC UA Server,
  plural=OPC UA Server,
  description={Modelliert eine Menge aus Sensoren einer \gls{Produktionsanlage}. Sendet per \gls{OPC UA} Alarme an \glspl{OPC UA Client}. Antwortet \glspl{OPC UA Client} auf Anfragen mit \glspl{Sensordatum}}
}

\newglossaryentry{Produktionsanlage}
{
  name=Produktionsanlage,
  plural=Produktionsanlagen,
  description={Großmaschinen, wie sie in der Industrie eingesetzt werden. Hier symbolisiert durch Tanks mit Flüssigkeiten}
}

\newglossaryentry{Prozessvariable}
{
  name=Prozessvariable,
  plural=Prozessvariablen,
  description={Ein Wert, den der Benutzer in der \gls{Fertigungssimulation} einstellen kann, z.B. Zu- oder Abflussmengen, Temperaturen oder Motordrehzahlen}
}

\newglossaryentry{Sensordatum}
{
  name=Sensordatum,
  plural=Sensordaten,
  description={Ausgabewert eines simulierten Sensors wie Temperatur oder Durchflussmenge}
}

\newglossaryentry{Simulations-Szenario}
{
  name=Simulations-Szenario,
  plural=Simulations-Szenarien,
  description={Automatisierte Veränderung von Eigenschaften der Fertigungssimulation}
}

\newglossaryentry{Systemadapter}
{
  name=Systemadapter,
  plural=Systemadapter,
  description={Konvertiert die Daten eines Gesamtsystems (hier Industrieanlage) in ein anderes Format}
}

\newglossaryentry{TCP/IP}
{
  name=TCP/IP,
  plural=TCP/IP,
  description={Ein Protokoll zum Datenaustausch über das Internet}
}

\newglossaryentry{Uberwachungskonsole}
{
  name=Überwachungskonsole,
  plural=Überwachungskonsolen,
  description={Anzeige der Sensorwerte der Fertigungssimulation}
}
