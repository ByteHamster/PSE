%!TEX root = Pflichtenheft.tex

\newglossaryentry{Industrial Data Space}
{
  name=Industrial Data Space,
  plural=Industrial Data Spaces,
  description={Eine Infrastruktur zum Austausch von Daten in der Industrie}
}

\newglossaryentry{Systemadapter}
{
  name=Systemadapter,
  plural=Systemadapter,
  description={Konvertiert die Daten eines Gesamtsystems (hier Industrieanlage) in ein anderes Format}
}

\newglossaryentry{OPC UA}
{
  name=OPC UA,
  plural=OPC UA,
  description={Ein Protokoll zur Übertragung von Daten und Steuersignalen von Industrieanlagen}
}

\newglossaryentry{Produktionsanlage}
{
  name=Produktionsanlage,
  plural=Produktionsanlagen,
  description={Großmaschinen, wie sie in der Industrie eingesetzt werden. Hier symbolisiert durch Tanks mit Flüssigkeiten}
}

\newglossaryentry{TCP/IP Verbindung}
{
  name=TCP/IP Verbindung,
  plural=TCP/IP Verbindungen,
  description={Ein Protokoll zum Datenaustausch über das Internet}
}

\newglossaryentry{Uberwachungskonsole}
{
  name=Überwachungskonsole,
  plural=Überwachungskonsolen,
  description={Anzeige der Sensorwerte der Fertigungssimulation}
}

\newglossaryentry{Fertigungssimulation}
{
  name=Fertigungssimulation,
  plural=Fertigungssimulationen,
  description={Darstellung der Industrieanlage und Simulieren des Verhaltens}
}

\newglossaryentry{Dockerimage}
{
  name=Dockerimage,
  plural=Dockerimages,
  description={Ein Container mit Programmen, der unabhängig vom zugrunde liegenden Betriebssystem gleichbleibende Bedingungen herstellt}
}

\newglossaryentry{GUI}
{
  name=GUI,
  plural=GUIs,
  description={Die grafische Benutzerumgebung des Programms, welche vom Bediener gesehen wird}
}

\newglossaryentry{Sensordatum}
{
  name=Sensordatum,
  plural=Sensordaten,
  description={Ausgabewert eines simulierten Sensors wie Temperatur oder Durchflussmenge}
}

\newglossaryentry{Jitter}
{
  name=Jitter,
  description={Simulation von variierenden Werten, um näher an echten Anlagen zu liegen}
}

\newglossaryentry{Makro}
{
  name=Makro,
  plural=Makros,
  description={Eine aufgezeichnete Folge von Einstellungen, die gesetzt werden. Wird benutzt, um einen komplexeren Ablauf in der Anlage zu simulieren}
}
